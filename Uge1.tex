\documentclass{Class}
\usepackage[utf8]{inputenc}
\usepackage{graphicx}
\usepackage{listings}
\usepackage{color}
\usepackage{float}
\usepackage{amsmath} 
\usepackage{hyperref}
% \usepackage{calrsfs}
\usepackage{mathrsfs}
\usepackage{aligned-overset}
\usepackage{amssymb}
\usepackage{mathtools}
\usepackage[mathscr]{euscript}
\usepackage{tikz}
\usepackage{bbm}
\usepackage{booktabs}
\newcommand{\N}{\mathbb{N}}
\newcommand{\Q}{\mathbb{Q}}
\newcommand{\R}{\mathbb{R}}
\newcommand{\C}{\mathbb{C}}
\newcommand{\F}{\mathbb{F}}
\newcommand{\E}{\mathbb{E}}
\newcommand{\1}{\mathbbm{1}}
\newcommand{\Prob}{\mathbb{P}}
\usepackage{geometry}
    \geometry{
        a4paper,
        left=3.5cm,
        right=3.5cm   ,
    }
\definecolor{dkgreen}{rgb}{0,0.6,0}
\definecolor{gray}{rgb}{0.5,0.5,0.5}
\definecolor{mauve}{rgb}{0.58,0,0.82}

\lstset{frame=tb,
  language=Python,
  aboveskip=3mm,
  belowskip=3mm,
  showstringspaces=false,
  columns=flexible,
  basicstyle={\small\ttfamily},
  numbers=none,
  numberstyle=\tiny\color{gray},
  keywordstyle=\color{blue},
  commentstyle=\color{dkgreen},
  stringstyle=\color{mauve},
  breaklines=true,
  breakatwhitespace=true,
  tabsize=3
}
\title{Uge 1}
\author{Markus Alexander Hohwü Larsen}
\usepackage{graphicx} % Required for inserting images
\renewcommand{\thesubsection}{\thesection.\alph{subsection}}
\begin{document}
\problem{13.7} 
In this problem we consider the Gaussian (or normal) distribution $N\left(\xi, \sigma^2\right)$, where $\xi \in \mathbb{R}$ and $\sigma^2>0$, i.e. the probability measure on $(\mathbb{R}, \mathcal{B}(\mathbb{R}))$ with density
    $$
    f_{\varepsilon, \sigma^2}(x)=\frac{1}{\sqrt{2 \pi \sigma^2}} \exp \left(-\frac{1}{2 \sigma^2}(x-\xi)^2\right), \quad(x \in \mathbb{R})
    $$
    with respect to the Lebesgue measure $\lambda$. We start by focusing on the case where $\xi=0$, and $\sigma^2=1$, and we consider accordingly a random variable $X$ on a probability space $(\Omega, \mathcal{F}, P)$, which is Gaussian distributed with parameters $(0,1)$.
\begin{enumerate}
    \item Use Dominated Convergence and integration by parts to verify the identity:

    $$
    \frac{1}{\sqrt{2 \pi}} \int_{\mathbb{R}} x^p \exp \left(-\frac{1}{2} x^2\right) \lambda(\mathrm{d} x)=\frac{p-1}{\sqrt{2 \pi}} \int_{\mathbb{R}} x^{p-2} \exp \left(-\frac{1}{2} x^2\right) \lambda(\mathrm{d} x)
    $$
    
    for all $p$ in $\N$, such that $p \geq 2$.

    \item Use an induction argument to verify that for any $p$ in $\mathbb{N}$, it holds that
        $$\mathbb{E}\left[X^p\right]= \begin{cases}(p-1)(p-3) \cdots 1, & \text { if } p \text { is even } \\ 0, & \text { if } p \text { is odd }\end{cases}$$
    \item Show that if $Y$ is a random variable on $(\Omega, \mathcal{F}, P)$, which is Gaussian distributed with parameters $\left(\xi, \sigma^2\right)$, then the random variable $\left(\sigma^2\right)^{-1 / 2}(Y-\xi)$ is Gaussian distributed with parameters $(0,1)$.
    \item Assume that Y is a random variable on $(\Omega, \mathcal{F}, P)$, which is Gaussian distributed with parameters $\left(\xi, \sigma^2\right)$. Argue then that $Y$ has finite $p^{\prime}$ th moment for any $p$ in $\mathbb{N}$, and calculate $\mathbb{E}[Y], \operatorname{Var}[Y]$ and the moments $\mathbb{E}\left[Y^3\right]$ and $\mathbb{E}\left[Y^4\right]$.
\end{enumerate}
\solution
\part Since $\exp$ is a positive operation, we use monotone convergence to rewrite the integral:
$$
    \frac{1}{\sqrt{2 \pi}} \int_{\mathbb{R}} x^p \exp \left(-\frac{1}{2} x^2\right) \lambda(\mathrm{d} x)=\frac{1}{\sqrt{2 \pi}} \lim_{n \rightarrow \infty }\int_{-n}^{n} x^{p} \exp \left(-\frac{1}{2} x^2\right) \lambda(\mathrm{d} x)
    $$
We now use integration by parts:
$$\int u dv = uv - \int v du$$
We identify:
$$u = x^{p-1} \Rightarrow du = (p-1)x^{p-2}$$
$$dv = xe^{-\frac{1}{2}x^2} \Rightarrow v = -e^{-\frac{1}{2}x^2} $$
So we insert this into the formula:
$$=\frac{1}{\sqrt{2\pi}}\lim_{n\rightarrow \infty}\left(\left[-x^{p-1}e^{-\frac{1}{2}x^2}\right]_{-n}^{n}-\int_{-n}^n-(p-1)x^{p-2}e^{-\frac{1}{2}x^2}\right)$$
$$=\frac{1}{\sqrt{2\pi}}\lim_{n\rightarrow \infty}\left(\left[-x^{p-1}e^{-\frac{1}{2}x^2}\right]_{-n}^{n}+\int_{-n}^n(p-1)x^{p-2}e^{-\frac{1}{2}x^2}\right)$$
$$=\frac{1}{\sqrt{2\pi}}\lim_{n\rightarrow \infty}\left(\int_{-n}^n(p-1)x^{p-2}e^{-\frac{1}{2}x^2}\right)$$
$$=\frac{p-1}{\sqrt{2 \pi}} \int_{\mathbb{R}} x^{p-2} \exp \left(-\frac{1}{2} x^2\right) \lambda(\mathrm{d} x)$$
\part
\textbf{Case 1: $p=1$}
\\We use our result in a) to rewrite the expectation of the $p'th$ moment to:
$$
\mathbb{E}[X^1]=\frac{1}{\sqrt{2 \pi}} \int_{\mathbb{R}} x \exp \left(-\frac{1}{2} x^2\right) \lambda(\mathrm{d} x)=0
$$
Or just because $X$ has mean 0. 
\\\textbf{Case 2: $p=2$}
We use our result in a):
$$
\mathbb{E}[X^2]=\frac{1}{\sqrt{2 \pi}} \int_{\mathbb{R}} x^{0} \exp \left(-\frac{1}{2} x^2\right) \lambda(\mathrm{d} x)
$$
$$= \frac{1}{\sqrt{2 \pi}}\int_{\mathbb{R}}\exp \left(-\frac{1}{2} x^2\right) \lambda(\mathrm{d} x)$$
We notice that this is just the density of a $N(0,1)$ distribution, and we know that it integrates to 1. Therefore:
$$\mathbb{E}[X^2]=1$$
\\\textbf{Inductive step:}
Assume that the formula holds for some $p=k$. We now need to show that it holds for $k+1$. 
\\\textbf{Case 1: $k$ is even}
If $k$ is even we have:
$$\mathbb{E}[X^k]=(k-1)(k-3)\cdots 1$$
And we must show that $\mathbb{E}[X^{k+1}]=0$ since $k+1$ is odd.
\\In a) we showed that if $p\geq 2$, we have $$\mathbb{E}[X^p]=(p-1)\mathbb{E}[X^{p-2}]$$
By this recursive structure, if we keep doing it we will end up where $\mathbb{E}[X^{p-n}]=\mathbb{E}[X^1]=0$ so therefore it holds for $p=k+1$ odd. 
\\\textbf{Case 2: $k$ is odd}
If $k$ is odd we have $k+1$ is even. In a we showed that:
$$
\frac{1}{\sqrt{2 \pi}} \int_{\mathbb{R}} x^p \exp \left(-\frac{1}{2} x^2\right) \lambda(\mathrm{d} x)=\frac{p-1}{\sqrt{2 \pi}} \int_{\mathbb{R}} x^{p-2} \exp \left(-\frac{1}{2} x^2\right) \lambda(\mathrm{d} x)
$$
for all $p$ in $\N$, such that $p \geq 2$, and thus:
$$\int_{\mathbb{R}} x^p \exp \left(-\frac{1}{2} x^2\right) \lambda(\mathrm{d} x)=(p-1)\int_{\mathbb{R}} x^{p-2} \exp \left(-\frac{1}{2} x^2\right) \lambda(\mathrm{d} x)$$
So if we put in $k+1$ in we get:
$$\mathbb{E}[X^{k+1}]=\frac{k}{\sqrt{2 \pi}} \int_{\mathbb{R}} x^{k-1} \exp \left(-\frac{1}{2} x^2\right) \lambda(\mathrm{d} x)$$
Which is $(k+1)-1(E[X^{k-1}])$ and so on. So by the recursive nature of the moments, we have that it holds for $p=k+1$ if $k+1$ is even.
\part Let $Y\sim N(\xi, \sigma^2)$, so the density function is given by:
$$
f_Y(y)=\frac{1}{\sqrt{2 \pi \sigma^2}} \exp \left(-\frac{1}{2 \sigma^2}(y-\xi)^2\right), \quad(y \in \mathbb{R})
$$
Now we consider random variable $Z=(\sigma^2)^{-0.5}(Y-\xi)=\frac{Y-\xi}{\sqrt{\sigma^2}}$
\\So the inverse is equal to $Y=\xi+\sqrt{\sigma^2}Z$ with $\frac{dY}{dZ}=\sqrt{\sigma^2}$
\\By 13.3.7 we have 
$$f_Z(z)=f_Y(\xi+\sqrt{\sigma^2}z)\left|\frac{d}{dz}(\xi+\sqrt{\sigma^2}z)\right|$$
$$
=\frac{1}{\sqrt{2 \pi \sigma^2}} \exp \left(-\frac{1}{2 \sigma^2}((\xi+\sqrt{\sigma^2}z)-\xi)^2\right)\sqrt{\sigma^2}
$$
$$
=\frac{1}{\sqrt{2 \pi}} \exp \left(-\frac{1}{2 \sigma^2}((\xi+\sqrt{\sigma^2}z)-\xi)^2\right)
$$
$$
=\frac{1}{\sqrt{2 \pi}}\exp \left(-\frac{1}{2}z^2\right)
$$
Which is the desired result.
\part
$$\mathbb{E}[Y]=\xi$$
$$\operatorname{Var}[Y]=\sigma^2$$
$$\mathbb{E}[Y^3]=\xi^3+3\xi\sigma^2$$
$$\mathbb{E}[Y^4]=3\sigma^2\xi^3+6\xi^2\sigma^2+\xi^4$$

\problem{13.10}
Let $X$ be a random variable defined on a probability space $(\Omega, \mathcal{F}, P)$, and assume that $X \in \mathcal{L}^2(P)$.
\begin{enumerate}
\item Show that $\operatorname{Var}[\mathrm{X}]=\mathrm{E}[\mathrm{X}(\mathrm{X}-1)]-\mathbb{E}[\mathrm{X}](\mathbb{E}[\mathrm{X}]-1)$.
\item Use (a) to calculate $\operatorname{Var}[X]$, when $X$ has a binomial distribution.
\item Use (a) to calculate $\operatorname{Var}[\mathrm{X}]$, when X has a Poisson distribution.
\end{enumerate}
\solution
\part 
Recall that $$\operatorname{Var}[X]=\E[X^2]-(\E[X])^2$$
And express $X^2=X(X-1)+X$
Then $$\E[X^2]=\E[X(X-1)]+\E[X]$$
We now input this into our equation for the variance:
$$\operatorname{Var}[X]=\E[X(X-1)]+\E[X]-(\E[X])^2$$
$$=\E[X(X-1)]+\E[X]-\E[X]\E[X]$$
$$=\E[\mathrm{X}(\mathrm{X}-1)]-\mathbb{E}[\mathrm{X}](\mathbb{E}[\mathrm{X}]-1)$$
\part 
We have $\E[X]=np$ and $\E[X(X-1)]=n(n-1)^2$
\\So by (a) we have:
$$\operatorname{Var}[X]=\E[\mathrm{X}(\mathrm{X}-1)]-\mathbb{E}[\mathrm{X}](\mathbb{E}[\mathrm{X}]-1)$$
$$=n(n-1)p^2-np(np-1)$$
$$=n(n-1)p^2-np^2+np$$
$$=np(1-p)$$
\part
We have $\E[X]=\lambda$ and $\E[X(X-1)]=\lambda^2$
\\So by (a) we have:
$$\operatorname{Var}[X]=\E[\mathrm{X}(\mathrm{X}-1)]-\mathbb{E}[\mathrm{X}](\mathbb{E}[\mathrm{X}]-1)$$
$$=\lambda^2-\lambda(\lambda-1)$$
$$=\lambda^2-(\lambda^2-\lambda)$$
$$=\lambda$$
\problem{13.12}
Let $\left(X_n\right)_{n \in \mathbb{N}}$ and $\left(Y_n\right)_{n \in \mathbb{N}}$ be two sequences of random variables defined on a probability space $(\Omega, \mathcal{F}, P)$. Assume further that
$$
P\left(\left|\mathrm{X}_n-\mathrm{Y}_n\right|>n^{-1}\right) \leq n^{-2} \quad \text { for all } n \text { in } \mathbb{N} .
$$
Show then that the following assertions are equivalent:
\begin{enumerate}
    \item There exists a random variable $X$ on $(\Omega, \mathcal{F}, P)$ such that $X_n \rightarrow X$ P-n.o.
    \item There exists a random variable $Y$ on $(\Omega, \mathcal{F}, P)$ such that $Y_n \rightarrow Y P$-n.o.
\end{enumerate}
Show also that if (a) and (b) are satisfied, then $X=Y P$-n.o.
\solution
    \part We have that it is enough to show one of the ways, since we have not assumed anything about the sequence $(X_n)_{n\in\N}$ that we have not assumed about $(Y_n)_{n\in\N}$. Therefore we only show (a)$\Rightarrow$(b). Note that:
    $$\sum_{n=1}^{\infty}\Prob(|X_n-Y_n|>n^-1)\leq \sum_{n=1}^{\infty}\frac{1}{n^2}=\frac{\pi^2}{6}<\infty$$
    So we use Borel-Cantelli I, where we set $F_n = \{|X_n-Y_1|>n^-1\}$ for all $n\in\N$, then $\Prob(\bigcup_{n\in\N} F_n)=0$
    \\So if $U=\bigcup_{n\in\N}F_n$, then $$U^C=\{\omega \in\Omega : |X_n(\omega)-Y_n(\omega)|\leq n^-1 \text{for n large enough}\}$$
    We set $A=U^C\bigcap \{X_n\rightarrow X\}$ so for $\omega\in A$ there exists $N'\in\N$ such that $$|X_n(\omega)-Y_n(\omega)|\leq \frac{1}{n} \forall n \geq N'$$
    So $(X_n(\omega))_{n\in\N}$ converges, it is a cauchy sequence. 
    \\We now let $\epsilon>0$, and choose $N''\in\N$ such that:
    $$\frac{1}{n}\leq \frac{\epsilon}{3}$$
    and 
    $$|X_n(\omega)-X_m(\omega)|\leq \frac{\epsilon}{3}\forall n,m\geq N''$$
    We now set $N=\max\{N',N''\}$. Then we have for $n,m\geq N$ it holds that:
    $$|Y_n(\omega)-Y_m(\omega)|\leq |Y_n(\omega)-X_n(\omega)|+|X_n(\omega)-X_m(\omega)|+|X_m(\omega)-Y_m(\omega)|\leq 3\frac{\epsilon}{3}=\epsilon$$
    So $(Y_n(\omega))_{n\in\N}$ is a cauchy sequence, so it converges. We set $Y=\lim_{n\rightarrow \infty}Y_n\mathbbm{1}_A$. Then we have by 4.3.7(ii) that $Y$ is a stochastic variable. Since we have:
    $$\Prob(A^C)\leq \Prob (U) +\Prob(\{X_n\rightarrow X\}^C)=0+0=0$$
    So $Y_n\rightarrow Y$ p-a.e., so (a)$\Rightarrow$(b)
    \\Now assume (a) and (b) both holds. Then we have ofr $$B=\omega \in \{X_n\rightarrow X\}\cap \{Y_n\rightarrow Y\}\cap U^C$$ it holds for all $n\in\N$ that
    $$|X(\omega)-Y(\omega)|\leq |X(\omega)-X_n(\omega)|+|X_n(\omega)-Y_n(\omega)|+|Y_n(\omega)-Y(\omega)|$$
    and since the RHS tends to 0 as $n\rightarrow \infty$ then $X(\omega)=Y(\omega)$ and since $\Prob(B^C)=0$ as per the same argument as above, then $X=Y$ p-a.e.

\problem{13.15}
Let $X$ and $Y$ be random variables defined on a probability space $(\Omega, \mathcal{F}, P)$.
\begin{enumerate}
    \item Assume that $X, Y \in \mathcal{L}^2(P)$. Show then that if $X$ and $Y$ are independent, it holds that $\operatorname{Cov}[X, Y]=0$, and that $\operatorname{Var}[X+Y]=\operatorname{Var}[X]+\operatorname{Var}[Y]$.
    \item Show that X and Y are independent, if and only if $\operatorname{Cov}[f(\mathrm{X}), g(\mathrm{Y})]=0$ for all bounded Borel functions $f, g: \mathbb{R} \rightarrow \mathbb{R}$.
\end{enumerate}
\textit{Hint for the "if-part": Let $f$ and $g$ be indicator functions.}
\solution
\part 
It follows from corollary 13.5.6(i) that $XY\in\mathcal{L}^1(P)$, since $X$ and $Y$ are independent and has the second moment, and it further holds that:
$$\operatorname{Cov}[X,Y]=\mathbb{E}[XY]-\mathbb{E}[X]\mathbb{E}[Y]=\mathbb{E}[X]\mathbb{E}[Y]-\mathbb{E}[X]\mathbb{E}[Y]=0$$
and therefore it follows that:
$$\operatorname{Var}[X+Y]=\operatorname{Var}[X]+\operatorname{Var}[Y]-2 \operatorname{Cov}[X,Y]=\operatorname{Var}[X]+\operatorname{Var}[Y]-2\cdot 0 = \operatorname{Var}[X]+\operatorname{Var}[Y]$$
\part 
First we show $\Rightarrow$. Assume $X$ and $Y$ are independent and let $f, g: \mathbb{R} \rightarrow \mathbb{R}$ be bounded borel-functions, then it follows from remark 13.5.2(ii) that $f(X)$ and $g(Y)$ are independent, and since $f, g$ are bounded, $f(X), g(Y) \in$ $\mathcal{L}^2(\mathbb{P})$,and then by (a), $\operatorname{Cov}[f(X), g(Y)]=0$. \\We now show $\Leftarrow$. Assume $\operatorname{Cov}[f(X), g(Y)]= 0$ for all Borel-functions $f, g: \mathbf{R} \rightarrow \mathbf{R}$. Let $A, B \in \mathcal{F}$, then $\mathbf{1}_A, \mathbf{1}_B$ are bounded Borel-functions, then it follows that
$$
\begin{aligned}
0 & =\operatorname{Cov}\left[\mathbf{1}_A(X), \mathbf{1}_B(Y)\right]=\mathbb{E}\left[\mathbf{1}_A(X) \mathbf{1}_B(Y)\right]-\mathbb{E}\left[\mathbf{1}_A(X)\right] \mathbb{E}\left[\mathbf{1}_B(Y)\right] \\
& =\mathbb{E}\left[\mathbf{1}_{\{X \in A, Y \in B\}}\right]-\mathbb{E}\left[\mathbf{1}_{\{X \in A]}\right] \mathbf{E}\left[\mathbf{1}_{\mid Y \in B]}\right] \\
& =\mathbb{P}(X \in A, Y \in B)-\mathbb{P}(X \in A) \mathbb{P}(Y \in B) \\
& \Longrightarrow \mathbb{P}(X \in A, Y \in B)=\mathbb{P}(X \in A) \mathbb{P}(Y \in B)
\end{aligned}
$$
so $X$ and $Y$ are independent. 
\problem{13.16}
Let $X$ be a random variable defined on a probability space et $(\Omega, \mathcal{F}, P)$, and assume that the distribution of $X$ is the uniform distribution on $[-1,1]$, i.e. the measure with density
$$
f_{\mathrm{X}}(t)=\frac{1}{2} \mathbf{1}_{[-1,1]}(t), \quad(t \in \mathbb{R})
$$
with respect to Lebesgue measure $\lambda$ on $\mathbb{R}$.
\begin{enumerate}
    \item Show that $\operatorname{Cov}\left[\mathrm{X}, \mathrm{X}^2\right]=0$.
    \item Decide whether $X$ and $X^2$ are independent.
\end{enumerate}
\solution
\part We first see that $X,X^2\in\mathcal{L}^2(P)$ since $f_X$ is bounded. Since $x\mapsto x^2, x^3$ are borel functions, 13.3.6 entails that $x\mapsto x,x^3$ are uneven functions, and it follows that:
$$\operatorname{Cov}[X,X^2]=\E[X^3]-\E[X]\E[X^2]$$
$$=\frac{1}{2}\int_{-1}^1 x^3dx-\left(\frac{1}{2}\int_{-1}^1 x dx\right)(\frac{1}{2}\int_{-1}^1x\cdot x^2 dx)$$
$$=0-0\cdot \left(\frac{1}{2}\int_{-1}^1 x^3 dx\right)=0$$
\part 
No. We see that the mean for $X\cdot X^2=X^3$ does not split in one product since:
$$\E[X^3]=\int_{-1}^1 x \cdot x^3 \mathrm{~d} x=\int_{-1}^1 x^4 \mathrm{~d} x=\frac{2}{5}$$
but $\E[X^2]=0$ 

\problem{13.17}
Let X be a random variable defined on the probability space $(\Omega, \mathcal{F}, P)$, and assume that $\mathbb{E}\left[X^2\right]<\infty$.
\begin{enumerate}
\item Prove Chebychev's inequality:
$$
P(|\mathrm{X}-\mathbb{E}[\mathrm{X}]| \geq \epsilon) \leq \frac{1}{\epsilon^2} \operatorname{Var}[\mathrm{X}]
$$
for any positive number $\epsilon$.
\item Assume that $X_1, X_2, X_3, \ldots$ is a sequence of i.i.d. random variables with the same distribution as $X$. Prove then for any positive number $\epsilon$ that
$$
\left.\left.P\left(\left|\frac{1}{n} \sum_{k=1}^n X_k-\mathbf{E}\right| X\right] \right\rvert\,>\epsilon\right) \underset{n \rightarrow \infty}{\longrightarrow} 0
$$
This result is a version of the so-called Weak Law of Large Numbers; compare with Proposition 13.6.1.
\end{enumerate}
\solution
\part
We first notice that $\Delta=\left\{(y, y) \in \mathbb{R}^2: y \in R\right\}$ is a $\lambda_2$-null set as per remark 13.3.5 and since $X$ and $Y$ are independent, $(X, Y)$ are absolutely continuous by example 13.5.7(A). So it follows from proposition 5.3.6(ii) that
$$
\begin{aligned}
\mathbf{P}(X=Y) & =\mathbb{P}((X, Y) \in \Delta)=\int_{\mathbb{R}^2} \mathbb{1}_{\Delta} \mathrm{d} \mathbb{P}_{(X, Y)} \\
& =\int_{\mathbb{R}^2} f_{(X, Y)}(x, y) \mathbf{1}_{\Delta}(x, y) \lambda_2(\mathrm{~d} x, \mathrm{~d} y)=0
\end{aligned}
$$
\part 
Since $\mathbf{P}(X=Y)=0$, it follows that $\mathbf{P}(X<Y)+\mathbb{P}(X>Y)=1$, and since $X$ and $Y$ are identically distributed, $\mathbb{P}_X=\mathbf{P}_Y$, since $X$ and $Y$ are independent, and $\mathbf{P}_{(X, Y)}=\mathbf{P}_X \otimes \mathbf{P}_Y$, then it holds that,

$$
\begin{aligned}
\mathbf{P}(X<Y) & =\mathbb{P}_{(X, Y)}\left(\left\{(x, y) \in \mathbb{R}^2: x<y\right\}\right) \\
& =\left(\mathbb{P}_X \otimes \mathbb{P}_Y\right)\left(\left\{(x, y) \in \mathbb{R}^2: x<y\right\}\right) \\
& =\left(\mathbb{P}_Y \otimes \mathbb{P}_X\right)\left(\left\{(x, y) \in \mathbb{R}^2: x<y\right\}\right) \\
& =\mathbb{P}_{(Y, X)}\left(\left\{(x, y) \in \mathbb{R}^2: x<y\right\}\right) \\
& =\mathbb{P}(X>Y)
\end{aligned}
$$

and therefore we have:

$$
2 \mathbb{P}(X<Y)=\mathbb{P}(X<Y)+\mathbb{P}(X>Y)=1 \Longrightarrow \mathbb{P}(X<Y)=\mathbb{P}(X>Y)=\frac{1}{2}
$$

\end{document}
