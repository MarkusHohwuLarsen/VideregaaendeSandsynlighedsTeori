\documentclass{Class}
\usepackage[utf8]{inputenc}
\usepackage{graphicx}
\usepackage{listings}
\usepackage{color}
\usepackage{float}
\usepackage{amsmath} 
\usepackage{hyperref}
% \usepackage{calrsfs}
\usepackage{mathrsfs}
\usepackage{aligned-overset}
\usepackage{amssymb}
\usepackage{mathtools}
\usepackage[mathscr]{euscript}
\usepackage{tikz}
\usepackage{bbm}
\usepackage[most]{tcolorbox}
\usepackage{booktabs}
\usepackage{amsthm}
\newcommand{\N}{\mathbb{N}}
\newcommand{\Q}{\mathbb{Q}}
\newcommand{\R}{\mathbb{R}}
\newcommand{\C}{\mathbb{C}}
\newcommand{\F}{\mathbb{F}}
\newcommand{\E}{\mathbb{E}}
\newcommand{\1}{\mathbbm{1}}
\newcommand{\X}{\mathsf{X}}
\newcommand{\Y}{\mathsf{Y}}
\newcommand{\B}{\mathcal{B}}
\newcommand{\lclass}{\mathcal{L}}
\newcommand{\Prob}{\mathbb{P}}
\newcommand{\deriv}{\operatorname{d}}
\newcommand{\icomp}{\operatorname{i}}
\newcommand{\varx}{\varphi_\X}
\newcommand{\pfield}{(\Omega, \mathcal{F}, P)}
\newcommand \independent{\protect\mathpalette{\protect\independenT}{\perp}}
\def\independenT#1#2{\mathrel{\rlap{$#1#2$}\mkern2mu{#1#2}}}
\newtheorem{theorem}{Theorem}[subsection]
\newtheorem{definition}[theorem]{Definition} 
\newtheorem{lemma}[theorem]{Lemma} 
\newtheorem{corollary}[theorem]{Korollar} 
\newtheorem{remark}[theorem]{Bemærkning} 
\newtheorem{proposition}[theorem]{Sætning} 
\newtheorem{example}[theorem]{Eksempel} 
\usepackage{geometry}
    \geometry{
        a4paper,
        left=3.5cm,
        right=3.5cm   ,
    }
\definecolor{dkgreen}{rgb}{0,0.6,0}
\definecolor{gray}{rgb}{0.5,0.5,0.5}
\definecolor{mauve}{rgb}{0.58,0,0.82}
\newtcolorbox{theorem-box}{
    colback=gray!10, % Light grey background
    colframe=black,  % Black frame
    sharp corners,   % Square corners
    boxrule=0.8pt,   % Border thickness
    before skip=10pt, % Space before the box
    after skip=10pt,  % Space after the box
}

\newtheoremstyle{boxed}  % Define a new theorem style
  {10pt}   % Space above
  {10pt}   % Space below
  {}       % Body font
  {}       % Indent amount
  {\bfseries} % Theorem head font (bold)
  {.}      % Punctuation after theorem head
  { }      % Space after theorem head
  {\thmname{#1}~\thmnumber{#2}\thmnote{ (#3)}}  % Theorem head spec

\theoremstyle{boxed}
\lstset{frame=tb,
  language=Python,
  aboveskip=3mm,
  belowskip=3mm,
  showstringspaces=false,
  columns=flexible,
  basicstyle={\small\ttfamily},
  numbers=none,
  numberstyle=\tiny\color{gray},
  keywordstyle=\color{blue},
  commentstyle=\color{dkgreen},
  stringstyle=\color{mauve},
  breaklines=true,
  breakatwhitespace=true,
  tabsize=3
}
\renewcommand{\thesubsection}{\thesection.\arabic{subsection}}

\title{Assignment 3}
\author{Markus Hohwü Larsen - 202205800}
\begin{document}
\problem{1}
Consider the measure space $(X,\mathcal{E}, \mu)$, and let $f,g,f_1,f_2,f_3,\ldots$ be functions from $\mathcal{M}(\mathcal{E})$. Further let
$$\stackrel{(1)}{\longrightarrow}\quad \text{ and }\quad \stackrel{(2)}{\longrightarrow}$$ 
denote two (possibly different) of the three fundamental types of convergence defined in Definition 2.1.1 in the notes.
\begin{enumerate}
    \item Generalize Proposition 2.1.3 by proving, that for any choice of $\stackrel{(1)}{\longrightarrow},\stackrel{(2)}{\longrightarrow}$, the following implication holds:
    \begin{equation}f_n\stackrel{(1)}{\longrightarrow}f\; \text{ and }\; f_n\stackrel{(2)}{\longrightarrow}g \Longleftrightarrow f=g \; \mu\text{-a.e.}\end{equation} 
    \item Can we in (a) generally conclude, that $f(x) = g(x)$ for all $x \in X$?
\end{enumerate}
\solution
\begin{enumerate}
    \item We have the following 4 cases:
    \begin{enumerate}
        \item If $\stackrel{(1)}{\longrightarrow} \text{ and } \stackrel{(2)}{\longrightarrow}$ denote the same type of convergence, then by 2.1.3 we have that (1) is satisfied.
        \item If $\stackrel{(1)}{\longrightarrow} \text{ and } \stackrel{(2)}{\longrightarrow}$ are convergence in $\mu$-p-mean and convergence in $\mu$-measure such that 
        $$f_n \longrightarrow f \text{ in }\mu\text{-p-mean, and } f_n\longrightarrow g \text{ in }\mu\text{-measure}$$
        then by 2.1.4(i), 
        $f_n \longrightarrow f \text{ in }\mu\text{-p-measure}$, and again by 2.1.3, (1) is satisfied.
        \item If $\stackrel{(1)}{\longrightarrow} \text{ and } \stackrel{(2)}{\longrightarrow}$ are convergence $\mu$-a.e. and convergence in $\mu$-p-mean such that
         $$f_n \longrightarrow f \text{ in }\mu\text{-a.e., and } f_n\longrightarrow g \text{ in }\mu\text{-p-mean}$$
         then, by 7.4.10 [M\&I] there exists an increasing sequence of natural numbers $(n_k)_{k\in\N}$ such that $f_{n_k}\longrightarrow g \;\mu\text{-a.e. for }k\rightarrow \infty$.
         \\We also have that, since $f_n\longrightarrow f\; \mu\text{-a.e.}$, this also holds for any subsequence. Specifically, we have that $f_{n_k}\longrightarrow f \;\mu\text{-a.e. for }k\rightarrow \infty$ for the specified sequence $(n_k)$. Again by 2.1.3, (1) is satisfied.
        \item If $\stackrel{(1)}{\longrightarrow} \text{ and } \stackrel{(2)}{\longrightarrow}$ are convergence $\mu$-a.e. and convergence in $\mu$-measure such that
         $$f_n \longrightarrow f \text{ in }\mu\text{-a.e., and } f_n\longrightarrow g \text{ in }\mu\text{-measure}$$
         then by 2.1.4(iii) there exists an increasing sequence of natural numbers $(n_k)_{k\in\N}$ such that $f_{n_k}\longrightarrow g \;\mu\text{-a.e. for }k\rightarrow \infty$. Again, by the same argument as in (iii) we have $f_{n_k}\longrightarrow f \; \mu\text{-a.e.}$, and again by 2.1.3, (1) is satisfied. 
    \end{enumerate}
    Therefore we have shown the desired property.
    \item If we consider the functions $f_n=\1_{\{0\}}, f=\1_{\{0\}}$ and $g=0$ from $(\R, \mathcal{B}(\R),\lambda)$, then we have that $$f_n \rightarrow f$$ for all three main types of convergence, since $f_n=f \;\forall n \in \N$. \\In the following, we show that $f_n \rightarrow g$ for all three main types of convergence:
\\For $f_n \rightarrow g$ convergence $\lambda$-a.e. and convergence in the $\lambda$-p-mean we have that $\1_{\{0\}}$ and 0 are both equal to $0 \;\lambda$-a.e. and $$\int_{\mathbb{R}}\left|\1_{\{0\}}-0\right|^p d \lambda=\int_{\mathbb{R}} \1_{\{0\}} d \lambda=\lambda(\{0\})=0 \forall n \in \mathbb{N}$$.

We now argue that $f_n \rightarrow g$ in $\lambda$-measure:\\Let $\epsilon>0$. Then $\forall n \in \mathbb{N}$ it holds that:

$$
\lambda\left(\left\{x \in X:\left|f_n(x)-g(x)\right|>\epsilon\right\}\right)=\lambda\left(\left\{x \in X: \1_{\{0\}}(x)>\epsilon\right\}\right)$$$$=\begin{cases}
    \lambda(\{0\})=0, \quad& \text{ for }\epsilon<1 \\
\lambda(\emptyset)=0, \quad& \text {    otherwise }
\end{cases}
$$
So to use the notation from (a), we have that $$f_n \xrightarrow{(1)} \1_{\{0\}} \quad \text{ and } \quad f_n \xrightarrow{(2)} 0$$ for any two of the three main types of convergence. We also have that $\1_{\{0\}}$ and 0 are not equal on the $\lambda$-null set $\{0\}$. Therefore, we cannot generally conclude that $f(x)=g(x)$ for all $ x \in X$ in (a).
\end{enumerate}
\end{document}