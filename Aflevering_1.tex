\documentclass{Class}
\usepackage[utf8]{inputenc}
\usepackage{graphicx}
\usepackage{listings}
\usepackage{color}
\usepackage{float}
\usepackage{amsmath} 
\usepackage{hyperref}
% \usepackage{calrsfs}
\usepackage{mathrsfs}
\usepackage{aligned-overset}
\usepackage{amssymb}
\usepackage{mathtools}
\usepackage[mathscr]{euscript}
\usepackage{tikz}
\usepackage{bbm}
\usepackage[most]{tcolorbox}
\usepackage{booktabs}
\usepackage{amsthm}
% \usepackage[scr=boondox,  % heavily sloped
%             cal=esstix]   % slightly sloped
%            {mathalpha}
\newcommand{\N}{\mathbb{N}}
\newcommand{\Q}{\mathbb{Q}}
\newcommand{\R}{\mathbb{R}}
\newcommand{\C}{\mathbb{C}}
\newcommand{\F}{\mathbb{F}}
\newcommand{\E}{\mathbb{E}}
\newcommand{\1}{\mathbbm{1}}
\newcommand{\X}{\mathsf{X}}
\newcommand{\Y}{\mathsf{Y}}
\newcommand{\B}{\mathcal{B}}
\newcommand{\lclass}{\mathcal{L}}
\newcommand{\Prob}{\mathbb{P}}
\newcommand{\deriv}{\operatorname{d}}
\newcommand{\icomp}{\operatorname{i}}
\newcommand{\varx}{\varphi_\X}
\newcommand{\pfield}{(\Omega, \mathcal{F}, P)}
\newtheorem{theorem}{Theorem}[subsection]
\newtheorem{definition}[theorem]{Definition} 
\newtheorem{lemma}[theorem]{Lemma} 
\newtheorem{corollary}[theorem]{Korollar} 
\newtheorem{remark}[theorem]{Bemærkning} 
\newtheorem{proposition}[theorem]{Sætning} 
\newtheorem{example}[theorem]{Eksempel} 
\usepackage{geometry}
    \geometry{
        a4paper,
        left=3.5cm,
        right=3.5cm   ,
    }
\definecolor{dkgreen}{rgb}{0,0.6,0}
\definecolor{gray}{rgb}{0.5,0.5,0.5}
\definecolor{mauve}{rgb}{0.58,0,0.82}
\newtcolorbox{theorem-box}{
    colback=gray!10, % Light grey background
    colframe=black,  % Black frame
    sharp corners,   % Square corners
    boxrule=0.8pt,   % Border thickness
    before skip=10pt, % Space before the box
    after skip=10pt,  % Space after the box
}

\newtheoremstyle{boxed}  % Define a new theorem style
  {10pt}   % Space above
  {10pt}   % Space below
  {}       % Body font
  {}       % Indent amount
  {\bfseries} % Theorem head font (bold)
  {.}      % Punctuation after theorem head
  { }      % Space after theorem head
  {\thmname{#1}~\thmnumber{#2}\thmnote{ (#3)}}  % Theorem head spec

\theoremstyle{boxed}
\lstset{frame=tb,
  language=Python,
  aboveskip=3mm,
  belowskip=3mm,
  showstringspaces=false,
  columns=flexible,
  basicstyle={\small\ttfamily},
  numbers=none,
  numberstyle=\tiny\color{gray},
  keywordstyle=\color{blue},
  commentstyle=\color{dkgreen},
  stringstyle=\color{mauve},
  breaklines=true,
  breakatwhitespace=true,
  tabsize=3
}
\renewcommand{\thesubsection}{\thesection.\arabic{subsection}}

\title{Assigment 1}
\author{Markus Hohwü Larsen - 202205800}
\begin{document}
\problem{Exercise 1.4}
Let $\X$ be a stochastic variable on the probability field $\pfield,$ and consider the characteristic function $\varx$. 
\begin{enumerate}
  \item Assume, that $\X$ is binomially distributed with parameters $n\in\N$ and probability parameter $p\in[0,1].$ Show, that $$\varx(t)=(1-p+pe^{\icomp t})^n\quad (t\in\R)$$
  \item Assume, that $\X$ is poisson distributed with parameters $\ell\in (0,\infty)$. Show, that $$\varx(t)=\exp(\ell(e^{\icomp t}-1)) \quad (t\in\R)$$
\end{enumerate}
\solution
In general we use example 13.2.7 and proposition 13.2.9 from [M\&I] to define:
  $$\psi_{\X}(t)=e^{\icomp t\X}$$
  and recall that the characteristic function for a one-dimensional stochastic vector is defined as $$\varx = \E\left[e^{\icomp t\X}\right]$$
  and then use 13.2.9:
  $$\mathbb{E}[\psi(\mathrm{X})]=\sum_{i \in I} \psi\left(x_i\right) p_{\mathrm{X}}\left(x_i\right).$$
\begin{enumerate}
  \item $$\varx=\E[e^{\icomp t\X}]=\E[\psi_\X]\stackrel{13.2.9+13.2.7(a)}{=}\sum_{k=1}^n \begin{pmatrix}
    n\\k
  \end{pmatrix}p^k(1-p)^{n-k}e^{\icomp tk} $$
  $$=\sum_{k=1}^n \begin{pmatrix}
    n\\k
  \end{pmatrix}(pe^{\icomp t})^k(1-p)^{n-k}$$
  $$\stackrel{13.2.7(A)}{=}(pe^{\icomp t}+(1-p))^n$$
  \item $$\varx=\E[e^{\icomp t\X}]=\E[\psi_\X]\stackrel{13.2.9+13.2.7(b)}{=}\sum_{k=0}^{\infty} \frac{e^{i \cdot t \cdot k} \ell^k e^{-\ell}}{k!}$$$$=e^{-\ell} \sum_{k=0}^{\infty} \frac{\left(e^{i t} \ell\right)^k}{k!}$$$$
  =e^{-\ell} \exp \left(\ell e^{\mathrm{it}}\right)$$$$=\exp \left(\ell\left(e^{\mathrm{i} t}-1\right)\right)
  $$
  
\end{enumerate}
Which are the desired results.


\end{document}