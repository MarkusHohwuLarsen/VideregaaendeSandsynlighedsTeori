\documentclass{Class}
\usepackage[utf8]{inputenc}
\usepackage{graphicx}
\usepackage{listings}
\usepackage{color}
\usepackage{float}
\usepackage{amsmath} 
\usepackage{hyperref}
% \usepackage{calrsfs}
\usepackage{mathrsfs}
\usepackage{aligned-overset}
\usepackage{amssymb}
\usepackage{mathtools}
\usepackage[mathscr]{euscript}
\usepackage{tikz}
\usepackage{bbm}
\usepackage[most]{tcolorbox}
\usepackage{booktabs}
\usepackage{amsthm}
\newcommand{\N}{\mathbb{N}}
\newcommand{\Q}{\mathbb{Q}}
\newcommand{\R}{\mathbb{R}}
\newcommand{\C}{\mathbb{C}}
\newcommand{\F}{\mathbb{F}}
\newcommand{\E}{\mathbb{E}}
\newcommand{\1}{\mathbbm{1}}
\newcommand{\X}{\mathsf{X}}
\newcommand{\Y}{\mathsf{Y}}
\newcommand{\B}{\mathcal{B}}
\newcommand{\lclass}{\mathcal{L}}
\newcommand{\Prob}{\mathbb{P}}
\newcommand{\deriv}{\operatorname{d}}
\newcommand{\icomp}{\operatorname{i}}
\newcommand{\varx}{\varphi_\X}
\newcommand{\pfield}{(\Omega, \mathcal{F}, P)}
\newcommand \independent{\protect\mathpalette{\protect\independenT}{\perp}}
\def\independenT#1#2{\mathrel{\rlap{$#1#2$}\mkern2mu{#1#2}}}
\newtheorem{theorem}{Theorem}[subsection]
\newtheorem{definition}[theorem]{Definition} 
\newtheorem{lemma}[theorem]{Lemma} 
\newtheorem{corollary}[theorem]{Korollar} 
\newtheorem{remark}[theorem]{Bemærkning} 
\newtheorem{proposition}[theorem]{Sætning} 
\newtheorem{example}[theorem]{Eksempel} 
\usepackage{geometry}
    \geometry{
        a4paper,
        left=3.5cm,
        right=3.5cm   ,
    }
\definecolor{dkgreen}{rgb}{0,0.6,0}
\definecolor{gray}{rgb}{0.5,0.5,0.5}
\definecolor{mauve}{rgb}{0.58,0,0.82}
\newtcolorbox{theorem-box}{
    colback=gray!10, % Light grey background
    colframe=black,  % Black frame
    sharp corners,   % Square corners
    boxrule=0.8pt,   % Border thickness
    before skip=10pt, % Space before the box
    after skip=10pt,  % Space after the box
}

\newtheoremstyle{boxed}  % Define a new theorem style
  {10pt}   % Space above
  {10pt}   % Space below
  {}       % Body font
  {}       % Indent amount
  {\bfseries} % Theorem head font (bold)
  {.}      % Punctuation after theorem head
  { }      % Space after theorem head
  {\thmname{#1}~\thmnumber{#2}\thmnote{ (#3)}}  % Theorem head spec

\theoremstyle{boxed}
\lstset{frame=tb,
  language=Python,
  aboveskip=3mm,
  belowskip=3mm,
  showstringspaces=false,
  columns=flexible,
  basicstyle={\small\ttfamily},
  numbers=none,
  numberstyle=\tiny\color{gray},
  keywordstyle=\color{blue},
  commentstyle=\color{dkgreen},
  stringstyle=\color{mauve},
  breaklines=true,
  breakatwhitespace=true,
  tabsize=3
}
\renewcommand{\thesubsection}{\thesection.\arabic{subsection}}

\title{Uge 5}
\author{Markus Hohwü Larsen}
\begin{document}
\problem{2.14}
Lad $(X, \mathcal{E}, \mu)$ være et målrum. Vis da, at konvergens $\mu$-n.o. er et fuldstændigt konvergensbegreb. Mere præcist: Vis, at hvis $\left(f_n\right)$ er en følge fra $\mathcal{M}(\mathcal{E})$, som er en Cauchy-følge $\mu$-n.o., da findes en funktion $f$ fra $\mathcal{M}(\mathcal{E})$, således at $f_n \rightarrow f \mu$-n.o.
\solution 
Antag, at $\left(f_n\right)$ er en følge fra $\mathcal{M}(\mathcal{E})$, som er en Cauchy-følge $\mu$-n.o. Så findes der en $\mu$-nulmængde $N \in \mathcal{E}$, således at $\left(f_n(x)\right)$ er en Cauchy-følge for alle $x \in N^{\mathrm{c}}$. For hvert $x \in N^c$ må der derfor gælde, at $\lim _{n \rightarrow \infty} f_n(x)$ eksisterer i $\mathbb{R}$. Lad nu $C=\left\{\lim _{n \rightarrow \infty} f_n\right.$ eksisterer i $\left.\mathbb{R}\right\}$, og betragt funktionen $f: X \rightarrow \mathbb{R}$ givet ved
$$
f(x)= \begin{cases}\lim _{n \rightarrow \infty} f_n(x) & \text { hvis } x \in C \\ 0 & \text { hvis } x \in C^c\end{cases}
$$
Vi bemærker, at $f_n \rightarrow f \mu$-n.o. Desuden giver Korollar 4.3.11 i [M\&I], at $f \in \mathcal{M}(\mathcal{E})$.
\problem{2.15(c)}
Lad $(X, \mathcal{E}, \mu)$ være et målrum, og lad $f, f_1, f_2, \ldots$ være funktioner fra $\mathcal{M}(\mathcal{E})$.
\begin{enumerate}
    \item[(c)]Vis at der gælder folgende implikation: $f_n \rightarrow f$ i $\mu$-mål $\Longrightarrow\left(f_n\right)$ er en Cauchy-følge i $\mu$-mål.
\end{enumerate}
\solution
Antag, at $f_n \rightarrow f$ i $\mu$-mål. Vi skal nu vise, at

$$
\forall \epsilon>0: \lim _{n, m \rightarrow \infty} \mu\left(\left\{\left|f_n-f_m\right|>\epsilon\right\}\right)=0
$$


Lad derfor $\epsilon>0$ være givet. For $m, n \in \mathbb{N}$ gælder der, at

$$
\left|f_n-f_m\right|=\left|f_n-f+f-f_m\right| \leq\left|f_n-f\right|+\left|f-f_m\right|
$$


Derfor gælder der, at

$$
\begin{aligned}
\mu\left(\left\{\left|f_n-f_m\right|>\epsilon\right\}\right) & \leq \mu\left(\left\{\left|f_n-f\right|+\left|f-f_m\right|>\epsilon\right\}\right) \\
& \leq \mu\left(\left\{\left|f_n-f\right|>\epsilon / 2\right\} \cup\left\{\left|f-f_m\right|>\epsilon / 2\right\}\right) \\
& \leq \mu\left(\left\{\left|f_n-f\right|>\epsilon / 2\right\}\right)+\left\{\left|f-f_m\right|>\epsilon / 2\right\}
\end{aligned}
$$

for alle $m, n \in \mathbb{N}$. Idet højresiden gå $\bmod 0$ for $n, m \rightarrow \infty$, har vi det ønskede.
% \problem{2.16}

\problem{3.1}
Lad $(X, \mathcal{E}, \mu)$ være et endeligt målrum, og lad $f$ være en funktion fra $\mathcal{M}(\mathcal{E})$. Vis da bi-implikationen:

$$
f \in \mathcal{L}^1(\mu) \Longleftrightarrow \forall \epsilon>0 \exists K>0: \int_{\{|f|>K\}}|f| \mathrm{d} \mu \leq \epsilon
$$
Gælder nogen af implikationerne, hvis $\mu$ ikke er et endeligt mål?
\solution
Antag først, at $f \in \mathcal{L}^1(\mu)$. Vi bemærker, at $1_{\{|f|>K\}}|f| \rightarrow 0$ for $K \rightarrow \infty$. Med $|f|$ som majorant, fär vi vha. domineret konvergens, at

$$
\int_{\{|f|>K\}}|f| \mathrm{d} \mu \rightarrow 0
$$

for $K \rightarrow \infty$. For ethvert $\epsilon>0$ kan vi derfor vælge $K>0$ med

$$
\int_{\{|f|>K\}}|f| \mathrm{d} \mu \leq \epsilon .
$$


Antag omvendt, at

$$
\forall \epsilon>0 \exists K>0: \int_{\{|f|>K\}}|f| \mathrm{d} \mu \leq \epsilon .
$$


Vi kan specielt valge $K>0$ saledes, at

$$
\int_{\{|f|>K\}}|f| \mathrm{d} \mu \leq 1
$$


Idet $\mu(X)<\infty$ gælder der, at

$$
\int_X|f| \mathrm{d} \mu=\int_{\{|f| \leq K\}}|f| \mathrm{d} \mu+\int_{\{|f|>K\}}|f| \mathrm{d} \mu \leq K \mu(X)+1<\infty,
$$

hvilket netop viser, at $f \in \mathcal{L}^1(\mathbb{P})$.
Vi ser, at implikationen fra venstre mod højre ogsa holder, hvis $\mu$ ikke er et endeligt mål. Argumentet for den modsatte implikation benyttede derimod, at $\mu(X)<\infty$. Vi kan let give et modeksempel, som viser, at implikationen ikke gælder generelt, hvis $\mu(X)=\infty$. Betragt $(X, \mathcal{E}, \mu)=(\mathbb{R}, \mathcal{B}(\mathbb{R}), \lambda)$ og $f \equiv 1$. Da er $f$ ikke et element i $\mathcal{L}^1(\lambda)$. For et givet $\epsilon>0$, vælger vi $K=1$. Da gælder der, at
$$
\int_{\{|f|>1\}}|f| \mathrm{d} \lambda=\int_{\emptyset} 1 \mathrm{~d} \lambda=0 \leq \epsilon
$$
Altså har vi her et modeksempel, der viser, at implikationen fra højre mod venstre ikke gælder generelt, hvis $\mu(X)=\infty$.
% \problem{3.2}
\problem{3.4}
Lad $(X, \mathcal{E}, \mu)$ være et endeligt målrum, og lad $\left(f_n\right)$ være en følge af funktioner fra $\mathcal{M}(\mathcal{E})$. Antag endvidere, at der findes funktioner $f \in \mathcal{M}(\mathcal{E})$ og $g \in \mathcal{M}(\mathcal{E})^{+}$, således at

$$
f_n \rightarrow f \text { i } \mu \text {-màl for } n \rightarrow \infty
$$

og

$$
\left|f_n\right| \leq g \mu \text {-n.o. for alle } n, \quad \text { og } \quad \int_X g \mathrm{~d} \mu<\infty
$$
\begin{enumerate}
    \item Vis, at $\int_X f_n \mathrm{~d} \mu \rightarrow \int_X f \mathrm{~d} \mu$.
    \item Gør rede for, at resultatet i (a) er en generalisering af Lebesgues sætning om domineret konvergens for endelige målrum.
    \item Giv et eksempel på et endeligt målrum $(X, \mathcal{E}, \mu)$ og en følge af funktioner $\left(f_n\right)$ fra $\mathcal{M}(\mathcal{E})$, således at $f_n \rightarrow 0 \mu$-n.o., og $\left(f_n\right)$ er uniformt integrabel, men også sådan at der ikke findes en majorant $g$ fra $\mathcal{M}(\mathcal{E})^{+}$, således at

    $$
    \left|f_n\right| \leq g \mu \text {-n.o. for alle } n, \quad \text { og } \quad \int_X g \mathrm{~d} \mu<\infty
    $$
    
    
    Vis, at der i denne situation alligevel gælder, at $\int_X f_n \mathrm{~d} \mu \rightarrow 0$ for $n \rightarrow \infty$.
\end{enumerate}
\solution
\begin{enumerate}
    \item Idet $g \in \mathcal{L}^1(\mu)^{+}$, følger det fra Lemma 3.1.3 (ii), at $\left\{f_n \mid n \in \mathbb{N}\right\}$ er uniformt integrabel. Da vi samtidig har, at $f_n \rightarrow f$ i $\mu$-mål, giver Sætning 3.2.1, at $f \in \mathcal{L}^1(\mu)$, $f_n \in \mathcal{L}^1(\mu)$ for alle $n, \operatorname{og} f_n \rightarrow f$ i $\mu$-middel. Vi ser nu, at

    $$
    \left|\int_X f_n \mathrm{~d} \mu-\int_X f \mathrm{~d} \mu\right|=\left|\int_X f_n-f \mathrm{~d} \mu\right| \leq \int_X\left|f_n-f\right| \mathrm{d} \mu \rightarrow 0
    $$
    
    
    Dette viser, at $\int_X f_n \mathrm{~d} \mu \rightarrow \int_X f \mathrm{~d} \mu$.
    \item Lebesgues sætning om domineret konvergens for endelige målrum kan formuleres således:

    \begin{theorem-manual}{1}
        Lad $(X, \mathcal{E}, \mu)$ vare et endeligt mälrum, og lad $f_1, f_1, f_2, \ldots$ vare funktioner fra $\mathcal{M}(\mathcal{E})$, sáledes at $f_n \rightarrow f \mu$-n.o. Antag endvidere, at der findes $g \in \mathcal{L}^1(\mu)^{+}$, salledes at $\left|f_n\right| \leq g \mu$-n.o. for alle $n$. Da galder der, at
    
    $$
    \int_X f_n \mathrm{~d} \mu \rightarrow \int_X f \mathrm{~d} \mu
    $$
    \end{theorem-manual}
    
    
    Opgaven er nu at vise denne sætning vha. delopgave (a). Idet målrummet er endeligt, og $f_n \rightarrow f \mu$-n.o., giver Sætning 2.1 .5 (ii), at $f_n \rightarrow f$ i $\mu$-mål. Vi ser nu, at antagelserne i (a) er opfyldt, så det følger, at $\int_X f_{\mathrm{n}} \mathrm{d} \mu \rightarrow \int_X f \mathrm{~d} \mu$ som ønsket.
    \item Lad $(X, \mathcal{E}, \mu)=\left([0,1], \mathcal{B}([0,1]), \lambda_{[0,1]}\right)$, og lad $f_n(x)=x^{-1} 1_{[1 /(n+1), 1 / n]}(x)$ for $x \in[0,1]$ og $n \in \mathbb{N}$. Bemærk, at $f_n \rightarrow 0 \mu$-n.o. For ethvert $n \in \mathbb{N}$ gælder der, at

    $$
    \int_X f_n^2 \mathrm{~d} \mu=\int_{1 /(n+1)}^{1 / n} x^{-2} \lambda(\mathrm{~d} x) \leq(n+1)^2(1 / n-1 /(n+1))=\frac{(n+1)^2}{n(n+1)} \leq 2
    $$
    
    
    Da har vi, at
    
    $$
    \sup _{n \in \mathbb{N}} \int_X f_n^2 \mathrm{~d} \mu \leq 2<\infty
    $$
    
    
    Eksempel 3.1.7 med $p=2$ giver så, at $\left\{f_n \mid n \in \mathbb{N}\right\}$ er uniformt integrabel. Hvis vi nu betragter en funktion $g \in \mathcal{M}(\mathcal{E})^{+}$, som opfylder, at $g \geq\left|f_n\right| \mu$-n.o. for alle $n$, så må der gælde, at $g(x) \geq x^{-1}$ for $\mu$-n.a. $x \in(0,1]$. Det følger, at $g \notin \mathcal{L}^1(\mu)$.
    Lad nu $(X, \mathcal{E}, \mu)$ være et endeligt målrum, og lad $f, f_1, f_2, \ldots$ være funktioner fra $\mathcal{M}(\mathcal{E})$. Antag, at $\left\{f_n \mid n \in \mathbb{N}\right\}$ er uniformt integrabel, og at $f_n \rightarrow 0 \mu$-n.o. Sætning 2.1.5 (ii) giver, at $f_n \rightarrow 0$ i $\mu$-mål. På samme måde som i delopgave (a) får vi, vha. Sætning 3.2.1, at $\int_X f_n \mathrm{~d} \mu \rightarrow \int_X 0 \mathrm{~d} \mu=0$.
\end{enumerate}
\end{document}