\documentclass{article}
\usepackage[utf8]{inputenc}
\usepackage{graphicx}
\usepackage{listings}
\usepackage{color}
\usepackage{float}
\usepackage{amsmath} 
\usepackage{hyperref}
% \usepackage{calrsfs}
\usepackage{mathrsfs}
\usepackage{aligned-overset}
\usepackage{amssymb}
\usepackage{mathtools}
\usepackage[mathscr]{euscript}
\usepackage{tikz}
\usepackage{bbm}
\usepackage[most]{tcolorbox}
\usepackage{booktabs}
\usepackage{amsthm}
\newcommand{\N}{\mathbb{N}}
\newcommand{\Q}{\mathbb{Q}}
\newcommand{\R}{\mathbb{R}}
\newcommand{\C}{\mathbb{C}}
\newcommand{\F}{\mathbb{F}}
\newcommand{\E}{\mathbb{E}}
\newcommand{\1}{\mathbbm{1}}
\newcommand{\X}{\mathsf{X}}
\newcommand{\Y}{\mathsf{Y}}
\newcommand{\B}{\mathcal{B}}
\newcommand{\lclass}{\mathcal{L}}
\newcommand{\Prob}{\mathbb{P}}
\newcommand{\deriv}{\operatorname{d}}
\newcommand{\icomp}{\operatorname{i}}
\newcommand{\varx}{\varphi_\X}
\newcommand{\pfield}{(\Omega, \mathcal{F}, P)}
\newcommand \independent{\protect\mathpalette{\protect\independenT}{\perp}}
\def\independenT#1#2{\mathrel{\rlap{$#1#2$}\mkern2mu{#1#2}}}
\newtheoremstyle{boxed}  % Define a new theorem style
  {10pt}   % Space above
  {10pt}   % Space below
  {}       % Body font
  {}       % Indent amount
  {\bfseries} % Theorem head font (bold)
  {.}      % Punctuation after theorem head
  { }      % Space after theorem head
  {\thmname{#1}~\thmnumber{#2}\textbf{\thmnote{ (#3)}}} % Theorem head spec

\theoremstyle{boxed}
\newtheorem{theorem}{Theorem}[subsection]
\newtheorem{definition}[theorem]{Definition} 
\newtheorem{lemma}[theorem]{Lemma} 
\newtheorem{corollary}[theorem]{Korollar} 
\newtheorem{remark}[theorem]{Bemærkning} 
\newtheorem{proposition}[theorem]{Sætning} 
\newtheorem{example}[theorem]{Eksempel} 
\newtheorem{terminology}[theorem]{Terminologi}
\newtheorem{problemstilling}[theorem]{Problemstilling} 

\usepackage{geometry}
    \geometry{
        a4paper,
        left=3.5cm,
        right=3.5cm   ,
    }
\definecolor{dkgreen}{rgb}{0,0.6,0}
\definecolor{gray}{rgb}{0.5,0.5,0.5}
\definecolor{mauve}{rgb}{0.58,0,0.82}
\newtcolorbox{theorem-box}{
    colback=gray!10, % Light grey background
    colframe=black,  % Black frame
    sharp corners,   % Square corners
    boxrule=0.8pt,   % Border thickness
    before skip=10pt, % Space before the box
    after skip=10pt,  % Space after the box
}
\lstset{frame=tb,
  language=Python,
  aboveskip=3mm,
  belowskip=3mm,
  showstringspaces=false,
  columns=flexible,
  basicstyle={\small\ttfamily},
  numbers=none,
  numberstyle=\tiny\color{gray},
  keywordstyle=\color{blue},
  commentstyle=\color{dkgreen},
  stringstyle=\color{mauve},
  breaklines=true,
  breakatwhitespace=true,
  tabsize=3
}
\usepackage{graphicx} % Required for inserting images
\renewcommand{\thesubsection}{\thesection.\arabic{subsection}}

\title{Noter - Videregående sandsynlighedsteori}
\begin{document}

\section{Fourier-transformation og karakteristiske funktioner}
\subsection{Definition og indledende bemærkninger}
\begin{theorem-box}
\begin{definition}
    Lad $\mu$ være et sandsynlighedsmål på $(\R^d, \mathcal{B}(\R^d))$. Den Fourier-transofrmerede af $\mu$ er funktionen $\hat{\mu}:\R^d\rightarrow \mathbb{C}$ givet ved 
    $$\hat{\mu}(t)=\int_{\R^d}e^{\operatorname{i}\langle t,x\rangle}\mu(\operatorname{d}x)=\int_{\R^d}\cos(\langle t,x\rangle)\mu \operatorname{d}x+\operatorname{i}\int_{\R^d}\sin(\langle t,x\rangle)\mu \operatorname{d}x$$
    for ethvert $t$ i $\R^d$. I tilfældet $d=1$ ser vi specielt, at 
    $$\hat{\mu}(t)=\int_{\R}e^{\operatorname{i}tx}\mu(\operatorname{d}x)=\int_{\R}\cos(tx)\mu \operatorname{d}x+\operatorname{i}\int_{\R}\sin(tx)\mu \operatorname{d}x$$
    for ethvert $t$ i $\R$
\end{definition}
\end{theorem-box}

    \begin{remark}
        Antag, at $\mu$ er et sandsynlighedsmål på $(\R, \B(\R))$ med tæthed $f$ fra $\lclass^1(\lambda)^+$ med hensyn til $\lambda$. Det følger da for ethvert $t$ i $\R$, at 
        $$\hat{\mu}(t)=\int_{\R}e^{\operatorname{i}tx}\mu(\operatorname{d}x)=\int_{\R}e^{\operatorname{i}tx}f(x)\lambda(\operatorname{d}x)=\sqrt{2\pi}\hat{f}(-t),$$
        hvor $\hat{f}$ betegner den Fourier-transformerede af $f$ (jvf. Definition 12.1.1 i [M\&I])
    \end{remark}
\begin{example}[Den Fourier-transformerede af normalfordelingen]
    Vi har
    $$\widehat{N(\xi, \sigma^2)}(t)=e^{\operatorname{i}t\xi}e^{-\sigma^2t^2/2}$$
    for ethvert $t$ i $\R$
\end{example}
\begin{theorem-box}
    \begin{proposition}
        Lad $\mu$ og $\nu$ være ssh.-mål på $(\R^d, \B(\R^d))$ hhv. $(\R^m, \B(\R^m))$. Da gælder følgende udsagn:
        \begin{enumerate}
            \item[\textnormal{(i)}] $|\hat{\mu}(t)|\leq \hat{\mu}(0) = 1 $ for alle $t$ i $\R^d$
            \item[\textnormal{(ii)}] $\hat{\mu}:\R^d \rightarrow \C$ er en kontinuert funktion.
            \item[\textnormal{(iii)}] $\hat{\mu}(-t)=\overline{\hat{\mu}(t)}$ for alle $t$ i $\R^d$
            \item[\textnormal{(iv)}] Hvis $d=m$ gælder der, at 
            $$\int_{\R^d}\hat{\mu}(t)\nu(\deriv t)=\int_{\R^d}\hat{\nu}(t)\mu(\deriv t)$$ 
            \item[\textnormal{(v)}]$\widehat{\mu\otimes\nu}(t,s)=\hat{\mu}(t)\hat{\nu}(s)$ for alle $(t,s)$ i $\R^d\times\R^m=\R^{d+m}$
            \item[\textnormal{(vi)}] $\widehat{\mu * \nu}(t,s)=\hat{\mu}(t)\cdot\hat{\nu}(t)$ for alle $t$ i $\R^d$
        \end{enumerate}
    \end{proposition}
\end{theorem-box}

\begin{theorem-box}
    \begin{definition} 
        Lad $\mathsf{X}$ være en d-dimensionel stokastisk vektor defineret op sandsynlighedsfeltet $\pfield$. Den karakteristiske funktion for $\mathsf{X}$ er funktionen $\varphi_{\mathsf{X}}:\R^d\rightarrow \C$ givet ved 
        $$\varphi_\mathsf{X}=\hat{P_\X}, \quad \text{hvor }P_\X=P \circ \X^{-1}$$
        For ethvert $t$ i $\R^d$ hat vi altså, at
        $$\varphi_\X(t)=\int_{\R^d}e^{\icomp \langle t,x\rangle}P_\X(\deriv x) = \int_{\Omega}e^{\icomp \langle t, \X(\omega)\rangle}P(\deriv\omega)=\E[e^{\icomp \langle t,x\rangle}]$$
    \end{definition}
\end{theorem-box}
\begin{example}
    Hvis $\X$ er normalfordelt, så har vi
    $$\varphi_\X(t)=\widehat{N(\xi, \sigma^2)}(t)=e^{\icomp t\xi}e^{-\sigma^2t^2/2}$$
    for alle t i $\R$
\end{example}
\begin{theorem-box}
    \begin{corollary}[Egenskaber ved den karakteristiske funktion]
        Lad $\X$ og $\Y$ være hhv. d- og m-dimensionale stokastiske vektorere definerede på $\pfield$. Da gælder følgende udsagn:
        \begin{enumerate}
            \item[\textnormal{(i)}] $|\varphi_\X(t)|\leq \varphi_\X(0)=1$ for alle t i $\R^d$
            \item[\textnormal{(ii)}] Funktionen $\varphi_\X:\R^d\rightarrow \C$ er kontinuert.
            \item[\textnormal{(iii)}] $\varphi_\X(-t)=\varphi_{-\X}(t)=\overline{\varx(t)}$ for alle t i $\R^d$
            \item[\textnormal{(iv)}] For enhver $m\times n$ matrix A og enhver vektor b i $\R^m$ gælder formlen:
            $$\varphi_{A\X+b}(s)=e^{\icomp\langle s,b\rangle}\varx (A^Ts), \quad (s\in\R^m)$$
            \item[\textnormal{(v)}] Hvis $d=m$, gælder formlen: $\E[\varphi_\Y(\X)]=\E[\varx(\Y)]$
            \item[\textnormal{(vi)}] Hvis $\X$ og $\Y$ er uafhængige, gælder formlen: 
            $$\varphi_{(\X,\Y)}(t,s)=\varx(t)\varphi_\Y(s), \quad (t\in\R^d,s\in\R^m)$$ 
            \item[\textnormal{(vii)}] Hvis $d=m$, og $\X$ og $\Y$ er uafhængige, gælder formlen:
            $$\varphi_{\X+\Y}(t)=\varx(t)\varphi_\Y(t)=\varphi_{(\X,\Y)}(t,t), \quad (t\in\R^d)$$
        \end{enumerate}
    \end{corollary}
\end{theorem-box}
\subsection{Entydighed og Inversionsætningen for karakteristiske funktioner}
\begin{theorem-box}
    \begin{lemma}
        Lad $\mu$ og $\nu$ være to mål på $(\R^d, \B(\R^d))$ og antag at $\mu((-n,n)^d)<\infty$ for alle n i $\N$, samt at
        $$\int_{\R^d}\psi \deriv \mu =\int_{\R^d}\psi \deriv \nu\quad \text{for alle }\psi \text{ i } C_c(\R^d,\R)^+$$
        Da gælder der, at $\mu=\nu$
    \end{lemma}
\end{theorem-box}
Altså at $\psi$ tilhører alle kontinuerte funktioner med kompakt støtte.
\begin{theorem-box}
    \begin{lemma}
        Lad $\X$ og $\Y$ være uafhængiged d-dimensionale stokastiske vektorer definerede på sandsynlighedsfeltet $\pfield$, og antag, at $\X$ er absolut kontinuert med tæthed $f_\X$ (med hensyn til $\lambda_d$).

        Da er $\X+\Y$ ligeledes absolut kontinuert med $\lambda_d$-tæthed $f_{\X+\Y}$ givet ved
        $$f_{\X+\Y}(z)=\int_{\R^d}f_\X(z-y)P_\Y(\deriv y), \quad (z\in\R^d)$$
    \end{lemma}
\end{theorem-box}

\begin{theorem-box}
    \begin{lemma}
        Lad $\X$ og $\mathsf{U}$ være uafhængige d-dimensionale stokastiske vektorer på sandsynlighedsfeltet $\pfield$, og antag, at $\mathsf{U}=(\mathsf{U}_1, \dots, \mathsf{U}_d)$, hvor $\mathsf{U}_1, \dots, \mathsf{U}_d$ er uafhængige identisk $N(0,1)$-fordelte stokastiske variable. 
        
        For ethvert $\sigma$ i $(0,\infty)$ gælder der da, at $\X+\sigma\mathsf{U}$ er absolut kontinuert med tæthed $f_{\X+\sigma\mathsf{U}}$ givet ved:
        $$f_{\X+\sigma\mathsf{U}}(t)=(2\pi)^{-d}\int_{\R^d}e^{-\frac{1}{2}\sigma^2\|s\|^2}e^{-\icomp \langle t,s\rangle}\varx(s)\lambda_d(\deriv s), \quad (t\in\R^d),$$
        hvor $\varx$ er den karakteristiske funktion for $\X$
    \end{lemma}
\end{theorem-box}
\begin{theorem-box}
    \begin{lemma}
        Lad $\X$ og $\mathsf{U}$ være d-dimensionale stokastiske vektorer defineret på $\pfield$, og betragt for ethvert n i $\N$ den stokastiske vektor $\X+\frac{1}{n}\mathsf{U}$.

        For enhver funktion $\psi$ fra $C_b(\R^d,\C)$ gælder der da, at 
        $$\int_{\R^d}\psi(t)P_{\X+\frac{1}{n}\mathsf{U}}(\deriv t)\xrightarrow[n\rightarrow \infty]{} \int_{\R^d}\psi(t)P_\X(\deriv t)$$
    \end{lemma}
\end{theorem-box}
\begin{theorem-box}
    \begin{proposition}
        \begin{enumerate}
            \item[\textnormal{(i)}] Lad $\X$ og $\Y$ være d-dimensionale stokastiske vektorer. Da gælder implikationen $$\varx=\varphi_\Y\Longrightarrow \X\sim\Y$$
            \item[\textnormal{(ii)}] Lad $\mu$ og $\nu$ være sandsynlighedsmål på $(\R^d, \B(\R^d))$. Da gælder implikationen:
            $$\hat{\mu}=\hat{\nu}\Longrightarrow \mu = \nu$$ 
        \end{enumerate}
    \end{proposition}
\end{theorem-box}
\begin{remark}
    $\X$ og $\Y$ behøver ikke at være defineret på samme sandsynlighedsfelt. Der kan derfor findes stokastiske variable $\tilde{\X}, \tilde{\Y}$ således at $\tilde{\X}\sim \X$ og $\tilde{\Y}\sim \Y$ og ræssonere:$$\varx = \varphi_\Y \Longleftrightarrow \varphi_{\tilde{\X}}=\varphi_{\tilde{\Y}}\Longrightarrow \tilde{\X}\sim \tilde{\Y} \Longrightarrow \X \sim \Y$$
\end{remark}
\begin{theorem-box}
    \begin{corollary}
        Lad $\X$ og $\Y$ være hhv. d- og m-dimensionale stokastiske vektorer definerede på sandsynlighedsfeltet $\pfield$. Da er $\X$ og $\Y$ uafhængige, hvis og kun hvis der gælder, at $$\varphi_{(\X,\Y)}(t,s)=\varx(t)\varphi_\Y(s)\quad \text{for alle }t\text{ i }\R^d, \text{og }s\text{ i }\R^m$$
    \end{corollary}
\end{theorem-box}
\begin{theorem-box}
    \begin{proposition}[Inversionssætningen for karakteristiske funktioner]
        Lad $\X$ være en d-dimensional stokastisk vektor på sandsynlighedsfeltet $\pfield$, og antag, at dens karakteristiske funktion $\varx$ er element i $\lclass^1_{\C}(\lambda_d)$. Da er $P_\X$ absolut kontinuert med tæthed $f_\X$ givet ved:
        $$f_\X(t)=(2\pi)^{-d}\int_{\R^d}e^{\icomp \langle t,s\rangle}\varx(s)\lambda_d(\deriv s), \quad (t\in\R^d)$$
    \end{proposition}
\end{theorem-box}
\begin{proof}
    Proof
\end{proof}
\subsection{Differentiabilitet og Taylor-udvikling for karakteristiske funktioner}
\begin{theorem-box}
    \begin{proposition}[Differentiabilitet for den karakteristiske funktion]
        \begin{enumerate}
            \item[\textnormal{(i)}] Lad $\mu$ være et sandsynlighedsmål på $(\R, \B(\R))$, og antag at $p\in\N_0$, således at $\int_\R|x|^p\mu(\deriv x)<\infty$.
            \\Da er $\hat{\mu}$ p-gange differentiabel med afledede
            $$\hat{\mu}^{(k)}(t)=\icomp^k\int_{\R}\mathsf{x}^ke^{\icomp\mathsf{x}t}\mu(\deriv x), \quad (t\in\R, k=0,1,\dots, p).$$
            \item[\textnormal{(ii)}] Lad $\X$ være en stokastisk variabel defineret på $\pfield$, og antag, at $p\in\N_0$, således at $\E[|\X|^p]<\infty$.
            \\Da er $\varx$ p-gange differentiabel med afledede:
            $$\varx^{(k)}(t)=\icomp\E[\X^ke^{\icomp t\X}], \quad (t\in\R, k=0,1,\dots, p).$$ 
            Specielt er 
            $$\E[\X^k]=\icomp^{-k}\varx^{(k)}(0), \quad (k=0,1,\dots, p).$$
        \end{enumerate}
    \end{proposition}
\end{theorem-box}
Bemærk at ii) følger af i)
\begin{theorem-box}
    \begin{lemma}
        For hvert $n\in\N_0$ defineres funktionen $r_n:\R\rightarrow \C$ ved $$r_n(t)=e^{\icomp t}-\sum_{k=0}^n\frac{\icomp^k t^k}{k!}, \quad (t\in\R).$$
        Da gælder vurderingen:
        $$|r_n(t)|\leq \frac{2|t|^n}{n!}\wedge \frac{|t|^{n+1}}{(n+1)!},\quad  (t\in\R).$$
    \end{lemma}
\end{theorem-box}
Ovenstående kan også skrives som 
$$|r_n(t)|\leq \min\{\frac{2|t|^n}{n!},\frac{|t|^{n+1}}{(n+1)!}\},\quad  (t\in\R).$$
\begin{theorem-box}
    \begin{corollary}
        Lax $\X$ være en stokastisk variabel på $\pfield$, således at $\E[X^2]<\infty$. For ethvert $\alpha$ i $[2,3]$ gælder da vurderingen:
        $$\left|\varx(t)-1-\icomp t\E[\X]+\frac{1}{2}t^2\E[\X^2]\right|\leq |t|^\alpha\E[|\X|^\alpha], \quad (t\in\R).$$
    \end{corollary}
\end{theorem-box}
\begin{theorem-box}
    \begin{corollary}
        Lad $\X$ være en stokastisk variabel på $\pfield$, og antag, at $\sigma^2:=\E[\X^2]<\infty$, samt at $\E[\X]=0.$
        
        Da gælder der, at 
        $$\frac{\varx(t)-1}{t^2}\longrightarrow -\frac{\sigma^2}{2}\quad \text{ for }t\rightarrow 0.$$
    \end{corollary}
\end{theorem-box}
\begin{theorem-box}
    \begin{proposition}[Taylor-udvikling af den karakteristiske funktion]
        Lad $\X$ være en stokastisk variabel på $\pfield$ med momenter af enhver orden. Antag yderligere at følgende betingelse er opfyldt: 
        \begin{align}\exists\rho\in(0,\infty):\lim_{n\rightarrow\infty}\frac{\rho^n\E[|\X|^n]}{n!}=0,\end{align}
        og vælg et $\rho$ i henhold hertil. For ethvert a i $\R$ gælder der da, at Taylor-rækken for $\varx$ i a er konvergent i $[a-\rho, a+\rho]$ med sum $\varx$, dsv.
        $$\varx(t)=\sum_{k=0}^{\infty}\frac{\varx^{(k)}(a)}{k!}(t-a)^k=\sum_{k=0}^{\infty}\frac{\icomp^k\E[\X^ke^{\icomp a\X}]}{k!}(t-a)^k, \quad (t\in[a-\rho, a+\rho]).$$ 
    \end{proposition}
\end{theorem-box}
\begin{remark}
    Betingelsen (1) er ækvivalent med følgende betingelse:
    $$\exists c\in(0,\infty)\forall n\in\N:\E[|\X|^n]\leq c^n n!$$
    Betingelsen er altså en begrænsning på hvor hurtigt momenterne må vokse med $n$.
\end{remark}
\subsection{Anvendelser af den karakteristiske funktion}
\begin{theorem-box}
    \begin{proposition}
        \begin{enumerate}
            \item[\textnormal{(i)}] Lad $\X$ være en symmetrisk stokastisk variabel, og lad videre $\X_1,\X_2,\X_3,\dots$ være i.i.d stokastiske variable, således at $\X_n\sim \X$ for alle n.
            
            Hvis yderligere $\X\sim\frac{\X_1+\cdots+\X_n}{\sqrt{n}}$ for alle n, da gælder der, at $\X\sim N(0,\sigma^2)$ for passende $\sigma$ i $[0,\infty)$
            \item[\textnormal{(ii)}] Lad $\X$ være en stokastisk variabel, og antag at $\sigma^2:=\E[\X^2]<\infty.$ Antag endvidere, at $\X\sim\frac{\X_1+\X_2}{\sqrt{2}}$, hvor $\X_1,\X_2$ er i.i.d, og $\X_1\sim \X$.

            Da gælder der, at $\X\sim N(0,\sigma^2)$
        \end{enumerate}
    \end{proposition}
\end{theorem-box}
For $\sigma=0$ tænker vi på $\X$ som dirac-målet.
\begin{theorem-box}
    \begin{lemma}
        Lad $(a_n)_n\in\N$ være en følge af komplekse tal, således at $a_n\longrightarrow a \in\C$ for $n\rightarrow \infty$. Da gælder der, at $$\lim_{n\rightarrow\infty}\left(1+\frac{a_n}{n}\right)^n=\exp(a),$$
        hvor $\exp(a)=e^{\operatorname{Re}(a)}(\cos(\operatorname*{Im}(a)))+\icomp\sin(\operatorname{Im}(a)).$
    \end{lemma}
\end{theorem-box}
\subsection{Momentproblemet}
\begin{theorem-box}
    \begin{problemstilling}[Momentproblemet]
        Lad $\X$ og $\Y$ være to stokastiske variable, og antag, at $\E[|\X|^p], \E[|\Y|^p]<\infty$ for alle p i $\N$, samt at
        $$\E[|\X|^p]=\E[|\Y|^p] \quad \text{for alle p i }\N$$
        Under hvilke yderligere betingelser kan man da slutte at $X\sim\Y$?
    \end{problemstilling}
\end{theorem-box}
\begin{theorem-box}
    \begin{proposition}
        Lad $\X$ og $\Y$ være to stokastiske variable, og antag, at $\E[|\X|^p], \E[|\Y|^p]<\infty$ for alle p i $\N$, samt at
        $$\E[|\X|^p]=\E[|\Y|^p] \quad \text{for alle p i }\N$$
        Hvis yderligere 
        $$\exists\rho\in(0,\infty):\E[e^{\rho|\X|}]<\infty,$$
        da gælder der, at $\X\sim\Y$
    \end{proposition}
\end{theorem-box}
Til beviset for Sætning 1.5.2 får vi brog for følgende lemma:
\begin{theorem-box}
    \begin{lemma}
        Lad $\X$ være en stokastisk variabel på $\pfield$. Da er følgende betingelser ækvivalente:
        \begin{enumerate}
            \item[\textnormal{(i)}] $\exists\rho\in(0,\infty):\E[e^{\rho|\X|}]<\infty.$
            \item[\textnormal{(ii)}] $\exists c\in(0,\infty)\forall n\in\N:\E[|\X|^n]\leq c^nn!.$ 
            \item[\textnormal{(iii)}] $\exists c\in(0,\infty)\forall n\in\N:\E[\X^{2n}]\leq c^{2n}(2n)!.$ 
        \end{enumerate}
    \end{lemma}
\end{theorem-box}
\begin{theorem-box}
    \begin{corollary}
        Lad $\X$ og $\Y$ være stokastiske variable på $\pfield$, og antag, at $P_\X$ og $P_\Y$ begge har kompakt støtte, dvs.
        $$\exists b<0:P(\X\in[-b,b])=1=P(\Y\in[-b,b])$$
        Da $\X$ og $\Y$ har momenter af enhver orden. Hvis yderligere $\E[\X^p]=\E[\Y^p]$ for alle p i $\N$, da gælder der, at $\X\sim\Y.$
    \end{corollary}
\end{theorem-box}
\begin{theorem-box}
    \begin{corollary}
        Lad $\X$ være en ikke-negativ stokastisk variabel, og betragt dens \textbf{Laplace transformerede}:
        $$L_\X(s)=\E[e^{-s\X}],\quad (s\in[0,\infty)).$$
        Da er $P_\X$ entydigt bestemet af $L_\X$. Med andre ord: Hvis $\Y$ er en anden ikke-negativ stokastisk variabel, således at $\L_\Y(s)=L_\X(s)$ for alle s i $[0,\infty)$, da gælder der, at $\X\sim\Y.$
    \end{corollary}
\end{theorem-box}
\section{Konvergens i mål og i sandsynlighed}
\subsection{De tre fundamentale konvergenstyper og deres indbyrdes styrkeforhold}
\begin{theorem-box}
    \begin{definition}
        Lad $(X,\mathcal{E}, \mu)$ være et målrum, og lad $(f_n)_{n\in\N}$ være en følge af funktioner fra $\mathcal{M}(\mathcal{E})$, og lad $f$ være endnu en funktion fra $\mathcal{M}(\mathcal{E}).$ Lad endvidere p være et (strengt) positivt tal. Vi siger da, at 
        \begin{enumerate}
            \item[\textnormal{(a)}] $f_n$ konvergerer mod f i $\mu-$mål for $n\rightarrow \infty$, hvis $$\forall\epsilon>0:\mu\left(\{x\in\X\big{|}|f_n(x)-f(x)|>\epsilon\}\right)\longrightarrow 0 \quad \text{for }n\rightarrow \infty.$$
            I så fald benyttes notationen: $f_n\rightarrow f$ i $\mu-$mål.
            \item[\textnormal{(b)}] $f_n$ konvergerer mod f $\mu-$n.o. for $n\rightarrow\infty$, hvis $$\mu\left(\{x\in X|\lim_{n\rightarrow \infty}f_n(x)=f(x)\}^c\right)=0.$$
            I så fald benyttes notationen: $f_n\rightarrow f \mu-$n.o.
            \item[\textnormal{(c)}] $f_n$ konvergerer mod $f$ i $\mu-p$ middel for $n\rightarrow \infty$, hvis 
            $$\int_X|f_n-f|^p\deriv\mu\longrightarrow 0, \quad \text{for }n\rightarrow \infty$$
            I så fald benyttes notationen: $f_n\rightarrow f$ i $\mu-p$-middel.
        \end{enumerate}
    \end{definition}
\end{theorem-box}
\begin{remark}
    Blandt andet linearitet bevarer konvergens.
\end{remark}
\begin{theorem-box}
    \begin{proposition}
        Lad $"\rightarrow"$ betegne én af de tre konvergensformer indført i Definition 2.1.1, og betragt funktioner $f,g,f_1,f_2,f_3,\dots$ fra $\mathcal{M}(\mathcal{E})$. Da gælder implikationen:
        $$f_n\longrightarrow f, \;\; og \;\; f_n\longrightarrow g \;\;\Longrightarrow\;\;f=g\;\; \mu-\text{n.o}$$
    \end{proposition}
\end{theorem-box}
\begin{theorem-box}
    \begin{proposition}
        Lad $(f_n)_{n\in\N}$ være en følge af funktioner fra $\mathcal{M}(\mathcal{E})$, og lad f være endnu en funktion fra $\mathcal{M}(\mathcal{E})$. Lad endvidere p være et positivt tal. Da gælder følgende udsagn:
        \begin{enumerate}
            \item[\textnormal{(i)}] Hvis $f_n\rightarrow f$ i $\mu-p$ middel, så gælder der også at $f_n\rightarrow f$ i $\mu-$mål.
            \item[\textnormal{(ii)}] Hvis $\sum_{n=1}^\infty\int_X|f_n-f|^p\deriv\mu<\infty$, så gælder der, at $f_n\rightarrow f \mu-n.o.$
            \item[\textnormal{(iii)}] Hvis $f_n\rightarrow f$ i $\mu-$mål, så findes en voksende følge $(n_k)_{k\in\N}$ af naturlige tal, således at $f_{n_k}\rightarrow f \mu$-n.o. for $k\rightarrow \infty$ 
        \end{enumerate}
    \end{proposition}
\end{theorem-box}
\begin{theorem-box}
    \begin{proposition}
        Antag, at $\mu$ er et \textbf{endeligt mål}, lad $(f_n)_{n\in\N}$ være en følge af funktioner fra $\mathcal{M}(\mathcal{E})$, og lad f være endnu en funktion fra $\mathcal{M}(\mathcal{E})$. Lad endvidere p,r være positive tal. Da gælder følgende udsagn:
        \begin{enumerate}
            \item[\textnormal{(i)}] Følgende betingelser er endbetydende:
            \begin{enumerate}
                \item[\textnormal{(i1)}] $f_n \rightarrow f \text{i }\mu-\text{mål}.$
                \item[\textnormal{(i2)}] $\forall K\in(0,\infty):\lim_{n\rightarrow\infty}\int_\X|f_n-f| \wedge K \deriv \mu = 0.$
                \item[\textnormal{(i3)}] $\lim_{n\rightarrow \infty}\int_{\X}|f_n-f|\wedge 1 \deriv \mu = 0.$ 
            \end{enumerate}
            \item[\textnormal{(ii)}] Hvis $f_n\rightarrow f$ $\mu-$n.o., så gælder der også, at $f_n\rightarrow f$ i $\mu-$mål.
            \item[\textnormal{(iii)}] Hvis $r<p$, og $f_n\rightarrow f$ i $\mu-$p-middel, da gælder der også at $f_n\rightarrow f$ i $\mu-$r-middel.
            \end{enumerate}
    \end{proposition}
\end{theorem-box}
\subsection{Fuldstændighed}
\begin{theorem-box}
    \begin{definition}
        Lad $(X,\mathcal{E},\mu)$ være et målrum, og lad $(f_n)_{n\in\N}$ være en følge af funktioner fra $\mathcal{M}(\mathcal{E})$. Lad endvidere p være et strengt positivt tal. Vi siger da, at 
        \begin{enumerate}
            \item[\textnormal{(a)}] $(f_n)_{n\in\N}$ er en \textbf{Cauchy-følge} i $\mu-$mål, hvis
            $$\forall\epsilon>0: \lim_{n,m\rightarrow \infty}\mu\left(\{|f_n-f_m|>\epsilon\}\right)=0.$$
            eller udskrevet hvis 
            $$\forall\epsilon,\delta>0\exists N\in\N\forall n,m\geq N: \mu\left(\{|f_n-f_m|>\epsilon\}\right)\leq \delta$$
            \item[\textnormal{(b)}] $(f_n)_{n\in\N}$ er en \textbf{Cauchy-følge} $\mu-$n.o., hvis $\mu(F^C)=0$, hvor
            $$F=\{x\in X|(f_n(x))_{n\in\N}\text{ er en Cauchy-følge i }\R\}$$
            \item[\textnormal{(c)}] $(f_n)_{n\in\N}$ er en \textbf{Cauchy-følge} i $\mu-p$-middel, hvis
            $$\lim_{n,m\rightarrow \infty}\int_\X f_n-f_m|^p\deriv \mu  = 0,$$
            eller udskrevet hvis 
            $$\forall \epsilon>0\exists N\in\N\forall n,m\geq N:\int_\X |f_n-f_m|^p\deriv\mu\leq\epsilon.$$
        \end{enumerate}
    \end{definition}
\end{theorem-box}
\begin{remark}
    Mængden $F$ er målelig - det følger af omskrivningen $$F=\bigcap_{K\in\N}\bigcup_{N\in\N}\bigcap_{n,m\geq N}\left\{x\in X\big{|}|f_n(x)-f_m(x)|\leq \frac{1}{K}\right\}.$$
\end{remark}
\begin{theorem-box}
    \begin{lemma}
        Lad $(X,\mathcal{E},\mu)$ være et målrum, og lad $(f_n)_{n\in\N}$ være en følge af funktioner fra $\mathcal{M}(\mathcal{E})$. Da gælder følgende udsagn:
        \begin{enumerate}
            \item[\textnormal{(i)}] Lad f være endnu en funktion fra $\mathcal{M}(\mathcal{E}),$ og antag, at der findes en følge $(\epsilon_n)_{n\in\N}$ af (strengt) positive tal, således at
            $$\lim_{n\rightarrow \infty}\epsilon_n = 0, \quad \text{og}\quad \sum_{n=1}^{\infty}\mu\left(\left\{|f_n-f|>\epsilon_n\right\}\right)<\infty.$$
            Da gælder der, at
            $$f_n\rightarrow\; \mu-\text{n.o.,}\quad \text{og}\quad f_n\rightarrow f \; \text{i }\mu\text{-mål.}$$
            \item[\textnormal{(ii)}] Antag, at der findes en følge $(\epsilon_n)_{n\in\N}$ af (strengt) positive tal, således at
            $$\sum_{n=1}^{\infty}\epsilon_n < \infty,\quad \text{og} \quad \sum_{n=1}^{\infty} \mu\left(\left\{|f_{n+1}-f_n|>\epsilon_n\right\}\right)<\infty $$
            Da findes der en funtkion f fra $\mathcal{M}(\mathcal{E})$, således at
            $$f_n \rightarrow f \;\mu\text{-n.o.,}\quad \text{og} \quad f_n\rightarrow f \; \text{i }\mu\text{-mål.}$$ 
        \end{enumerate}
    \end{lemma}
\end{theorem-box}
\begin{theorem-box}
    \begin{proposition}
        Lad $(X,\mathcal{E},\mu)$ være et målrum, og lad $(f_n)_{n\in\N}$ være en følge af funktioner fra $\mathcal{M}(\mathcal{E})$. Da er følgende betingelser ækvivalente:
        \begin{enumerate}
            \item[\textnormal{(i)}] Der findes en funktion f fra $\mathcal{M}(\mathcal{E})$, således at $f_n\rightarrow f$ i $\mu-$mål.
            \item[\textnormal{(ii)}] $(f_n)_{n\in\N}$ er en Cauchy-følge i $\mu-$mål.
        \end{enumerate}
        Med andre ord er konvergens i $\mu$-mål et fuldstændigt konvergensbegreb.
        \end{proposition}
\end{theorem-box}
\begin{theorem-box}
    \begin{corollary}
        Lad $(X,\mathcal{E}, \mu)$ være et målrum, lad $(f_n)$ være en følge af funktioner fra $\mathcal{M}(\mathcal{E})$, og lad p være et strengt positivt tal. Da er følgende betingelser ævkvivalente:
        \begin{enumerate}
            \item[\textnormal{(i)}] Der findes en funktion f fra $\mathcal{M}(\mathcal{E})$, således at $f_n\rightarrow f$ i $\mu$-p-middel.
            \item[\textnormal{(ii)}] $(f_n)$ er en Cauchy-følge i $\mu$-p-middel.   
        \end{enumerate}
    \end{corollary}
\end{theorem-box}
\begin{theorem-box}
    \begin{corollary}
        Lad $(X,\mathcal{E},\mu)$ være et målrum, lad p være et tal i $[1,\infty)$, og lad $(f_n)$ være en følge af funktioner fra $\lclass^p(\mu)$. Da gælder implikationen:
        $$\sum_{n=1}^{\infty}\|f_n\|_p<\infty\quad\Longrightarrow\quad \sum_{n=1}^{\infty}f_n\text{ er konvergent i }\mu\text{-p-middel.}$$
        Med andre ord gælder der, at \textbf{absolut konvergens medfører konvergens} i $\lclass^p(\mu)$.
    \end{corollary}
\end{theorem-box}
\subsection{Konvergens af $f_n$ vs. konvergens af $|f_n|^p$}
\subsection{Konvergens i sandsynlighed}
\begin{theorem-box}
    \begin{definition}
        Lad $(\X_n)$ være en følge af stokastiske variable defineret på sandsynlighedsfeltet $\pfield$, og lad $\X$ være endnu en stokastisk variabel herpå. Lad endvidere r være et positivt tal. Vi siger da, at $\X_n$ konvergerer mod $\X$
        \begin{itemize}
            \item \textbf{i sandsynlighed}, hvis der for ethvert positivt $\epsilon$ gælder, at $$\lim_{n\rightarrow\infty}P(|\X_n-\X|>\epsilon)=0.$$ I bekræftende fald skriver vi: $\X_n\stackrel{P}{\longrightarrow}\X$ for $n\rightarrow \infty.$
            \item \textbf{i r-middel}, hvis $$\lim_{n\rightarrow\infty}\E\left[|\X_n-\X|^r\right]=0.$$ I bekræftende fald skriver vi: $\X_n\stackrel{\lclass^r(P)}{\longrightarrow}\X$ for $n\rightarrow \infty.$
            \item \textbf{P-næsten overalt (eller P-næsten sikkert)}, hvis $$P(\lim_{n\rightarrow\infty}\X_n=\X)=1,$$ eller mere udførligt, hvis $P(F)=1,$ hvor $$F=\{\omega\in\Omega\big{|}\lim_{n\rightarrow\infty}\X_n(\omega)=\X(\omega)\}\in \mathcal{F}.$$ I begræftende fald skriver vi: $\X_n\stackrel{\text{n.o.}}{\longrightarrow}\X$ (eller $\X_n\stackrel{\text{n.s.}}{\longrightarrow}\X$) for $n\rightarrow\infty.$
        \end{itemize}
    \end{definition}
\end{theorem-box}
\subsection{Konvergens i sandsynlighed på generelle metriske rum}
\begin{theorem-box}
    \begin{definition}[Produktmetrikker]
        Lad $(S, \rho)$ og $(T, \delta)$ betegne metriske rum.
En metrik $\eta$ på $S \times T$ kaldes en \textbf{produktmetrik}, hvis den opfylder følgende betingelse:

For alle $(x, y),\left(x_1, y_1\right),\left(x_2, y_2\right),\left(x_3, y_3\right), \ldots$ i $S \times T$ galder bi-implikationen:

$$
\lim _{n \rightarrow \infty} \eta\left(\left(x_n, y_n\right),(x, y)\right)=0 \quad\Longleftrightarrow \quad\lim _{n \rightarrow \infty} \rho\left(x_n, x\right)=\lim _{n \rightarrow \infty} \delta\left(y_n, y\right)=0
$$

    \end{definition}
\end{theorem-box}
\begin{remark}
    Afbildningen $\rho:S\times S\rightarrow\R$ er $\mathcal(B)(S\times S)-\mathcal{B}(\R)$-målelig.
\end{remark}
\begin{theorem-box}
    \begin{definition}[Borel-algebraen på $S\times S$]
        Lad $(S, \rho)$ være et metrisk rum.
Borel-algebraen $\mathcal{B}(S \times S)$ på $S \times S$ defineres da ved
$$
\mathcal{B}(S \times S)=\sigma(\mathcal{G}(\eta))
$$
hvor $\eta$ er en vilkårlig produktmetrik på $S \times S$.
    \end{definition}
\end{theorem-box}
\begin{remark}
    Hvis $(S,\rho)$ er separabelt, så gælder: $$\mathcal{B}(S\times S)=\mathcal{B}(S)\otimes\mathcal{B}(S).$$ Ydermere hvis $\X,\Y$ er stokastiske funktioner på sandsynlighedsfeltet $\pfield$ med værdier i et separabelt metrisk rum $(S,\rho)$, da er afbildningen $$D:=\rho(\X,\Y):\Omega\rightarrow \R$$ $\mathcal{F}-\mathcal{B}(\R)$-målelig.
\end{remark}
\begin{theorem-box}
    \begin{definition}
        Lad $(S, \rho)$ være et separabelt metrisk rum, og lad $\X, \X_1, \X_2, \X_3, \ldots$ være stokastiske funktioner på $(\Omega, \mathcal{F}, P)$ med værdier i $(S, \rho)$.

Vi siger da, at
\begin{enumerate}
    \item[\textnormal{(a)}] $\X_n$ konvergerer mod $\X$ næsten overalt (skrevet: $\X_n \xrightarrow{\text{n.o.}} \X$ ), hvis $P(F^C)=0$, hvor

    $$
    F=\left\{\omega \in \Omega \mid \lim _{n \rightarrow \infty} \rho\left(\X_n(\omega), \X(\omega)\right)=0\right\}
    $$
    
    \item[\textnormal{(b)}] $\X_n$ konvergerer mod $\X$ i sandsynlighed (skrevet: $\X_n \xrightarrow{\mathrm{P}} \X$), hvis

    $$
    \forall \epsilon>0: \lim _{n \rightarrow \infty} P\left(\rho\left(\X_n, \X\right)>\epsilon\right)=0
    $$
\end{enumerate}
    \end{definition}
\end{theorem-box}
\begin{remark}
    Betragt for hert $n$ i $\mathbb{N}$ den stokastiske variable $D_n:=\rho\left(\X_n, \X\right)$. Så har vi bi-implikationerne:
$$
\X_n \xrightarrow{\text { n.o. }} \X \Longleftrightarrow D_n \xrightarrow{\text {n.o.}} 0, \quad \text { og } \quad \X_n \xrightarrow{\mathrm{P}} \X \Longleftrightarrow D_n \xrightarrow{\mathrm{P}} 0 .
$$

\end{remark}
\begin{theorem-box}
    \begin{proposition}
        Lad $(S, \rho)$ være et separabelt metrisk rum, og lad $\X, \X_1, \X_2, \X_3, \ldots$ være stokastiske funktioner på $(\Omega, \mathcal{F}, P)$ med værdier i $(S, \rho)$.
Da gælder følgende udsagn:
\begin{enumerate}
    \item[\textnormal{(i)}] $\X_n \xrightarrow{\text { n.o. }} \X \Longrightarrow \X_n \xrightarrow{\mathrm{P}} \X$.
    \item[\textnormal{(ii)}] Hvis $\X_n \xrightarrow{P} \X$, findes en voksende følge $n_1<n_2<n_3<\cdots$ af naturlige tal, således at $\X_{n_k} \xrightarrow{\text { n.o. }} \X$.
    \item[\textnormal{(iii)}] $\X_n \xrightarrow{P} \X \Longleftrightarrow \lim _{n \rightarrow \infty} \mathbb{E}\left[\rho\left(\X_n, \X\right) \wedge 1\right]=0$.

\end{enumerate}
    \end{proposition}
\end{theorem-box}
\begin{theorem-box}
    \begin{proposition}
        Lad $(S, \rho)$ være et separabelt metrisk rum, og lad $\X, \X_1, \X_2, \X_3, \ldots$ være stokastiske funktioner på $(\Omega, \mathcal{F}, P)$ med værdier i $(S, \rho)$.
Betragt endvidere endnu et separabelt metrisk rum ( $T, \delta$ ), og en $\mathcal{B}(S)-\mathcal{B}(T)$-målelig afbildning $f: S \rightarrow T$.
Antag, at der findes en mængde $C$ i $\mathcal{B}(S)$, således at $$P(\X \in C)=1, \quad \text{og}\quad f \text{ er kontinuert i ethvert punkt x fra C}.$$
Da gælder følgende implikationer:
\begin{enumerate}
    \item[\textnormal{(i)}]$\X_n \xrightarrow{\text { n.o. }} \X \Longrightarrow f\left(\X_n\right) \xrightarrow{\text{n.o.}} f(\X)$.
    \item[\textnormal{(ii)}]$\X_n \xrightarrow{P} \X \Longrightarrow f\left(\X_n\right) \xrightarrow{P} f(\X)$.
\end{enumerate}
    \end{proposition}
\end{theorem-box}
\begin{remark}
    Antag, at $\rho, \rho^{\prime}$ er to ækvivalente metrikker på $S$, således at $(S, \rho)$ og $\left(S, \rho^{\prime}\right)$ er separable.

Betragt afbildningerne id: $(S, \rho) \rightarrow\left(S, \rho^{\prime}\right) \circ \mathrm{og} \mathrm{id}^{\prime}:\left(S, \rho^{\prime}\right) \rightarrow(S, \rho)$ givet ved

$$
\operatorname{id}(x)=\operatorname{id}^{\prime}(x)=x, \quad(x \in S)
$$


Da $\rho$ og $\rho^{\prime}$ er ækvivalente, er id og id ${ }^{\prime}$ begge kontinuerte.
Det følger derfor umiddelbart fra Sætning 2.5.8, at

$$
\X_n \xrightarrow{\text { n.o. } / \mathrm{P}} \X \text { mht. } \rho \Longrightarrow \X_n=\mathrm{id}\left(\X_n\right) \xrightarrow{\text { n.o. } / \mathrm{P}} \mathrm{id}(\X)=\X \text { mht. } \rho^{\prime} .
$$


Overgang til en ækvivalent metrik ændrer altså ikke på, om $\X_n \rightarrow \X$ n.o./ i sandsynlighed eller ej.
\end{remark}
\begin{theorem-box}
    \begin{proposition}
        Lad $(S, \rho)$ og $(T, \delta)$ være separable metriske rum, og lad $\X, \X_1, \X_2, \X_3, \ldots$ samt $\Y, \Y_1, \Y_2, \Y_3, \ldots$ være stokastiske funktioner på $(\Omega, \mathcal{F}, P)$ med værdier i hhv. $(S, \rho)$ og $(T, \delta)$.
        Udstyr endvidere $S \times T$ med en produktmetrik $\eta$.
        Da gælder bi-implikationerne:
        \begin{enumerate}
            \item[\textnormal{(i)}] $\left(\X_n, \Y_n\right) \xrightarrow{\text { n.o. }}(\X, \Y) \Longleftrightarrow \X_n \xrightarrow{\text { n.o. }} \X$ og $\Y_n \xrightarrow{\text { n.o. }} \Y$.
            \item[\textnormal{(ii)}] $\left(\X_n, \Y_n\right) \xrightarrow{P}(\X, \Y) \Longleftrightarrow \X_n \xrightarrow{P} \X$ og $\Y_n \xrightarrow{P} \Y$.
        \end{enumerate}
    \end{proposition}
\end{theorem-box}
\section{Uniform integrabilitet}
\subsection{Definition og indledende begreber}
\begin{theorem-box}
    \begin{definition}
        En delmængde $\mathcal{H}$ af $\mathcal{M}(\mathcal{E})$ siges at være uniformt integrabel (mht. $\mu$ ), hvis den opfylder følgende betingelse:

$$
\forall \epsilon>0 \exists K>0 \forall f \in \mathcal{H}: \int_{\{|f|>K\}}|f| \mathrm{d} \mu \leq \epsilon .
$$

eller ækvivalent:

$$
\forall \epsilon>0 \exists K>0: \sup _{f \in \mathcal{H}} \int_{\{|f|>K\}}|f| \mathrm{d} \mu \leq \epsilon .
$$

    \end{definition}
\end{theorem-box}
\begin{remark}
    \begin{enumerate}
        \item[\textnormal{(i)}] Hvis $\mathcal{H}$ er uniformt integrabel, da gælder der automatisk at $\mathcal{H} \subseteq \mathcal{L}^1(\mu)$.
        For hvis $\mathcal{H}$ er uniformt integrabel kan vi f.eks. vælge $K>0$, således at
        $$
        \sup _{f \in \mathcal{H}} \int_{\{|f|>K\}}|f| \mathrm{d} \mu \leq 1
        $$
        For hvert $f$ fra $\mathcal{H}$ har vi da, at
        $$
        \begin{aligned}
        \int_X|f| \mathrm{d} \mu & =\int_{\{|f| \leq K\}}|f| \mathrm{d} \mu+\int_{\{|f|>K\}}|f| \mathrm{d} \mu \\
        & \leq \int_{\{|f| \leq K\}} K \mathrm{~d} \mu+1 \leq K \mu(X)+1<\infty
        \end{aligned}
        $$
        \item[\textnormal{(ii)}] Hvis $\mathcal{H}$ er uniformt integrabel, gælder dette også enhver delmængde $\mathcal{H}_0$ af $\mathcal{H}$.

        Hvis $\mathcal{H}_1, \ldots, \mathcal{H}_n$ er endeligt mange uniformt integrable delmængder af $\mathcal{M}(\mathcal{E})$, da er $\bigcup_{j=1}^n \mathcal{H}_j$ ligeledes uniformt integrabel.
        
        Specielt fremgår det, at enhver endelig delmængde $\left\{f_1, \ldots, f_n\right\}$ af $\mathcal{L}^1(\mu)$ er uniformt integrabel.
    \end{enumerate}
\end{remark}
\begin{theorem-box}
    \begin{lemma}
        Lad $\mathcal{H}$ være en delmængde af $\mathcal{M}(\mathcal{E})$, og lad $(f_n)$ og $(g_n)$ være følger af funktioner fra $\mathcal{M}(\mathcal{E})$.
        \begin{enumerate}
            \item[\textnormal{(i)}] Hvis $\mathcal{H}$ er uniformt integrabel, da er også mængden 
            $$\tilde{\mathcal{H}}:=\{f\in\mathcal{M}(\mathcal{E})\big{|}\exists g\in\mathcal{H}:|f|\leq |g| \mu\text{-n.o.}\},$$
            uniformt integrabel.
            \item[\textnormal{(ii)}] For enhver funktion $g$ fra $\lclass^1(\mu)^+$ er mængden $\{f\in\mathcal{M}(\mathcal{E})\big{|}|f|\leq g \mu\text{-n.o.}\}$ uniformt integrabel.
            \item[\textnormal{(iii)}] Hvis mængden $\{g_n|n\in\N\}$ er uniformt integrabel, og $|f_n|\leq|g_n| \mu$-n.o. for alle n, da er mængden $\{f_n|n\in\N\}$ ligeledes uniformt integrabel.
        \end{enumerate}
    \end{lemma}
\end{theorem-box}
\begin{theorem-box}
    \begin{proposition}
        En delmængde $\mathcal{H}$ af $\mathcal{M}(\mathcal{E})$ er uniformt integrabel, hvis og kun hvis den opfylder følgende to betingelser:
        \begin{enumerate}
            \item[\textnormal{(i)}]$\sup _{f \in \mathcal{H}} \int_X|f| \mathrm{d} \mu<\infty$.
            \item[\textnormal{(ii)}] $\forall \epsilon>0 \exists \delta>0 \forall A \in \mathcal{E}: \mu(A) \leq \delta \Longrightarrow \sup _{\delta \in \nu} \int_{\Delta}|f| \mathrm{d} \mu \leq \epsilon$.
        \end{enumerate}
    \end{proposition}
\end{theorem-box}
\begin{theorem-box}
    \begin{corollary}
        Antag, at $\mathcal{H}_1$ og $\mathcal{H}_2$ er to uniformt integrable delmængder af $\mathcal{M}(\mathcal{E})$.
Da er mængden

$$
\mathcal{H}_1+\mathcal{H}_2=\left\{f_1+f_2 \mid f_1 \in \mathcal{H}_1, f_2 \in \mathcal{H}_2\right\}
$$

også uniformt integrabel.
    \end{corollary}
\end{theorem-box}
\begin{theorem-box}
    \begin{proposition}
        Lad $\mathcal{H}$ være en delmængde af $\mathcal{M}(\mathcal{E})$, og antag, at der findes en Borel-målelig funktion $\varphi:[0, \infty) \rightarrow[0, \infty)$, således at følgende to betingelser er opfyldte:
        \begin{enumerate}
            \item[\textnormal{(i)}] $\lim _{x \rightarrow \infty} \frac{x}{\varphi(x)}=0$
            \item[\textnormal{(ii)}] $\sup _{f \in \mathcal{H}} \int_X \varphi \circ|f| \mathrm{d} \mu<\infty$.
        \end{enumerate}
Da er $\mathcal{H}$ uniformt integrabel.
    \end{proposition}
\end{theorem-box}
\subsection{Uniform integrabilitet vs. konvergens i $\mu$-middel}
\begin{theorem-box}
    \begin{proposition}
        Lad $\left(f_n\right)_{n \in \mathbb{N}}$ være en følge af funktioner fra $\mathcal{M}(\mathcal{E})$, og lad $f$ være endnu en funktion fra $\mathcal{M}(\mathcal{E})$.

Da er følgende betingelser ækvivalente:
\begin{enumerate}
    \item[\textnormal{(i)}] $f \in \mathcal{L}^1(\mu), f_n \in \mathcal{L}^1(\mu)$ for alle $n$, og $f_n \rightarrow f$ i $\mu$-1-middel.
    \item[\textnormal{(ii)}] $f_n \rightarrow$ f i $\mu$-mål, og mængden $\mathcal{H}=\left\{f_n \mid n \in \mathbb{N}\right\}$ er uniformt integrabel.
\end{enumerate}
    \end{proposition}
\end{theorem-box}
\begin{theorem-box}
    \begin{corollary}
        Lad $\left(f_n\right)_{n \in \mathbb{N}}$ være en følge af funktioner fra $\mathcal{M}(\mathcal{E})$, lad $f$ være endnu en funktion fra $\mathcal{M}(\mathcal{E})$, og lad $p$ være et tal $i(0, \infty)$.

Da er følgende betingelser ækvivalente:
\begin{enumerate}
    \item[\textnormal{(i$_p$)}] $f \in \mathcal{L}^p(\mu), f_n \in \mathcal{L}^p(\mu)$ for alle $n$, og $f_n \rightarrow f$ i $\mu$-p-middel.
    \item[\textnormal{(ii$_p$)}] $f_n \rightarrow f$ i $\mu$-mål, og mængden $\mathcal{H}=\left\{\left|f_n\right| p \mid n \in \mathbb{N}\right\}$ er uniformt integrabel.
\end{enumerate}
    \end{corollary}
\end{theorem-box}
\section{Summer af uafhængige stokastiske variable og store tals stærke lov}
\subsection{Lévys Ulighed}
\begin{theorem-box}
    \begin{proposition}[Lévys Ulighed]
        Lad $\mathsf{X}_1, \ldots, \mathsf{X}_n$ være uafhængige, symmetriske stokastiske variable på $(\Omega, \mathcal{F}, P)$. Da gælder uligheden:

$$
P\left(\max _{k=1, \ldots, n}\left|\sum_{j=1}^k \mathsf{X}_j\right|>t\right) \leq 2 P\left(\left|\sum_{j=1}^n \mathsf{X}_j\right|>t\right) \quad \text { for alle } t i(0, \infty) \text {. }
$$


Hvis vi sætter

$$
\mathsf{S}_k=\mathsf{X}_1+\cdots+\mathsf{X}_k, \quad(k \in\{1,2, \ldots, n\})
$$

og

$$
\mathsf{M}_n=\max _{k=1, \ldots, n}\left|\mathsf{S}_k\right|
$$

da kan uligheden skrives:

$$
P\left(M_n>t\right) \leq 2 P\left(\left|\mathsf{S}_n\right|>t\right) \quad \text { for alle t i } (0, \infty) .
$$

    \end{proposition}
\end{theorem-box}
\begin{theorem-box}
    \begin{corollary}
        Lad $\mathsf{X}_1, \ldots, \mathsf{X}_n$ være uafhængige, symmetriske stokastiske variable på $(\Omega, \mathcal{F}, P)$.

Sæt

$$
\mathsf{S}_k=\mathsf{X}_1+\cdots+\mathsf{X}_k, \quad(k \in\{1,2, \ldots, n\})
$$

og

$$
\mathsf{M}_n=\max _{k=1, \ldots, n}\left|\mathsf{S}_k\right|
$$


Da gælder uligheden:

$$
\mathbb{E}\left[\mathsf{M}_n^p\right] \leq 2 \mathbb{E}\left[\left|\mathsf{S}_n\right|^p\right] \quad \text { for alle } p \text { i }(0, \infty)
$$

    \end{corollary}
\end{theorem-box}
\subsection{Konvergens af summer af uafhænige stokastiske variable}
\begin{theorem-box}
    \begin{lemma}
        Lad $\left(Y_n\right)$ være en følge af stokastiske variable på $(\Omega, \mathcal{F}, P)$, og definér for hvert $p$ i $\mathbb{N}$:

$$
\mathsf{L}_p=\sup _{k, \ell \geq p}\left|\mathsf{Y}_k-\mathsf{Y}_{\ell}\right| \in \overline{\mathcal{M}}(\mathcal{F})^{+}
$$
Da er følgende to udsagn ækvivalente:
\begin{enumerate}
    \item[\textnormal{(i)}] Der findes en stokastisk variabel $\Y$ på $(\Omega, \mathcal{F}, P)$, således at $\mathsf{Y}_n \rightarrow \mathsf{Y}$ P-n.o. for $n \rightarrow \infty$.
    \item[\textnormal{(ii)}] $\mathsf{L}_p \wedge 1 \rightarrow 0$ i sandsynlighed for $p \rightarrow \infty$.
\end{enumerate}
    \end{lemma}
\end{theorem-box}
\begin{theorem-box}
    \begin{lemma}
        Lad $\left(\mathsf{X}_n\right)_{n \in \mathbb{N}}$ være en følge af uafhængige, symmetriske stokastiske variable på $(\Omega, \mathcal{F}, P)$. Da gælder bi-implikationen:
$$\sum_{n=1}^{\infty} \mathsf{X}_n \text{ konvergerer P-n.o. } \Longleftrightarrow \sum_{n=1}^{\infty} \mathsf{X}_n \text{ konvergerer i sandsynlighed.}$$
    \end{lemma}
\end{theorem-box}
\begin{remark}[Det målelige rum $(\R^{\infty},\mathcal{B}(\R^{\infty}))$]
    Betragt vektorrummet

$$
\mathbb{R}^{\infty}=\left\{\left(x_n\right)_{n \in \mathbb{N}} \mid x_n \in \mathbb{R} \text { for alle } n \text { i } \mathbb{N}\right\}
$$


For $n$ i $\mathbb{N}$ og mængder $B_1, \ldots, B_n$ i $\mathcal{B}(\mathbb{R})$ sætter vi

$$
\left[B_1 \times \cdots \times B_n \times \mathbb{R} \times \mathbb{R} \times \cdots\right]=\left\{\left(x_n\right)_{n \in \mathbb{N}} \in \mathbb{R}^{\infty} \mid x_1 \in B_1, \ldots, x_n \in B_n\right\}
$$


Vi sætter endvidere

$$
\mathcal{J}=\left\{\left[B_1 \times \cdots \times B_n \times \mathbb{R} \times \mathbb{R} \times \cdots\right] \mid n \in \mathbb{N}, B_1, \ldots, B_n \in \mathcal{B}(\mathbb{R})\right\}
$$

og

$$
\mathcal{B}\left(\mathbb{R}^{\infty}\right)=\sigma(\mathcal{J})
$$
For hvert $k$ i $\mathbb{N}$ betragter vi afbildningen $p_k: \mathbb{R}^{\infty} \rightarrow \mathbb{R}$ givet ved

$$
p_k\left(\left(x_n\right)_{n \in \mathbb{N}}\right)=x_k, \quad\left(\left(x_n\right)_{n \in \mathbb{N}} \in \mathbb{R}^{\infty}\right)
$$


Vi bemærker for $B_k$ i $\mathcal{B}(\mathbb{R})$, at

$$
\begin{aligned}
p_k^{-1}\left(B_k\right) & =\left\{\left(x_n\right)_{n \in \mathbb{N}} \in \mathbb{R}^{\infty} \mid x_k \in B_k\right\} \\
& =\underbrace{[\mathbb{R} \times \cdots \times \mathbb{R}}_{k-1 \text { gange }} \times B_k \times \mathbb{R} \times \mathbb{R} \times \cdots] \in \mathcal{J} \subseteq \mathcal{B}\left(\mathbb{R}^{\infty}\right)
\end{aligned}
$$

og for $B_1, \ldots, B_n$ i $\mathcal{B}(\mathbb{R})$, at

$$
\left[B_1 \times B_2 \times \cdots \times B_n \times \mathbb{R} \times \mathbb{R} \times \cdots\right]=p_1^{-1}\left(B_1\right) \cap p_2^{-1}\left(B_2\right) \cap \cdots \cap p_n^{-1}\left(B_n\right)
$$


Dermed er $\mathcal{B}\left(\mathbb{R}^{\infty}\right)$ den mindste $\sigma$-algebra på $\mathbb{R}^{\infty}$, som $\mathrm{g} \not$ r $p_1, p_2, p_3, \ldots$ målelige.
Bemærk specielt, at

$$
\begin{aligned}
C & :=\left\{\left(x_n\right)_{n \in \mathbb{N}} \in \mathbb{R}^{\infty} \mid \lim _{n \rightarrow \infty} x_n \text { eksisterer i } \mathbb{R}\right\} \\
& =\left\{\left(x_n\right)_{n \in \mathbb{N}} \in \mathbb{R}^{\infty} \mid\left(x_n\right)_{n \in \mathbb{N}} \text { er en Cauchy-følge }\right\} \\
& =\bigcap_{m \in \mathbb{N}} \bigcup_{N \in \mathbb{N} k, \ell \geq N} \bigcap_n\left\{\left(x_n\right)_{n \in \mathbb{N}} \in \mathbb{R}^{\infty}| | p_k\left(\left(x_n\right)\right)-p_{\ell}\left(\left(x_n\right)\right) \left\lvert\, \leq \frac{1}{m}\right.\right\}=: A \\
& =\bigcap_{m \in \mathbb{N}} \bigcup_{N \in \mathbb{N} k, \ell \geq N} \bigcap_N\left(p_k-p_{\ell}\right)^{-1}\left(\left[-\frac{1}{m}, \frac{1}{m}\right]\right) \in \mathcal{B}\left(\mathbb{R}^{\infty}\right) .
\end{aligned}
$$
\end{remark}
\begin{remark}[Den simultane fordeling af en følge af stokastiske variable]
    Betragt nu et sandsynlighedsfelt $(\Omega, \mathcal{F}, P)$ og en følge $\left(\X_n\right)_{n \in \mathbb{N}}$ af stokastiske variable defineret herpå.
    Vi kan da betragte afbildningen $\mathbb{X}: \Omega \rightarrow \mathbb{R}^{\infty}$ givet ved
    
    $$
    \mathbb{X}(\omega)=\left(X_n(\omega)\right)_{n \in \mathbb{N}}, \quad(\omega \in \Omega)
    $$
    
    
    Vi bemærker, at $\mathbb{X}$ er $\mathcal{F}-\mathcal{B}\left(\mathbb{R}^{\infty}\right)$-målelig:
    
    $$
    \mathbb{X}^{-1}\left(\left[B_1 \times \cdots \times B_n \times \mathbb{R} \times \mathbb{R} \cdots\right]\right)=\left\{\X_1 \in B_1\right\} \cap \cdots \cap\left\{\X_n \in B_n\right\} \stackrel{?}{\in} \mathcal{F}
    $$
    
    for alle $n$ i $\mathbb{N}$ og $B_1, \ldots, B_n \in \mathcal{B}(\mathbb{R})$.
    Dermed kan vi betragte fordelingen $P_{\X}$ af $\mathbb{X}$, dvs. ssh-målet
    
    $$
    P_{\X}(A)=P(\mathbb{X} \in A)=P\left(\mathbb{X}^{-1}(A)\right), \quad\left(A \in \mathcal{B}\left(\mathbb{R}^{\infty}\right)\right)
    $$
    
    
    Da $\mathcal{J}$ er $\cap$-stabilt, er $P_{\X}$ entydigt bestemt af tallene:
    
    $$
    P_{\X}\left(\left[B_1 \times \cdots \times B_n \times \mathbb{R} \times \mathbb{R} \times \cdots\right]\right)=P\left(\X_1 \in B_1, \ldots, \X_n \in B_n\right)
    $$
    
    for $n \in \mathbb{N}$ og $B_1, \ldots, B_n \in \mathcal{B}(\mathbb{R})$ (jvf. Sætn. 2.2.1 i [M\&I]).
\end{remark}
\begin{remark}[Konvergens i termer af den simultane fordeling] 
    Vi bemærker specielt, at

$$
\left(\X_n\right)_{n \in \mathbb{N}} \text { konvergerer n.o. } \Longleftrightarrow P(\mathbb{X} \in C)=1 \Longleftrightarrow P_{\X}(C)=1
$$

og at
$\left(\X_n\right)_{n \in \mathbb{N}}$ konvergerer i ssh. $\Longleftrightarrow\left(\X_n\right)_{n \in \mathbb{N}}$ er en Cauchy-følge i ssh.

$$
\begin{aligned}
& \Longleftrightarrow \forall \epsilon>0: \lim _{n, m \rightarrow \infty} P\left(\left|\X_n-\X_m\right|>\epsilon\right)=0 \\
& \Longleftrightarrow \forall \epsilon>0: \lim _{n, m \rightarrow \infty} P_{\X}\left(\left(p_n-p_m\right)^{-1}\left([-\epsilon, \epsilon]^c\right)\right)=0 .
\end{aligned}
$$


Dermed afhænger konvergens n.o. og i ssh. kun af $P_{\X}$.
Hvis $P_{\X}=P_{\Y}$ gælder der altså, at
$\left(\X_n\right)_{n \in \mathbb{N}}$ konvergerer i ssh./n.o. $\Longleftrightarrow\left(\Y_n\right)_{n \in \mathbb{N}}$ konvergerer i ssh./n.o.
og at
$\left(\sum_{k=1}^n \X_k\right)_{n \in \mathbb{N}}$ konv. i ssh./n.o. $\Longleftrightarrow\left(\sum_{k=1}^n \Y_k\right)_{n \in \mathbb{N}}$ konv. i ssh./n.o.
\end{remark}
\begin{theorem-box}
    \begin{lemma}
        Lad $\left(\mathsf{X}_n\right)_{n \in \mathbb{N}}$ være en følge af uafhængige stokastiske variable på $(\Omega, \mathcal{F}, P)$.
Antag endvidere, at der findes endnu en følge $\left(\mathsf{Y}_n\right)_{n \in \mathbb{N}}$ af stokastiske variable på $(\Omega, \mathcal{F}, P)$, således at

$$
\mathbb{X}=\left(\mathsf{X}_n\right)_{n \in \mathbb{N}} \text { og } \mathbb{Y}=\left(\mathsf{Y}_n\right)_{n \in \mathbb{N}} \text { er uafhængige, og } \quad P_{\mathbb{X}}=P_{\mathbb{Y}}
$$


Da gælder bi-implikationen:

$$
\sum_{n=1}^{\infty} \mathsf{X}_n \text { konvergerer } P \text {-n.o. } \Longleftrightarrow \sum_{n=1}^{\infty} \mathsf{X}_n \text { konvergerer } i \text { sandsynlighed. }
$$

    \end{lemma}
\end{theorem-box}
\begin{theorem-box}
    \begin{proposition}
        Lad $\left(\mathsf{X}_n\right)_{n \in \mathbb{N}}$ være en følge af uafhængige stokastiske variable på $(\Omega, \mathcal{F}, P)$.
Da gælder bi-implikationen:
$\sum_{n=1}^{\infty} \mathsf{X}_n$ konvergerer $P$-n.o. $\Longleftrightarrow \sum_{n=1}^{\infty} \mathsf{X}_n$ konvergerer i sandsynlighed.
    \end{proposition}
\end{theorem-box}
\begin{theorem-box}
    \begin{corollary}
        Lad $\left(\mathsf{Z}_n\right)_{n \in \mathbb{N}}$ være en følge af uafhængige stokastiske variable på $(\Omega, \mathcal{F}, P)$.
Da gælder for ethvert $r>0$ implikationen:

$$
\sum_{n=1}^{\infty} \mathsf{Z}_n \text { konvergerer i P-r-middel } \Longrightarrow \sum_{n=1}^{\infty} \mathsf{Z}_n \text { konvergerer n.o. }
$$

    \end{corollary}
\end{theorem-box}
\begin{theorem-box}
    \begin{corollary}
        Lad $\left(\mathsf{X}_n\right)_{n \in \mathbb{N}}$ være en følge af ufhængige stokastiske variable på $(\Omega, \mathcal{F}, P)$, og antag, at $\mathsf{X}_n \in \mathcal{L}^2(P)$ for alle $n$.

Sæt endvidere $\mu_n=\mathbb{E}\left[\mathsf{X}_n\right]$ for alle $n$.
Da gælder implikationen:

$$
\sum_{n=1}^{\infty} \mathbb{V}\left[\mathsf{X}_n\right]<\infty \Longrightarrow \sum_{n=1}^{\infty}\left(\X_n-\mu_n\right) \quad \text { konvergerer P-n.o. og i P-2-middel. }
$$

    \end{corollary}
\end{theorem-box}
\subsection{Store tals stærke lov}
\begin{theorem-box}
    \begin{lemma}[Kroneckers lemma]
        Lad $\left(a_n\right)_{n \in \mathbb{N}}$ og $\left(b_n\right)_{n \in \mathbb{N}}$ være følger af reelle tal, således at

$$
0<b_1<b_2<b_3<\cdots, \quad \lim _{n \rightarrow \infty} b_n=\infty
$$

og at
$\sum_{k=1}^{\infty} \frac{a_k}{b_k} \;$ er konvergent i$ \mathbb{R}, \;$ dvs. $\; \lim _{n \rightarrow \infty} \sum_{k=1}^n \frac{a_k}{b_k} \;$ eksisterer i$ \mathbb{R}$.
Da gælder der, at

$$
\frac{1}{b_n} \sum_{k=1}^n a_k \xrightarrow[n \rightarrow \infty]{ } 0
$$

    \end{lemma}
\end{theorem-box}
\begin{theorem-box}
    \begin{proposition}[$\mathcal{L}^2$-udgave af Store tals lov]
        Lad $\left(\mathsf{X}_k\right)_{k \in \mathbb{N}}$ være en følge af uafhængige stokastiske variable på $(\Omega, \mathcal{F}, P)$, og antag, at $\X_k \in \mathcal{L}^2(P)$ for alle $k i \mathbb{N}$.

        Sæt endvidere $\mu_k=\mathbb{E}\left[\mathsf{X}_k\right]$ for alle $k i \mathbb{N}$.
        Da gælder implikationen:
        
        $$
        \sum_{k=1}^{\infty} \frac{\mathbb{V}\left[\mathsf{X}_k\right]}{k^2}<\infty \Longrightarrow \frac{1}{n} \sum_{k=1}^n\left(\mathsf{X}_k-\mu_k\right) \underset{n \rightarrow \infty}{ } 0 \quad \text { n.o. og i 2-middel. }
        $$
                
    \end{proposition}
\end{theorem-box}
\begin{example}
    
\end{example}
\begin{theorem-box}
    \begin{lemma}
        Lad $a, a_1, a_2, a_3, \ldots$ være reelle tal, således at $a_n \rightarrow a$ for $n \rightarrow \infty$.
Da gælder der også, at

$$
\lim _{n \rightarrow \infty} \frac{1}{n} \sum_{j=1}^n a_j=a .
$$

    \end{lemma}
\end{theorem-box}
\begin{theorem-box}
    \begin{lemma}
        \begin{enumerate}
            \item[\textnormal{(i)}] For ethvert naturligt tal $N$ gælder der, at

            $$
            \sum_{n=N}^{\infty} \frac{1}{n^2} \leq \frac{2}{N}
            $$
            
            \item[\textnormal{(ii)}] For ethvert x i $(0, \infty)$ gælder der, at
            
            $$
            \sum_{n \in \mathbb{N}: n \geq x} \frac{1}{n^2} \leq \frac{2}{x}
            $$
            
        \end{enumerate}
    \end{lemma}
\end{theorem-box}
\begin{theorem-box}
    \begin{proposition}[Store tals stærke lov]
        Lad $\left(\mathsf{X}_n\right)_{n \in \mathbb{N}}$ være en følge af i.i.d. stokastiske variable på $(\Omega, \mathcal{F}, P)$, således at $\mathbb{E}\left[\left|\mathsf{X}_1\right|\right]<\infty$, og sæt $\mathbb{E}\left[\mathsf{X}_1\right]=\mu$.

Da gælder der, at

$$
\lim _{n \rightarrow \infty} \frac{1}{n} \sum_{j=1}^n \mathsf{X}_j=\mu \quad \text { P-n.o. og i P-1-middel. }
$$

    \end{proposition}
\end{theorem-box}
\section{Konvergens i fordeling}
\subsection{Svag konvergens og konvergens i fordeling}
\begin{theorem-box}
    \begin{definition}[Svag konvergens af sandsynlighedsmål]
        Lad $(S,\rho)$ være et metrisk rum, og lad $\mu,\mu_1,\mu_2,\ldots$ være sandsynlighedsmål på $(S,\mathcal{B}(S))$. Vi siger da, at $\mu_n$ \textbf{konvergerer svagt} imod $\mu$ for $n\rightarrow \infty$ (skrevet: $\mu_n \xlongrightarrow{\mathrm{w}}\mu$), hvis følgende betingelse er opfyldt:
        $$\forall f\in C_b(S):\;\lim_{n\rightarrow \infty}\int_Sf(s)\mu_n(\deriv s)=\int_Sf(s)\mu(\deriv s).$$
    \end{definition}
\end{theorem-box}
\begin{theorem-box}
    \begin{definition}[Konvergens i fordeling]
        Lad $(S,\rho)$ være et metrisk rum, og lad $\X,\X_1,\X_2,\ldots$ være stokastiske funktioner på $\pfield$ med værdier i $(S,\rho)$.

        Vi siger da, at $\X_n$ konvergerer mod $\X$ \textbf{i fordeling} (skrevet: $\X_n\xlongrightarrow{\sim}\X$), hvis $P_{\X_n}\xlongrightarrow{\mathrm{w}}P_\X$.\\Udskrevet er betingelsen altså:
        $$
        \forall f \in C_b(S): \mathbb{E}\left[f\left(\mathsf{X}_n\right)\right]=\int_S f \mathrm{~d} P_{\mathsf{X}_n} \xlongrightarrow[n \rightarrow \infty]{ } \int_S f \mathrm{~d} P_{\mathsf{X}}=\mathbb{E}[f(\mathsf{X})] .
        $$
        
        
        Hvis $\mu$ er et sandsynlighedsmål på $(S, \mathcal{B}(S))$, siger vi tilsvarende, at $\X_n$ konvergerer mod $\mu$ i fordeling (skrevet: $\mathsf{X}_n \xrightarrow{\sim} \mu$ ), hvis $P_{\mathsf{X}_n} \xrightarrow{\mathrm{ w }} \mu$.
        
        Udskrevet er betingelsen altså:
        
        $$
        \forall f \in C_b(S): \mathbb{E}\left[f\left(\mathsf{X}_n\right)\right]=\int_S f \mathrm{~d} P_{\mathsf{X}_n} \xrightarrow[n \rightarrow \infty]{ } \int_S f \mathrm{~d} \mu
        $$
        
    \end{definition}
\end{theorem-box}
\begin{remark}
    \begin{enumerate}
        \item \textbf{Udvidelse til komplekse funktioner:}
        \\Lad ( $S, \rho$ ) være et metrisk rum, og lad $\X, \X_1, \X_2, \X_3, \ldots$ være stokastiske funktioner på $(\Omega, \mathcal{F}, P)$ med værdier i $(S, \rho)$.
        
        Lad videre $\mu$ være et sandsynlighedsmål på ( $S, \mathcal{B}(S)$ ), og antag, at $\X_k \xrightarrow{\sim} \mu$ for $k \rightarrow \infty$.
        
        For enhver funktion $f$ i $C_b(S, \mathbb{C})$ har vi oplagt, at $\operatorname{Re}(f), \operatorname{Im}(f) \in C_b(S, \mathbb{R})$, og dermed at
        
        $$
        \begin{aligned}
        \mathbb{E}\left[f\left(\X_k\right)\right] & =\mathbb{E}\left[\operatorname{Re}\left(f\left(\X_k\right)\right)\right]+\mathrm{i} \mathbb{E}\left[\operatorname{lm}\left(f\left(\X_k\right)\right)\right] \\
        & \xrightarrow[k \rightarrow \infty]{ } \int_S \operatorname{Re}(f) \mathrm{d} \mu+\mathrm{i} \int_S \operatorname{Im}(f) \mathrm{d} \mu=\int_S f \mathrm{~d} \mu
        \end{aligned}
        $$
        
        
        Specielt ser vi i tilfældet $S=\mathbb{R}^d$, at
        
        $$
        \varphi_{\X_k}(t)=\mathbb{E}\left[\mathrm{e}^{\mathrm{i}\left(t, \X_k\right\rangle}\right] \underset{k \rightarrow \infty}{\longrightarrow} \int_{\mathbb{R}^n} \mathrm{e}^{\mathrm{i}(t, x)} \mu(\mathrm{d} x)=\hat{\mu}(t) \quad \text { for alle } t \mathrm{i} \mathbb{R}^d .
        $$
        \item \textbf{Overgang til ækvivalent metrik}
        \\Lad $\mathsf{X}, \mathsf{X}_1, \mathsf{X}_2, \mathsf{X}_3, \ldots$ være stokastiske funktioner med værdier i et metrisk $\operatorname{rum}(S, \rho)$.

        Da ændres definitionen af, at $\X_n \xrightarrow{\sim} \X$, ikke, hvis $\rho$ erstattes af en ækvivalent metrik $\rho^{\prime}$ på $S$.
        
        I denne situation gælder der nemlig, at
        
        $$
        C_b(S, \rho)=C_b\left(S, \rho^{\prime}\right)
        $$
        
        og dermed ændres ikke på betingelsen:
        
        $$
        \forall f \in C_b(S, \rho): \mathbb{E}\left[f\left(\X_n\right)\right] \xrightarrow[n \rightarrow \infty]{\longrightarrow} \mathbb{E}[f(\X)]
        $$
        
    \end{enumerate}
\end{remark}
\begin{theorem-box}
    \begin{proposition}[Entydighedssætning for mål]
        Lad $(S, \rho)$ være et metrisk rum, og lad $\mu$ og $\nu$ være to sandsynlighedsmål på $(S, \mathcal{B}(S))$.

Antag videre, at

$$
\int_S f \mathrm{~d} \mu=\int_S f \mathrm{~d} \nu \quad \text { for alle f i }  C_b(S)^{+}
$$


Da er $\mu=\nu$.
    \end{proposition}
\end{theorem-box}
\begin{theorem-box}
    \begin{corollary}[Entydighed af grænse ved konvergens i fordeling]
        Betragt et metrisk rum $(S, \rho)$.
        \begin{enumerate}
            \item Lad $\nu, \mu, \mu_1, \mu_2, \mu_3, \ldots$ være sandsynlighedsmål på $(S, \mathcal{B}(S))$, og antag, at

            $$
            \mu_n \xrightarrow{\mathrm{~w}} \mu, \text { og } \mu_n \xrightarrow{\mathrm{~w}} \nu \text { for } n \rightarrow \infty .
            $$
            
            
            Da gælder $\mu=\nu$.
            \item Lad $\mathsf{Y}, \mathsf{X}, \mathsf{X}_1, \mathsf{X}_2, \mathsf{X}_3, \ldots$ være stokastiske funktioner på $(\Omega, \mathcal{F}, P)$ med værdier $i(S, \rho)$, og antag, at

            $$
            \mathsf{X}_n \xrightarrow{\sim} \mathsf{X}, \quad \text { og } \quad \mathsf{X}_n \xrightarrow{\sim} \mathsf{Y}
            $$
            
            
            Da gælder $\X\sim\Y$.
        \end{enumerate}

    \end{corollary}
\end{theorem-box}
\begin{theorem-box}
    \begin{proposition}[Styrkeforhold]
        Lad $\X, \X_1, \X_2, \X_3, \ldots$ være stokastiske funktioner på $(\Omega, \mathcal{F}, P)$ med værdier $i$ et separabelt metrisk rum $(S, \rho)$.

Da gælder implikationen:

$$
\X_n \xrightarrow{P} \X \Longrightarrow \X_n \xrightarrow{\sim} \X .
$$

    \end{proposition}
\end{theorem-box}
\begin{theorem-box}
    \begin{proposition}
        Lad $\X, \X_1, \X_2, \X_3, \ldots$ være stokastiske funktioner på $(\Omega, \mathcal{F}, P)$ med værdier $i$ et separabelt metrisk rum $(S, \rho)$.

Antag endvidere, at

$$
\exists a \in S: P(X=a)=1
$$


Da gælder bi-implikationen:

$$
\X_n \xrightarrow{\mathrm{P}} \X \Longleftrightarrow \X_n \xrightarrow{\sim} \X .
$$

    \end{proposition}
\end{theorem-box}
\begin{theorem-box}
    \begin{proposition}
        Lad $\X_1, \X_2, \X_3, \ldots$ være stokastiske funktioner på $(\Omega, \mathcal{F}, P)$ med værdier $i$ et metrisk rum $(S, \rho)$, og lad $\mu$ være et sandsynlighedsmål på $(S, \mathcal{B}(S))$.

Antag, at $\X_n \xrightarrow{\sim} \mu$.
Da gælder der, at

$$
\int_S g \mathrm{~d} \mu=\lim _{n \rightarrow \infty} \mathbb{E}\left[g\left(\X_n\right)\right]
$$

for enhver begr. funktion g i $\mathcal{M}(\mathcal{B}(S))$, som er kontinuert i $\mu$-n.a. x i S.
    \end{proposition}
\end{theorem-box}
\begin{theorem-box}
    \begin{definition}[Lipschitz-afbildninger]
        Lad $(S, \rho)$ og $(T, \delta)$ være metriske rum.
En afbildning $f: S \rightarrow T$ siges da at være en Lipschitz afbildning, hvis der findes en konstant $K$ i $(0, \infty)$, således at

$$
\delta(f(x), f(y)) \leq K \rho(x, y) \quad \text { for alle } x, y \text { i } S .
$$


Med $\operatorname{Lip}(S, \rho)$ betegnes mængden af Lipschitz funktioner $f:(S, \rho) \rightarrow \mathbb{R}$.
\\Med $\operatorname{Lip}_b(S, \rho)$ betegnes mængden af begrænsede Lipschitz funktioner $f:(S, \rho) \rightarrow \mathbb{R}$.
    \end{definition}
\end{theorem-box}
\begin{theorem-box}
    \begin{lemma}
        Lad $S$ og $T$ være ikke-tomme mængder, og lad $G: S \times T \rightarrow \mathbb{R}$ være en nedadtil begrænset funktion (dvs. G opfylder, at $\left.\inf _{(x, y) \in S \times T} G(x, y)>-\infty\right)$.

For vilkårlige $x, x^{\prime}$ i S gælder der da, at

$$
\left|\inf _{y \in T} G(x, y)-\inf _{y \in T} G\left(x^{\prime}, y\right)\right| \leq \sup _{y \in T}\left|G(x, y)-G\left(x^{\prime}, y\right)\right| .
$$

    \end{lemma}
\end{theorem-box}
\begin{theorem-box}
    \begin{lemma}
        Lad $(S, \rho)$ være et metrisk rum, og lad $g: S \rightarrow[0, \infty)$ være en vilkårlig ikke-negativ funktion. Betragt for hvert $k i \mathbb{N}$ funktionen $g_k: S \rightarrow[0, \infty)$ givet ved

$$
g_k(x)=\inf _{y \in S}(g(y)+k \rho(x, y)), \quad(x \in S)
$$
\begin{enumerate}
    \item[\textnormal{(i)}] For hvert $k i \mathbb{N}$ er $g_k$ en Lipschitz funktion med konstant $k$ :

    $$
    \left|g_k(x)-g_k\left(x^{\prime}\right)\right| \leq k \rho\left(x, x^{\prime}\right), \quad\left(x, x^{\prime} \in S\right)
    $$
    \item[\textnormal{(ii)}] For vilkårlige $k i \mathbb{N}, x$ iS og $r>0$ gælder der, at

    $$
    0 \leq\left(\inf _{y \in b(x, r)} g(y)\right) \wedge k r \leq g_k(x) \leq g_{k+1}(x) \leq g(x)
    $$
    \item[\textnormal{(iii)}] Hvis $g$ er kontinuert i $x \in S$, gæ/der der, at $g_k(x) \uparrow g(x)$ for $k \rightarrow \infty$.
\end{enumerate}
    \end{lemma}
\end{theorem-box}
\subsection{Portmanteau sætningerne}
\begin{theorem-box}
    \begin{proposition}
        Lad $\X_1, \X_2, \X_3, \ldots$ være stokastiske funktioner på $(\Omega, \mathcal{F}, P)$ med værdier $i$ et metrisk rum $(S, \rho)$, og lad $\mu$ være et sandsynlighedsmål på $(S, \mathcal{B}(S))$.

Da er følgende betingelser ækvivalente:
\begin{enumerate}
    \item[\textnormal{(i)}] $\X_n \xrightarrow{\sim} \mu$.
    \item[\textnormal{(ii)}] $\forall f \in C(S)^{+}: \int_S f \mathrm{~d} \mu \leq \liminf _{n \rightarrow \infty} \mathbb{E}\left[f\left(\X_n\right)\right]$.
    \item[\textnormal{(iii)}] $\forall f \in C_b(S)^{+}: \int_S f \mathrm{~d} \mu \leq \liminf _{n \rightarrow \infty} \mathbb{E}\left[f\left(\X_n\right)\right]$.
\end{enumerate}
    \end{proposition}
\end{theorem-box}
\begin{remark}
\end{remark}
\begin{theorem-box}
\begin{proposition}[Portmanteau Sætning I]
    Lad $\X, \X_1, \X_2, \X_3, \ldots$ være stokastiske funktioner på $(\Omega, \mathcal{F}, P)$ med værdier $i$ et metrisk rum $(S, \rho)$, og lad $\mu$ være et sandsynlighedsmål på $(S, \mathcal{B}(S))$.
    Da er følgende betingelser ækvivalente:
    \begin{enumerate}
        \item[\textnormal{(i)}] $\X_n \xrightarrow{\sim} \mu$.
        \item[\textnormal{(ii)}] $\int_S g \mathrm{~d} \mu \leq \liminf _{n \rightarrow \infty} \mathbb{E}\left[g\left(\X_n\right)\right] \quad$ for alle $g i \operatorname{Lip}_b(S, \rho)^{+}$.
        \item[\textnormal{(iii)}]$\mu(G) \leq \liminf _{n \rightarrow \infty} P\left(\X_n \in G\right) \quad$ for enhver åben delmængde. $G$ af $S$.
        \item[\textnormal{(iv)}] $\mu(F) \geq \underset{n \rightarrow \infty}{\limsup } P\left(\X_n \in F\right) \quad$ for enhver lukket delmængde. $F$ af $S$.
    \end{enumerate}
\end{proposition}
\end{theorem-box}
\begin{theorem-box}
    \begin{corollary}
        Lad $\X, \X_1, \X_2, \X_3, \ldots$ være stokastiske funktioner på $(\Omega, \mathcal{F}, P)$ med værdier $i$ et metrisk rum $(S, \rho)$, og lad $\mu$ være et sandsynlighedsmål på $(S, \mathcal{B}(S))$. Da er følgende betingelser ækvivalente:
\begin{enumerate}
    \item[\textnormal{(i)}] $\X_n \xrightarrow{\sim} \mu$.
    \item[\textnormal{(ii)}] For enhver mængde B i $\mathcal{B}(S)$ gælder der, at

    $$
    \mu\left(B^{\circ}\right) \leq \liminf _{n \rightarrow \infty} P\left(\X_n \in B\right) \leq \limsup _{n \rightarrow \infty} P\left(\X_n \in B\right) \leq \mu(\bar{B})
    $$
    
    hvor $B^{\circ}=\left(\overline{B^c}\right)^c \subseteq\left(B^c\right)^c=B \subseteq \bar{B}$.
\end{enumerate} 
Hvis (i) og (ii) er opfyldte, gælder der yderligere, at

$$
\mu(B)=\lim _{n \rightarrow \infty} P\left(\X_n \in B\right)
$$

for enhver mængde $B$ fra $\mathcal{B}(S)$, således at $\mu\left(\bar{B} \backslash B^{\circ}\right)=0$.
    \end{corollary}
\end{theorem-box}
\begin{theorem-box}
    \begin{proposition}[Portmanteau Sætning II]
        Lad $(S, \rho)$ og $(T, \delta)$ være metriske rum, og udstyr $S \times T$ med en produktmetrik $\eta$.
Betragt endvidere stokastiske funktioner $\X, \X_1, \X_2, \X_3, \ldots$ og $\Y, \Y_1, \Y_2, \Y_3, \ldots$ med værdier i hhv. $(S, \rho)$ og $(T, \delta)$.
Da gælder følgende udsagn:
\begin{enumerate}
    \item[\textnormal{(i)}]Hvis $\X_n \xrightarrow{\sim} \X$, gælder der også, at $f\left(\X_n\right) \xrightarrow{\sim} f(\X)$ for enhver kontinuert afbildning $f: S \rightarrow T$.
    \item[\textnormal{(ii)}]Antag, at $(T, \delta)$ er separabelt. Hvis $\X_n \xrightarrow{\sim} \X, \Y_{\boldsymbol{n}} \xrightarrow{\sim} \Y$, og Y er udartet, så gælder der også, at $\left(\X_n, \Y_n\right) \xrightarrow{\sim}(\X, \Y)$.
    \item[\textnormal{(iii)}]Antag, at $(S, \rho)$ og $(T, \delta)$ er separable. Hvis $\X_n \xrightarrow{\sim} \X, \Y_n \xrightarrow{\sim} \Y$ og $\X_n, \Y_n$ er uafhængige for alle $n$, da gælder der også, at

    $$
    \left(\X_n, \Y_n\right) \xrightarrow{\sim} P_{\X} \otimes P_{\Y}, \quad \text { dvs. } \quad P_{\X_n} \otimes P_{\Y_n} \xrightarrow{\mathrm{w}} P_{\X} \otimes P_{\Y}
    $$
\end{enumerate}


    \end{proposition}
\end{theorem-box}
\begin{remark}
    Udsagn (ii) i Portmanteau II gælder ikke generelt, hvis Y ikke er udartet.
Betragt nemlig f.eks. en symmetrisk stokastisk variabel $\X(\mathrm{dvs} . \X \sim-\X)$, og definér så

$$
\X_k=\X, \quad \text { og } \quad \Y_k=-\X, \quad(k \in \mathbb{N}) .
$$


Så gælder der oplagt, at $\X_k \xrightarrow{\sim} \X$, og $\Y_k \xrightarrow{\sim}-\X \sim \X$.
Hvis (ii) gjaldt generelt, kunne vi så slutte, at $\left(\X_k, \Y_k\right) \xrightarrow{\sim}(\X, \X)$.
Anvendes så (i) i Portmanteau II på funktionen $f(x, y)=x+y$, ville det følge, at

$$
0=\X_k+\Y_k=f\left(\X_k, \Y_k\right) \xrightarrow{\sim} f(\X, \X)=2 \X .
$$


Dette er oplagt forkert, med mindre $\X \sim \delta_0$ (jvf. 5.1.5).
\end{remark}
\subsection{Stramhed}
\begin{theorem-box}
    \begin{definition}[Stramhed]
        Lad $(S, \rho)$ være et metrisk rum, og lad $\mathcal{K}$ betegne systemet af kompakte delmængder af $S$.
\begin{enumerate}
    \item[\textnormal{(a)}] En familie $\mathcal{M}$ af sandsynlighedsmål på $(S, \mathcal{B}(S))$ siges at være stram, hvis følgende betingelse er opfyldt:
$$
\forall \epsilon>0 \exists K \in \mathcal{K}: \sup _{\mu \in \mathcal{M}} \mu\left(K^c\right) \leq \epsilon
$$
\item[\textnormal{(b)}] En familie $\mathcal{H}$ af stokastiske funktioner med værdier i $(S, \rho)$ siges at være stram, hvis mængden $\left\{P_{\X} \mid \X \in \mathcal{H}\right\}$ er stram i henhold til (a); dvs. hvis

$$
\forall \epsilon>0 \exists K \in \mathcal{K}: \sup _{\X \in \mathcal{H}} P\left(\X \in K^c\right) \leq \epsilon,
$$

eller ækvivalent:

$$
\forall \epsilon>0 \exists K \in \mathcal{K}: \inf _{\X \in \mathcal{H}} P(\X \in K) \geq 1-\epsilon
$$
\end{enumerate}


    \end{definition}
\end{theorem-box}
\begin{remark}
    \begin{enumerate}
        \item Lad $\mathcal{M}_1$ og $\mathcal{M}_2$ være to mængder af sandsynlighedsmål på $(S, \mathcal{B}(S))$. Da gælder implikationerne:

        $$
        \begin{aligned}
        \mathcal{M}_2 \text { stram og } \mathcal{M}_1 \subseteq \mathcal{M}_2 & \Longrightarrow \mathcal{M}_1 \text { stram } \\
        \mathcal{M}_1, \mathcal{M}_2 \text { stramme } & \Longrightarrow \mathcal{M}_1 \cup \mathcal{M}_2 \text { stram }
        \end{aligned}
        $$
        \item For ethvert sandsynlighedsmål $\mu$ på $\left(\mathbb{R}^d, \mathcal{B}\left(\mathbb{R}^d\right)\right)$ er $\{\mu\}$ stram. Vi har nemlig (i tilfældet $d=1$ ), at

        $$
        \mu\left([-N, N]^c\right)=1-\mu([-N, N]) \xrightarrow[N \rightarrow \infty]{ } 1-\mu(\mathbb{R})=0
        $$
        
        
        Og her er $[-N, N]$ kompakt for alle $N$ i $\mathbb{N}$.
        
        \item Specielt er enhver endelig familie af ssh.-mål på $\mathbb{R}^d$ (eller af $d$-dim. stokastiske vektorer) automatisk stram.
        \item En familie $\mathcal{H}$ af $d$-dimensionale stokastiske vektorer er stram, hvis der findes $\alpha>0$, således at

        $$
        \sup _{\mathbf{X} \in \mathcal{H}} \mathbb{E}\left[\|\X\|^\alpha\right]<\infty .
        $$
        
        
        Det følger nemlig fra Markovs Ulighed, at
        
        $$
        \sup _{\X \in \mathcal{H}} P(\|\X\|>N) \leq \frac{1}{N^\alpha} \sup _{\X \in \mathcal{H}} \mathbb{E}\left[\|\X\|^\alpha\right] \text {, hvor } \frac{1}{N^\alpha} \rightarrow 0 \text { for } N \rightarrow \infty .
        $$
              
    \end{enumerate}
\end{remark}
\begin{theorem-box}
    \begin{proposition}
        Lad $\left(\X_k\right)_{k \in \mathbb{N}}$ være en følge af d-dimensionale stokastiske vektorer.
Det er følgende betingelser hver især tilstrækkelige for, at $\left\{\X_k \mid k \in \mathbb{N}\right\}$ er stram:
\begin{enumerate}
    \item[\textnormal{(i)}] Der findes et sandsynlighedsmål $\mu$ på $\left(\mathbb{R}^d, \mathcal{B}\left(\mathbb{R}^d\right)\right)$, således at $\X_k \xrightarrow{\sim} \mu$.
    \item[\textnormal{(ii)}] $\forall \epsilon>0 \exists a>0: \liminf _{k \rightarrow \infty} \mathbb{E}\left[\mathrm{e}^{-a\left\|\X_k\right\|^2}\right]>1-\epsilon$. 
\end{enumerate}
    \end{proposition}
\end{theorem-box}
\subsection{Konvergens i fordeling for stokastiske variable}
\begin{theorem-box}
    \begin{proposition}
        Lad $\left(\X_n\right)_{n \in \mathbb{N}}$ være en følge af stokastiske variable på $(\Omega, \mathcal{F}, P)$, og lad $\mu$ være et sandsynlighedsmål på $(\mathbb{R}, \mathcal{B}(\mathbb{R}))$.

Betragt endvidere de tilhørende fordelingsfunktioner:

$$
\begin{aligned}
& F_n(x)=P_{\X_n}((-\infty, x])=P\left(\X_n \in(-\infty, x]\right), \quad(x \in \mathbb{R}, n \in \mathbb{N}), \\
& F_\mu(x)=\mu((-\infty, x]), \quad(x \in \mathbb{R})
\end{aligned}
$$
Da er følgende betingelser ækvivalente:
\begin{enumerate}
    \item[\textnormal{(i)}] $\X_n \xrightarrow{\sim} \mu$.
    \item[\textnormal{(ii)}]$F_\mu(x-) \leq \liminf _{n \rightarrow \infty} F_n(x) \leq \limsup _{n \rightarrow \infty} F_n(x) \leq F_\mu(x)$ for alle $x i \mathbb{R}$.

    \item[\textnormal{(iii)}] $\lim _{n \rightarrow \infty} F_n(x)=F_\mu(x)$ for alle $x i \mathbb{R}$, hvor $\mu(\{x\})=0$.

    \item[\textnormal{(iv)}] Der findes en tæt delmængde $D$ af $\mathbb{R}$, således at

    $$
    \lim _{n \rightarrow \infty} F_n(x)=F_\mu(x) \quad \text { for alle } x i D
    $$
    \item[\textnormal{(v)}]  $\mu((a, b)) \leq \liminf _{n \rightarrow \infty} P\left(a<\X_n<b\right)$ for alle $a, b i \overline{\mathbb{R}}$, så $a<b$.
\end{enumerate}
    \end{proposition}
\end{theorem-box}
\begin{theorem-box}
    \begin{proposition}
        Lad $\X,\X_1,\X_2, \ldots$ være stokastiske variable på $\pfield$, og betragt de tilhørende fordelingsfunktioner $F_\X, F_{\X_1},F_{\X_2},\ldots$. Antag at $\X_n \xrightarrow{\sim} \X$ for $n\rightarrow \infty$, og at $F_\X$ er kontinuert. Da gælder der at 
        $$\sup_{x\in\R}\left|F_{\X_n}(x)-F_\X(x)\right|\xrightarrow[n\rightarrow \infty]{}0,$$
        dvs. $F_{\X_n}\rightarrow F_\X$ uniformt på $\R$
    \end{proposition}
\end{theorem-box}
\begin{theorem-box}
    \begin{definition}[limespunkt]
        Lad ( $\X_k$ ) være en følge af stokastiske funktioner med værdier i et metrisk rum $(S, \rho)$.

Et sandsynlighedsmål $\mu$ på $(S, \mathcal{B}(S))$ kaldes da for et limespunkt for $\left(\X_k\right)_{k \in \mathbb{N}}$, hvis der findes en voksende følge $k_1<k_2<k_3<\cdots$ af naturlige tal, således at

$$
\X_{k_{\ell}} \xrightarrow{\sim} \mu \quad \text { for } \ell \rightarrow \infty
$$

    \end{definition}
\end{theorem-box}
\begin{theorem-box}
    \begin{proposition}[Hellys Lemma]
        Lad $\left(F_k\right)_{k \in \mathbb{N}}$ være en følge af fordelingsfunktioner.
Da findes en voksende følge $k_1<k_2<k_3<\cdots$ af naturlige tal, og en voksende, højrekontinuert funktion $F: \mathbb{R} \rightarrow[0,1]$, således at

$$
\lim _{\ell \rightarrow \infty} F_{k_{\ell}}(x)=F(x) \quad \text { for alle } x \text { i } C_F
$$


Her betegner $C_F$ mængden af kontinuitetspunkter for $F$, dvs.

$$
C_F=\{x \in \mathbb{R} \mid F \text { er kontinuert } i x\}
$$


Specielt gælder der, at $F_{k_{\ell}} \rightarrow F$ punktvist, for $\ell \rightarrow \infty$, hvis $F$ er kontinuert.
    \end{proposition}
\end{theorem-box}
\begin{remark}
    
\end{remark}
\begin{theorem-box}
    \begin{proposition}[Helly-Bray's Sætning]
        Lad $\left(\X_k\right)_{k \in \mathbb{N}}$ være en følge af d-dimensionale stokastiske vektorer, og antag, at $\left\{\X_k \mid k \in \mathbb{N}\right\}$ er stram.

Da gælder følgende udsagn:
\begin{enumerate}
    \item[\textnormal{(i)}] $\left(\X_k\right)_{k \in \mathbb{N}}$ har mindst ét limespunkt.
    \item[\textnormal{(ii)}] Hvis $\left(\X_k\right)_{k \in \mathbb{N}}$ kun har ét limespunkt $\mu$, så gælder der, at $\X_k \xrightarrow{\sim} \mu$ for $k \rightarrow \infty$.
\end{enumerate}
    \end{proposition}
\end{theorem-box}
\subsection{Kontinuitetssætningen}
\begin{theorem-box}
    \begin{corollary}
        Lad $\left(\mathsf{X}_k\right)_{h C N}$ æare en stram folge af $d$-dimensionale stokastiske vektorer, og antag, at

$$
\lim _{k \rightarrow \infty} \varphi_{\mathsf{X}_k}(t) \text { eksisterer } i \mathbb{C} \text { for alle } t i \mathbb{R}^d .
$$


Da findes et sandsynlighedsmål $\mu$ på $\mathbb{R}^d$, således at $\X_k \xrightarrow{\sim} \mu$ for $k \rightarrow \infty$.
Der gælder yderligere, at

$$
\hat{\mu} (t)=\lim _{k \rightarrow \infty} \varphi_{\X_k}(t) \text { for alle } t i \mathbb{R}^d
$$

    \end{corollary}
\end{theorem-box}
\begin{theorem-box}
    \begin{proposition}[Kontinuitetssætningen for karakteristiske funktioner]
        For en følge $\left(\X_k\right)_{k \in \mathbb{N}}$ af d-dimensionale stokastiske vektorer er følgende betingelser ækvivalente:
\begin{enumerate}
    \item[\textnormal{(i)}]  Der findes et ssh-mål $\mu$ på $\left(\mathbb{R}^d, \mathcal{B}\left(\mathbb{R}^d\right)\right)$, således at $X_k \xrightarrow{\sim} \mu$.
    \item[\textnormal{(ii)}] Der findes en funktion $\gamma: \mathbb{R}^d \rightarrow \mathbb{C}$, som er kontinuert i 0 , således at

    $$
    \lim _{k \rightarrow \infty} \varphi_{\X_k}(t)=\gamma(t) \quad \text { for alle } t i \mathbb{R}^d
    $$
    
\end{enumerate}
I bekræftende fald gælder der yderligere, at

$$
\hat{\mu}(t)=\gamma(t) \quad \text { for alle } t \;\text{i}\; \mathbb{R}^d
$$

    \end{proposition}
\end{theorem-box}
\begin{theorem-box}
    \begin{corollary}
        Lad $\X, \X_1, \X_2, \X_3, \ldots$ være d-dimensionale stokastiske vektorer og $\Y, \Y_1, \Y_2, \Y_3, \ldots m$-dimensionale stokastiske vektorer.

Antag endvidere, at $\X_n$ og $\Y_n$ er uafhængige for ethvert $n \text{ i } \mathbb{N}$, samt at $\X_n \xrightarrow{\sim} \X$ og $\Y_n \xrightarrow{\sim} \Y$.

Da gælder der også, at $\left(\X_n, \Y_n\right) \xrightarrow{\sim} P_{\X} \otimes P_{\Y}$.
    \end{corollary}
\end{theorem-box}
\section{Centrale Grænseværdisætninger}
\subsection{Laplaces version}
\begin{theorem-box}
    \begin{proposition}[Laplaces CLT]
        Lad $\left(\X_n\right)_{n \in \mathbb{N}}$ være en følge af i.i.d. stokastiske variable, således at $\mathbb{E}\left[\X_1^2\right]<\infty$.

Sæt endvidere

$$
\sigma^2:=\mathbb{V}\left[\X_1\right], \quad \text { og } \quad \mu:=\mathbb{E}\left[\X_1\right]
$$


Da gælder der, at

$$
\frac{\sum_{k=1}^n \X_k-n \mu}{\sqrt{n \sigma^2}}=\mathrm{U}_n:=\frac{1}{\sqrt{n \sigma^2}} \sum_{k=1}^n\left(\X_k-\mu\right) \stackrel{\sim}{\rightarrow} N(0,1) .
$$

    \end{proposition}
\end{theorem-box}
\subsection{Lindebergs version}
\begin{theorem-box}
    \begin{definition}
        Et uafhængigt trekantsskema er en familie $\left\{\X_{n k} \mid n \in \mathbb{N}, k \in\{1, \ldots, n\}\right\}$ af stokastiske variable, således at

$$
\X_{n 1}, \ldots, \X_{n n} \quad \text { er uafhængige for alle } n \text { i } \mathbb{N} \text {. }
$$


Et uafhængigt trekantsskema anskuelliggøres ofte på formen:

$$
\begin{aligned}
& \X_{11} \\
& \X_{21}, \X_{22} \\
& \X_{31}, \X_{32}, \X_{33} \\
& \vdots \\
& \vdots \\
& \X_{n 1}, \X_{n 2}, \ldots \ldots, \X_{n n}
\end{aligned}
$$

    \end{definition}
\end{theorem-box}
\begin{theorem-box}
    \begin{problemstilling}
        Lad $\{\X_{nk}|n\in\N,1\leq k \leq n\}$ være et uafhængigt trekantsskema. Det tilhørende CLT-problem udgøres da af følgende spørgsmål. Findes der:
        \begin{itemize}
            \item en familie $\{m_{nk}|n\in\N,  1\leq k \leq n\}$ af reelle tal,
            \item en familie $\{a_n|n\in\N\}$ af (strengt) positive tal,
            \item et ikke-udartet sandsynlighedsmål $\mu$ på $(\R, \mathcal{B}(\R))$,
        \end{itemize}
        således at 
        $$\frac{1}{a_n} \sum_{k=1}^n\left(\X_{n k}-m_{n k}\right) \underset{n \rightarrow \infty}{\sim} \mu ?$$
    \end{problemstilling}
\end{theorem-box}
\begin{theorem-box}
    \begin{proposition}[Lindebergs CLT]
        Lad $\left\{\X_{n k} \mid n \in \mathbb{N}, 1 \leq k \leq n\right\}$ være et uafhængigt trekantsskema, således at $\mathbb{E}\left[\X_{n k}^2\right]<\infty$ for alle $n, k$. Sæt

$$
\mu_{n k}=\mathbb{E}\left[\X_{n k}\right], \quad \sigma_{n k}^2=\mathbb{V}\left[\X_{n k}\right], \quad \text { og } \quad s_n=\left(\sigma_{n 1}^2+\sigma_{n 2}^2+\cdots+\sigma_{n n}^2\right)^{1 / 2}
$$

for alle $n, k$, og antag, at $s_n>0$ for alle $n$.
Sæt yderligere

$$
\mathrm{U}_n=\frac{1}{s_n} \sum_{k=1}^n\left(\X_{n k}-\mu_{n k}\right), \quad(n \in \mathbb{N})
$$


Antag endvidere Lindebergs betingelse:

$$
\forall \epsilon>0: \lim _{n \rightarrow \infty} \frac{1}{s_n^2} \sum_{k=1}^n \int_{\left\{\left|\mathsf{X}_{n k}-\mu_{n k}\right|>\epsilon s_n\right\}}\left(\X_{n k}-\mu_{n k}\right)^2 \mathrm{~d} P=0 .
$$


Da gælder der, at $\mathrm{U}_n \xrightarrow{\sim} N(0,1)$.
    \end{proposition}
\end{theorem-box}
\begin{theorem-box}
    \begin{lemma}
        Lad $z_1, \ldots, z_n$ og $w_1, \ldots, w_n$ være komplekse tal, således at $\left|z_k\right| \leq 1$ og $\left|w_k\right| \leq 1$ for alle $k i\{1,2, \ldots, n\}$.

Da gælder uligheden:

$$
\left|\prod_{k=1}^n z_k-\prod_{k=1}^n w_k\right| \leq \sum_{k=1}^n\left|z_k-w_k\right|
$$

    \end{lemma}
\end{theorem-box}
\begin{theorem-box}
    \begin{lemma}
        For ethvert $x$ i $[0,\infty)$ gælder uligheden:
        $$\left|e^{-x}-1+x\right|\leq \frac{1}{2}x^2.$$
    \end{lemma}
\end{theorem-box}
\subsection{Bevis for Laplaces version}
\subsection{Lyapounovs version}
\begin{theorem-box}
    \begin{proposition}[Lyapounovs CLT]
        Lad $\left\{\X_{n k} \mid n \in \mathbb{N}, 1 \leq k \leq n\right\}$ være et uafhængigt trekantsskema, således at $\mathbb{E}\left[\X_{n k}^2\right]<\infty$ for alle $n, k$. Sæt

$$
\mu_{n k}=\mathbb{E}\left[\X_{n k}\right], \quad \sigma_{n k}^2=\mathbb{V}\left[\X_{n k}\right], \quad \text { og } \quad s_n=\left(\sigma_{n 1}^2+\sigma_{n 2}^2+\cdots+\sigma_{n n}^2\right)^{1 / 2}
$$

for alle $n, k$, og antag, at $s_n>0$ for alle $n$.
Sæt yderligere

$$
\mathrm{U}_n=\frac{1}{s_n} \sum_{k=1}^n\left(\X_{n k}-\mu_{n k}\right), \quad(n \in \mathbb{N})
$$


Antag endvidere Lyapounov's betingelse:

$$
\exists \alpha>2: \lim _{n \rightarrow \infty} \frac{1}{s_n^\alpha} \sum_{k=1}^n \mathbb{E}\left[\left|\X_{n k}-\mu_{n k}\right|^\alpha\right]=0
$$


Da gælder der, at $\mathrm{U}_n \xrightarrow{\sim} N(0,1)$.
    \end{proposition}
\end{theorem-box}
\section{Betingede middelværdier}
\subsection{Definition, eksistens og entydighed}
\begin{theorem-box}
    \begin{definition}
        Lad $(\Omega, \mathcal{F}, P)$ være et sandsynlighedsfelt, og lad $\mathcal{B}$ være en del- $\sigma$-algebra af $\mathcal{F}($ dvs. $\mathcal{B} \subseteq \mathcal{F})$.
Lad videre X være en stokastisk variabel fra $\mathcal{L}^1(P)$.
En betinget middelværdi af X givet $\mathcal{B}$ er en stokastisk variabel U på $(\Omega, \mathcal{F}, P)$, der opfylder følgende tre betingelser:
\begin{enumerate}
    \item $U \in \mathcal{L}^1(P)$.
    \item U er $\mathcal{B}$-målelig.
    \item Der gælder, at

    $$
    \int_B \mathrm{U} \mathrm{~d} P=\int_B \mathrm{xd} P \quad \text { for alle } B \text { i } \mathcal{B} .
    $$
\end{enumerate}
    \end{definition}
\end{theorem-box}
\begin{theorem-box}
    \begin{lemma}[Restriktion af mål til del-$\sigma$-algebra.]
        Lad $(\Omega, \mathcal{F}, P)$ være et sandsynlighedsfelt, og lad $\mathcal{B}$ være en del- $\sigma$-algebra af $\mathcal{F}$. Da defineres ved formlen:

$$
P_{\mathcal{B}}(B)=P(B), \quad(B \in \mathcal{B})
$$

et sandsynlighedsmål på $(\Omega, \mathcal{B})$, som kaldes restriktionen af $P$ til $\mathcal{B}$.
Der gælder endvidere, at

$$
\mathcal{L}\left(P_{\mathcal{B}}\right)=\mathcal{L}(P) \cap \overline{\mathcal{M}}(\mathcal{B}), \quad \text { og } \quad \mathcal{L}^1\left(P_{\mathcal{B}}\right)=\mathcal{L}^1(P) \cap \mathcal{M}(\mathcal{B})
$$

samt at

$$
\int_{\Omega} \mathrm{U} \mathrm{~d} P_{\mathcal{B}}=\int_{\Omega} \mathrm{U} \mathrm{~d} P \quad \text { for alle } \mathrm{U} \in \mathcal{L}\left(P_{\mathcal{B}}\right) \subseteq \mathcal{L}(P)
$$
    \end{lemma}
\end{theorem-box}
\begin{theorem-box}
    \begin{corollary}
        Lad $(\Omega, \mathcal{F}, P)$ være et sandsynlighedsfelt, og lad $\mathcal{B}$ være en del- $\sigma$-algebra af $\mathcal{F}$.

Lad videre $\mathcal{D}$ være et $\cap$-stabilt frembringersystem for $\mathcal{B}$, således at $\Omega \in \mathcal{D}$.
Lad endelig U og $\mathrm{U}^{\prime}$ være to $\mathcal{B}$-målelige stokastiske variable fra $\mathcal{L}^1(P)$.
Da gælder følgende bi-implikationer:
\begin{enumerate}
    \item[\textnormal{(i)}] $\mathrm{U} \leq \mathrm{U}^{\prime}$ P-n.o. $\Longleftrightarrow \int_B \mathrm{U} \mathrm{d} P \leq \int_B \mathrm{U}^{\prime} \mathrm{d} P$ for alle $B$ i $\mathcal{B}$.
    \item[\textnormal{(ii)}] $\mathrm{U}=\mathrm{U}^{\prime}$ P-n.o. $\Longleftrightarrow \int_B \mathrm{U} \mathrm{d} P=\int_B \mathrm{U}^{\prime} \mathrm{d} P$ for alle $B ; \mathcal{D}$.
\end{enumerate}
    \end{corollary}
\end{theorem-box}
\begin{theorem-box}
    \begin{proposition}
        Lad $(\Omega, \mathcal{F}, P)$ være et sandsynlighedsfelt, og lad $\mathcal{B}$ være en del- $\sigma$-algebra af $\mathcal{F}$.

For enhver stokastisk variabel X i $\mathcal{L}^1(P)$ findes da en betinget middelværdi U af X givet $\mathcal{B}$.

Hvis U, U' begge er betingede middelværdier af X givet $\mathcal{B}$, da gælder der, at $\mathrm{U}=\mathrm{U}^{\prime} P-n . o$.

Enhver betinget middelværdi af X givet $\mathcal{B}$ betegnes med $\mathbb{E}[\X| \mathcal{B}]$.
    \end{proposition}
\end{theorem-box}
\subsection{Egenskaber for betingede middelværdier}
\begin{theorem-box}
    \begin{proposition}
        Lad $(\Omega, \mathcal{F}, P)$ være et ssh-felt, og lad $\mathcal{B}$ være en del- $\sigma$-algebra af $\mathcal{F}$. Antag videre, at $\X, \Y \in \mathcal{L}^1(P)$, og at $a \in \mathbb{R}$. Da gæ/der følgende udsagn:
\begin{enumerate}
    \item[\textnormal{(i)}]$\mathbb{E}[\mathbb{E}[\X \mid \mathcal{B}]]=\mathbb{E}[\X]$.
    \item[\textnormal{(ii)}] $\mathbb{E}[a \X+\Y \mid \mathcal{B}]=a \mathbb{E}[\X \mid \mathcal{B}]+\mathbb{E}[\Y \mid \mathcal{B}]$ P-n.o.
    \item[\textnormal{(iii)}] Hvis $\X \leq \Y P$-n.o. gælder følgende udsagn:
    \begin{enumerate}
        \item $\mathbb{E}[\X \mid \mathcal{B}] \leq \mathbb{E}[\Y \mid \mathcal{B}]$ P-n.o.
        \item Hvis $\X=\Y$ P-n.o., har vi også, at $\mathbb{E}[\X \mid \mathcal{B}]=\mathbb{E}[\Y \mid \mathcal{B}] P$-n.o.
        \item Hvis $\X<\Y$ P-n.o., har vi også, at $\mathbb{E}[\X \mid \mathcal{B}]<\mathbb{E}[\Y \mid \mathcal{B}] P$-n.o.
        \item Sættes $A=\{\mathbb{E}[\X \mid \mathcal{B}]=\mathbb{E}[\Y \mid \mathcal{B}]\}$, har vi, at $\X 1_A=\Y 1_A$ P-n.o.
    \end{enumerate}
    \item[\textnormal{(iv)}] Hvis X er $\mathcal{B}$-målelig, gælder der, at $\mathbb{E}[\X \mid \mathcal{B}]=\X P$-n.o.
\end{enumerate}
    \end{proposition}
\end{theorem-box}
\begin{theorem-box}
    \begin{proposition}
        Lad $(\Omega, \mathcal{F}, P)$ være et sandsynlighedsfelt, lad $\mathcal{B}$ være en del- $\sigma$-algebra af $\mathcal{F}$, og antag, at $\X \in \mathcal{L}^1(P)$.

Lad videre I være et interval i $\mathbb{R}$ med endepunkter:

$$
v \in[-\infty, \infty), \quad \text { og } \quad h \in(-\infty, \infty]
$$

og antag, at $P(\X \in I)=1$. Da gælder der, at

$$
P(\mathbb{E}[\X \mid \mathcal{B}] \in I)=1
$$

og at

$$
\X=v \text {-n.o. på }\{\mathbb{E}[\X \mid B]=v\}
$$

og at

$$
\X=h \text {-n.o. på }\{\mathbb{E}[\X \mid B]=h\} .
$$
    \end{proposition}
\end{theorem-box}
\begin{theorem-box}
    \begin{terminology}
        Lad X og Y være stokastiske variable på $(\Omega, \mathcal{F}, P)$, og lad $A$ være en mængde fra $\mathcal{F}$.

        Vi siger da (f.eks.), at $\X \geq \Y P$-n.o. på $A$, hvis følgende ækvivalente betingelser er opfyldte:
        \begin{enumerate}
            \item[(1)] $P(A \backslash\{\X \geq \Y\})=0$.
            \item[(2)] $P(A \cap\{\X \geq \Y\})=P(A)$.
            \item[(3)] $\X \1_A \geq \Y \1_A P$-n.o.
        \end{enumerate}
    
        Bemærk nemlig, at
        
        $$
        P(A)=P(A \backslash\{\X \geq \Y\})+P(A \cap\{\X \geq \Y\})
        $$
        
        og at
        
        $$
        P\left(\X \1_A<\Y \1_A\right)=P(A \cap\{\X<\Y\})=P(A \backslash\{\X \geq \Y\})
        $$
        
    \end{terminology}
\end{theorem-box}
\begin{theorem-box}
    \begin{proposition}[Tårnegenskaben]
        $\operatorname{Lad}(\Omega, \mathcal{F}, P)$ være et sandsynlighedsfelt, lad X være en integrabel stokastisk variabel herpå, og lad $\mathcal{B}, \mathcal{B}_1$ være del- $\sigma$-algebraer af $\mathcal{F}$.

Hvis $\mathcal{B}_1 \subseteq \mathcal{B}$, gælder der, at

$$
\mathbb{E}\left[\mathbb{E}[\X \mid \mathcal{B}] \mid \mathcal{B}_1\right]=\mathbb{E}\left[\X \mid \mathcal{B}_1\right]=\mathbb{E}\left[\mathbb{E}\left[\X \mid \mathcal{B}_1\right] \mid \mathcal{B}\right] \quad \text { P-n.o. }
$$
    \end{proposition}
\end{theorem-box}
\subsection{Jensens ulighed for betingede middelværdier}
\begin{theorem-box}
    \begin{lemma}
        Lad I være et interval i $\mathbb{R}$, og lad $\varphi: I \rightarrow \mathbb{R}$ være en konveks funktion.
Da findes en følge $\left(\ell_n\right)_{n \in \mathbb{N}}$ af affine funktioner på $\mathbb{R}$, således at

$$
\varphi(t) \geq \sup _{n \in \mathbb{N}} \ell_n(t) \quad \text { for alle } t i l,
$$

og

$$
\varphi(t)=\sup _{n \in \mathbb{N}} \ell_n(t) \quad \text { for alle } t i l^{\circ} .
$$

    \end{lemma}
\end{theorem-box}
\begin{theorem-box}
    \begin{proposition}
        Lad $(\Omega, \mathcal{F}, P)$ være et sandsynlighedsfelt, og lad $\mathcal{B}$ være en del- $\sigma$-algebra af $\mathcal{F}$. Lad videre Z være en stokastisk variabel i $\mathcal{L}^1(P)$.

For enhver mængde $A$ fra $\mathcal{B}$ gælder der da, at

$$
\mathbb{E}\left[\1_A \mathrm{Z} \mid \mathcal{B}\right]=\1_A \mathbb{E}[\mathrm{Z} \mid \mathcal{B}] \quad \text { P-n.o. }
$$

    \end{proposition}
\end{theorem-box}
\begin{theorem-box}
    \begin{proposition}[Jensens ulighed for betingede middelværdier.]
        Lad $(\Omega, \mathcal{F}, P)$ være et sandsynlighedsfelt, lad $\mathcal{B}$ være en del- $\sigma$-algebra af $\mathcal{F}$, og antag, at $\X \in \mathcal{L}^1(P)$.

Lad videre $\varphi: \mathbb{R} \rightarrow \mathbb{R}$ være en Borel-funktion, som er konveks på et interval $I \subseteq \mathbb{R}$.

Antag, at $P(\X \in I)=1$, og at $\varphi(\X) \in \mathcal{L}^1(P)$.
Da gælder der, at

$$
P(\mathbb{E}[\X \mid \mathcal{B}] \in I)=1, \quad \text { og } \quad \varphi(\mathbb{E}[\X \mid \mathcal{B}]) \leq \mathbb{E}[\varphi(\X) \mid \mathcal{B}] \text { P-n.o. }
$$
    \end{proposition}
\end{theorem-box}
\subsection{Konvergens resultater for betingede middelværdier}
\begin{theorem-box}
    \begin{proposition}
        Lad $(\Omega, \mathcal{F}, P)$ være et sandsynlighedsfelt, og lad $\mathcal{B}$ være en del- $\sigma$-algebra af $\mathcal{F}$.

For ethvert r i $[1, \infty)$ gælder da implikationen:

$$
\X \in \mathcal{L}^r(P) \Longrightarrow \mathbb{E}[\X \mid \mathcal{B}] \in \mathcal{L}^r(P)
$$


Hvis $\X, \Y \in \mathcal{L}^1(P)$, og $\X-\Y \in \mathcal{L}^r(P)$, gæ/der der endvidere, at

$$
\|\mathbb{E}[\X \mid \mathcal{B}]-\mathbb{E}[\Y \mid \mathcal{B}]\|_r \leq\|\X-\Y\|_r
$$


For stokastiske variable $\X, \X_1, \X_2, \X_3, \ldots$ i $\mathcal{L}^1(P)$, kan vi derfor slutte, at

$$
\X_n \rightarrow \X \text { i } r \text {-middel } \Longrightarrow \mathbb{E}\left[\X_n \mid \mathcal{B}\right] \rightarrow \mathbb{E}[\X \mid \mathcal{B}] \text { i } r \text {-middel. }
$$

    \end{proposition}
\end{theorem-box}
\begin{theorem-box}
    \begin{proposition}[Monoton konvergens for betingede middelværdier]
        Lad $(\Omega, \mathcal{F}, P)$ være et sandsynlighedsfelt, lad $\mathcal{B}$ være en del- $\sigma$-algebra af $\mathcal{F}$, og lad $\X, \X_1, \X_2, \X_3, \ldots$ være stokastiske variable i $\mathcal{L}^1(P)$.
        \begin{enumerate}
            \item[\textnormal{(i)}] Hvis $\X_n \uparrow \X$ P-n.o., da gælder der også, at $\mathbb{E}\left[\X_n \mid \mathcal{B}\right] \uparrow \mathbb{E}[\X \mid \mathcal{B}]$ P-n.o. og i 1-middel.
            \item[\textnormal{(ii)}] Hvis $\X_n \downarrow \X$ P-n.o., da gælder der også, at $\mathbb{E}\left[\X_n \mid \mathcal{B}\right] \downarrow \mathbb{E}[\X \mid \mathcal{B}]$ P-n.o. og i 1-middel. 
        \end{enumerate}
    \end{proposition}
\end{theorem-box}
\begin{theorem-box}
    \begin{corollary}[Fatous lemma for betingede middelværdier]
        Lad $(\Omega, \mathcal{F}, P)$ være et sandsynlighedsfelt, og lad $\mathcal{B}$ være en del- $\sigma$-algebra af $\mathcal{F}$.

Lad videre $\X, \X_1, \X_2, \X_3, \ldots$ være stokastiske variable i $\mathcal{L}^1(P)$, og antag, at $\X_n \geq 0$-n.o. for alle $n$ i $\mathbb{N}$, samt at $\X \leq \liminf _{n \rightarrow \infty} \X_n P$-n.o.

Da gælder uligheden:

$$
\mathbb{E}[\X \mid \mathcal{B}] \leq \liminf _{n \rightarrow \infty} \mathbb{E}\left[\X_n \mid \mathcal{B}\right] \text { P-n.o. }
$$


Hvis $\lim \inf _{n \rightarrow \infty} \X_n \in \mathcal{L}^1(P)$, kan vi specielt slutte, at (?)

$$
\mathbb{E}\left[\liminf _{n \rightarrow \infty} \X_n \mid \mathcal{B}\right] \leq \liminf _{n \rightarrow \infty} \mathbb{E}\left[\X_n \mid \mathcal{B}\right] \text { P-n.o. }
$$

    \end{corollary}
\end{theorem-box}
\begin{theorem-box}
    \begin{proposition}[Domineret konvergens for betingede middelværdier]
        Lad $(\Omega, \mathcal{F}, P)$ være et sandsynlighedsfelt, lad $\mathcal{B}$ være en del- $\sigma$-algebra af $\mathcal{F}$, og lad $\X, \X_1, \X_2, \X_3, \ldots$ være stokastiske variable på $(\Omega, \mathcal{F}, P)$.

Antag, at
- $\X_n \rightarrow \X$ P-n.o. for $n \rightarrow \infty$.
- Der findes en stokastisk variabel $\Y$ fra $\mathcal{L}^1(P)^{+}$, således at $\left|\X_n\right| \leq \Y$ $P$-n.o. for alle $n$ i $\mathbb{N}$.

Da gælder der, at $\X \in \mathcal{L}^1(P)$, at $\X_n \in \mathcal{L}^1(P)$ for alle $n$, og at

$$
\mathbb{E}\left[\X_n \mid \mathcal{B}\right] \underset{n \rightarrow \infty}{\longrightarrow} \mathbb{E}[\X \mid \mathcal{B}] \quad \text { P-n.o. og i 1-middel. }
$$
    \end{proposition}
\end{theorem-box}
\subsection{$\mathcal{B}$-målelige variable som konstanter}
\begin{theorem-box}
    \begin{proposition}
        Lad $(\Omega, \mathcal{F}, P)$ være et sandsynlighedsfelt, lad $\mathcal{B}$ være en del- $\sigma$-algebra af $\mathcal{F}$, og lad X være en stokastisk variabel i $\mathcal{L}^1(P)$.

Betragt endvidere to $\mathcal{B}$-målelige stokastiske variable $\mathrm{U}_1$ og $\mathrm{U}_2$, og antag, at

$$
P\left(\mathrm{U}_1 \leq \X \leq \mathrm{U}_2\right)=1
$$


Da gælder der også, at

$$
P\left(\mathrm{U}_1 \leq \mathbb{E}[\X \mid \mathcal{B}] \leq \mathrm{U}_2\right)=1 .
$$

    \end{proposition}
\end{theorem-box}
\begin{theorem-box}
    \begin{proposition}
        Lad $(\Omega, \mathcal{F}, P)$ være et sandsynlighedsfelt, lad $\mathcal{B}$ være en del- $\sigma$-algebra af $\mathcal{F}$, og lad X være en stokastisk variabel i $\mathcal{L}^1(P)$.
Betragt videre en $\mathcal{B}$-målelig stokastisk variabel U , således at $\mathrm{UX} \in \mathcal{L}^1(P)$.
Da gælder der, at

$$
\mathbb{E}[\mathrm{UX} \mid \mathcal{B}]=\mathrm{U} \mathbb{E}[\X \mid \mathcal{B}] \text { P-n.o. }
$$

    \end{proposition}
\end{theorem-box}
\subsection{Uafhængighed vs. betinget middelværdi}
\begin{theorem-box}
    \begin{proposition}
        Lad $(\Omega, \mathcal{F}, P)$ være et sandsynlighedsfelt, lad $\mathcal{B}$ være en del- $\sigma$-algebra af $\mathcal{F}$, og lad X være en stokastisk variabel i $\mathcal{L}^1(P)$.

Antag videre, at X og $\mathcal{B}$ er uafhængige.
Da gælder der, at

$$
\mathbb{E}[\X \mid \mathcal{B}]=\mathbb{E}[\X] \quad \text { P-n.o. }
$$

    \end{proposition}
\end{theorem-box}
\begin{theorem-box}
    \begin{proposition}
        Lad $(\Omega, \mathcal{F}, P)$ være et sandsynlighedsfelt, lad $\mathcal{B}, \mathcal{B}_1$ være del- $\sigma$-algebraer af $\mathcal{F}$, og lad X være en stokastisk variabel i $\mathcal{L}^1(P)$.

Antag, at $(\X, \mathcal{B})$ og $\mathcal{B}_1$ er uafhængige, dvs. at

$$
\sigma(\sigma(\X) \cup \mathcal{B}) \text { og } \mathcal{B}_1 \text { er uafhængige. }
$$


Da gælder formlen:

$$
\mathbb{E}\left[\X \mid \sigma\left(\mathcal{B} \cup \mathcal{B}_1\right)\right]=\mathbb{E}[\X \mid \mathcal{B}] \quad \text { P-n.o. }
$$

    \end{proposition}
\end{theorem-box}
\begin{theorem-box}
    \begin{proposition}
        Lad $(\Omega, \mathcal{F}, P)$ være et sandsynlighedsfelt, lad X, Y være stokastiske variable herpå, og lad $\mathcal{B}$ være en del- $\sigma$-alg. af $\mathcal{F}$.

Antag, at X og $\mathcal{B}$ er uafhængige, og at Y er $\mathcal{B}$-målelig.
Betragt endvidere en begrænset Borel-funktion $H: \mathbb{R}^2 \rightarrow \mathbb{R}$, og indfør funktionen $\tilde{H}: \mathbb{R} \rightarrow \mathbb{R}$ givet ved

$$
\tilde{H}(y)=\mathbb{E}[H(\X, y)], \quad(y \in \mathbb{R})
$$


Da gælder formlen:

$$
\mathbb{E}[H(\X, \Y) \mid \mathcal{B}]=\tilde{H}(\Y) \text { P-n.o. }
$$

    \end{proposition}
\end{theorem-box} 
\subsection{Uniform integrabilitet af betingede middelværdier}
\begin{theorem-box}
    \begin{proposition}
        Lad $(\Omega, \mathcal{F}, P)$ være et sandsynlighedsfelt, og lad $\left(\mathcal{B}_i\right)_{i \in I}$ være en familie af del- $\sigma$-algebraer af $\mathcal{F}$.

Lad videre $\mathcal{H}$ være en uniformt integrabel familie af stokastiske variable på $(\Omega, \mathcal{F}, P)$.

Da er familien

$$
\left\{\mathbb{E}\left[\mathrm{X} \mid \mathcal{B}_i\right] \mid \mathrm{x} \in \mathcal{H}, \quad i \in I\right\}
$$

igen uniformt integrabel.
    \end{proposition}
\end{theorem-box}
\subsection{Betinget middelværdi givet en stokastisk funktion}
\begin{theorem-box}
    \begin{definition}[Betinget middelværdi af $\X$ givet $\Y$]
        Lad $(\Omega, \mathcal{F}, P)$ være et sandsynlighedsfelt, og lad X være en stokastisk variabel i $\mathcal{L}^1(P)$.

        Betragt videre en stokastisk funktion $\mathrm{Y}: \Omega \rightarrow E$ med værdier i et måleligt rum $(E, \mathcal{E})$, og definér:
        
        $$
        \sigma(\mathrm{Y}):=\mathrm{Y}^{-1}(\mathcal{E})=\left\{\mathrm{Y}^{-1}(B) \mid B \in \mathcal{E}\right\}=\{\{\mathrm{Y} \in B\} \mid B \in \mathcal{E}\} \subseteq \mathcal{F}
        $$
        
        
        En betinget middelværdi $\mathbb{E}[\mathrm{X} \mid \mathrm{Y}]$ af X givet Y defineres da ved:
        
        $$
        \mathbb{E}[\mathrm{X} \mid \mathrm{Y}]=\mathbb{E}[\mathrm{X} \mid \sigma(\mathrm{Y})] \quad P \text {-n.o. }
        $$
        
        M.a.o. er $\mathbb{E}[\mathrm{X} \mid \mathrm{Y}]$ en integrabel, $\sigma(\mathrm{Y})$-målelig stokastisk variabel, så
        
        $$
        \int_{\{\mathrm{Y} \in B\}} \mathbb{E}[\mathrm{X} \mid \mathrm{Y}] \mathrm{d} P=\int_{\{\mathrm{Y} \in B\}} \mathrm{X} \mathrm{~d} P \quad \text { for alle } B \mathrm{i} \mathcal{E}
        $$        
    \end{definition}
\end{theorem-box}
\begin{theorem-box}
    \begin{lemma}
        Lad $(\Omega, \mathcal{F}, P)$ være et sandsynlighedsfelt, og lad X være en stokastisk variabel i $\mathcal{L}^1(P)$. Betragt videre en stokastisk funktion $\mathrm{Y}: \Omega \rightarrow E$ med værdier i et måleligt rum $(E, \mathcal{E})$.
\begin{enumerate}
    \item[\textnormal{(i)}] Der findes en funktion $\varphi: E \rightarrow \mathbb{R}$ fra $\mathcal{M}(\mathcal{E})$, således at $\mathbb{E}[\mathrm{X} \mid \mathrm{Y}]=\varphi(\mathrm{Y})$ P-n.o.

    \item[\textnormal{(ii)}] Hvis $\varphi \in \mathcal{M}(\mathcal{E})$ og opfylder, at $\mathbb{E}[\mathrm{X} \mid \mathrm{Y}]=\varphi(\mathrm{Y})$ P-n.o., så gælder der automatisk, at $\varphi \in \mathcal{L}^1\left(P_{\mathrm{Y}}\right)$.
    \item[\textnormal{(iii)}]  Antag, at $\varphi, \psi$ er to funktioner fra $\mathcal{M}(\mathcal{E})$, således at

    $$
    \varphi(\mathrm{Y})=\mathbb{E}[\mathrm{X} \mid \mathrm{Y}]=\psi(\mathrm{Y}) \quad \text { P-n.o. }
    $$
    Så gælder der automatisk, at $\psi(y)=\varphi(y)$ for $P_Y-n$.a. y i $E$.
\end{enumerate}
    \end{lemma}
\end{theorem-box}
\begin{theorem-box}
    \begin{definition}[Betinget middelværdi af $\X$ givet $\Y$]
        Lad $(\Omega, \mathcal{F}, P)$ være et sandsynlighedsfelt, og lad X være en stokastisk variabel i $\mathcal{L}^1(P)$.

Betragt videre en stokastisk funktion $\mathrm{Y}: \Omega \rightarrow E$ med værdier i et måleligt $\operatorname{rum}(E, \mathcal{E})$.

Enhver funktion $\varphi$ i $\mathcal{M}(\mathcal{E})$ der opfylder, at

$$
\mathbb{E}[\mathrm{X} \mid \mathrm{Y}]=\varphi(\mathrm{Y}), P \text {-n.o., }
$$

kaldes en betinget middelværdi af X givet værdien af Y .
Man benytter ofte notationen:

$$
\varphi(y)=\mathbb{E}[\mathrm{X} \mid \mathrm{Y}=y], \quad(y \in E)
$$

    \end{definition}
\end{theorem-box}
\begin{theorem-box}
    \begin{lemma}
        Lad $(\Omega, \mathcal{F}, P)$ være et sandsynlighedsfelt, lad X være en stokastisk variabel i $\mathcal{L}^1(P)$, og lad $Y: \Omega \rightarrow E$ være en stokastisk funktion med værdier i et måleligt rum $(E, \mathcal{E})$.

Betragt videre en funktion $\varphi: E \rightarrow \mathbb{R}$ fra $\mathcal{M}(\mathcal{E})$.
Da er $\varphi$ en betinget middelværdi af X givet værdien af Y , hvis og kun hvis den opfylder følgende to betingelser:
\begin{enumerate}
    \item[\textnormal{(i)}] $\varphi \in \mathcal{L}^1\left(P_{\mathrm{Y}}\right)$.
    \item[\textnormal{(ii)}] $\int_B \varphi(y) P_{\mathrm{Y}}(\mathrm{d} y)=\int_{\{\mathrm{Y} \in B\}} \mathrm{X} \mathrm{d} P \quad$ for alle $B$ fra $\mathcal{E}$. 
\end{enumerate}
    \end{lemma}
\end{theorem-box}
\begin{theorem-box}
    \begin{lemma}
        $\operatorname{Lad}(\Omega, \mathcal{F}, P)$ være et sandsynlighedsfelt, og lad $\mathrm{X}, \tilde{\mathrm{x}}, \mathrm{x}_1, \mathrm{X}_2, \mathrm{X}_3, \ldots$ være stokastiske variable fra $\mathcal{L}^1(P)$. Lad videre $\mathrm{Y}: \Omega \rightarrow E$ være en stokastisk funktion med værdier $i(E, \mathcal{E})$.
\begin{enumerate}
    \item[\textnormal{(i)}] For ethvert $\alpha i \mathbb{R}$ gælder der, at

    $$
    \mathbb{E}[\alpha \mathrm{X}+\tilde{\mathrm{X}} \mid \mathrm{Y}=y]=\alpha \mathbb{E}[\mathrm{X} \mid \mathrm{Y}=y]+\mathbb{E}[\tilde{\mathrm{X}} \mid \mathrm{Y}=y] \quad \text { for } P_{\mathrm{Y}}-\text { n.a. } y .
    $$
    \item[\textnormal{(ii)}] Hvis $\tilde{\mathrm{X}} \leq \mathrm{X}$ P-n.o., gælder der også, at $\mathbb{E}[\tilde{\mathrm{X}} \mid \mathrm{Y}=y] \leq \mathbb{E}[\mathrm{X} \mid \mathrm{Y}=y]$ fo $P_Y-$ n.a. $y$.
    \item[\textnormal{(iii)}] Hvis $\mathrm{X}_n \uparrow \mathrm{X}$ P-n.o., gælder der også, at $\mathbb{E}\left[\mathrm{X}_n \mid \mathrm{Y}=y\right] \uparrow \mathbb{E}[\mathrm{X} \mid \mathrm{Y}=y]$ for $P_Y-n . a . y$.
    \item[\textnormal{(iv)}] Hvis $\mathrm{X}_n \downarrow \mathrm{X}$ P-n.o., gælder der også, at $\mathbb{E}\left[\mathrm{X}_n \mid \mathrm{Y}=y\right] \downarrow \mathbb{E}[\mathrm{X} \mid \mathrm{Y}=y]$ for $P_{\mathrm{Y}}-$ n.a. $y$.
\end{enumerate}
\end{lemma}
\end{theorem-box}
\begin{theorem-box}
    \begin{proposition}
        Lad $\mathrm{X}, \mathrm{X}^{\prime}$ være stokastiske variable i $\mathcal{L}^1(P)$, og lad $\mathrm{Y}, \mathrm{Y}^{\prime}: \Omega \rightarrow E$ være to stokastiske funktioner med værdier i et måleligt rum $(E, \mathcal{E})$.

Antag, at

$$
(\mathrm{X}, \mathrm{Y}) \sim\left(\mathrm{x}^{\prime}, \mathrm{Y}^{\prime}\right), \quad \text { altså at } \quad P_{(\mathrm{x}, \mathrm{Y})}=P_{\left(\mathrm{x}^{\prime}, \mathrm{Y}^{\prime}\right)} p \text { å }(\mathbb{R} \times E, \mathcal{B}(\mathbb{R}) \otimes \mathcal{E})
$$

Da vil enhver betinget middelværdi af X givet værdien af Y også være en betinget middelværdi af $\mathrm{X}^{\prime}$ givet værdien af $\mathrm{Y}^{\prime}$.

For $\varphi$ i $\mathcal{M}(\mathcal{E})$ gælder der således implikationen:

$$
\mathbb{E}[\mathrm{X} \mid \mathrm{Y}]=\varphi(\mathrm{Y}) \text { P-n.o. } \quad \Longrightarrow \mathbb{E}\left[\mathrm{X}^{\prime} \mid \mathrm{Y}^{\prime}\right]=\varphi\left(\mathrm{Y}^{\prime}\right) \text { P-n.o. }
$$
    \end{proposition}
\end{theorem-box}
\section{Betingede fordelinger}
\subsection{Definition, eksempler og entydighed}
\begin{theorem-box}
    \begin{definition}
        Lad $(\Omega, \mathcal{F}, P)$ være et sandsynlighedsfelt, og lad X og Y være stok. fkt. med værdier i mâlelige rum hhv. $\left(E_1, \mathcal{E}_1\right)$ og $\left(E_2, \mathcal{E}_2\right)$.

En betinget fordeling af X givet (værdien af) Y er en afbildning $\varphi: \mathcal{E}_1 \times E_2 \rightarrow[0,1]$, der opfylder følgende tre betingelser:
\begin{enumerate}
    \item For hvert fast $y$ i $E_2$ er afbildningen

$$
\varphi(\cdot, y): A \mapsto \varphi(A, y): \mathcal{E}_1 \rightarrow[0,1]
$$

et sandsynlighedsmâl på ( $E_1, \mathcal{E}_1$ ).
\item For enhver fast mængde $A$ i $\mathcal{E}_1$ er afbildningen

$$
\varphi(A, \cdot): y \mapsto \varphi(A, y): E_2 \rightarrow[0,1]
$$

$\mathcal{E}_2$-målelig.
\item For enhver mængde $A$ i $\mathcal{E}_1$ og enher mængde $B$ i $\mathcal{E}_2$ gælder der, at

$$
\int_B \varphi(A, y) P_Y(\mathrm{~d} y)=P(\mathrm{X} \in A, \mathrm{Y} \in B)
$$

\end{enumerate}
Man skriver ofte $P_{\mathbf{X}}(A \mid \mathrm{Y}=y)$ i stedet for $\varphi(A, y)$.
For hvert $y$ i $E_2$ kaldes sandsynlighedsmâlet

$$
\varphi(\cdot, y)=P_{\mathrm{x}}(\cdot \mid \mathrm{Y}=y)
$$

for den betingede fordeling af X givet $\mathrm{Y}=y$.
    \end{definition}
\end{theorem-box}
\begin{remark}\end{remark}
\begin{example}
    Lad X og Y være stokastiske variable på $(\Omega, \mathcal{F}, P)$, og antag, at Y er Poisson fordelt med parameter $\ell>0$.

Vi definerer så $\varphi: \mathcal{B}(\mathbb{R}) \times \mathbb{R} \rightarrow[0,1]$ ved

$$
\begin{aligned}
\varphi(A, y) & = \begin{cases}\frac{P(\mathbb{X} \in A, \mathrm{Y}=y)}{P(\mathrm{Y}=y)}, & \text { hvis } y \in \mathbb{N}_0 \\
\delta_0(A), & \text { hvis } y \in \mathbb{R} \backslash \mathbb{N}_0\end{cases} \\
& = \begin{cases}\frac{P(\mathbb{X} \in A, \mathrm{Y}=y)}{\mathrm{e}^{-\varepsilon_{\ell} y / y!},} & \text { hvis } y \in \mathbb{N}_0 \\
\delta_0(A), & \text { hvis } y \in \mathbb{R} \backslash \mathbb{N}_0\end{cases}
\end{aligned}
$$


Da er $\varphi$ en betinget fordeling af X givet værdien af Y .
Bemærk, at der for alle $A$ i $\mathcal{B}(\mathbb{R})$ og y i $\mathbb{R}$ gælder formlen:

$$
\varphi(A, y)=\delta_0(A) 1_{\mathrm{R} \backslash \mathrm{~N}_0}(y)+\sum_{n=0}^{\infty} \frac{P(\mathrm{X} \in A, \mathrm{Y}=n)}{\mathrm{e}^{-\ell \ell^n / n!}} 1_{\{n\}}(y)
$$

\end{example}
\begin{example}
    Lad $(\Omega, \mathcal{F}, P)$ være et sandsynlighedsfelt, og lad X og Y være stokastiske variable herpå.

Antag, at $P_{(\mathrm{x}, \mathrm{Y})}$ er absolut kontinuert med tæthed $f \in \mathcal{M}\left(\mathcal{B}\left(\mathbb{R}^2\right)\right)^{+}$med hensyn til $\lambda_2$.

Husk, at $P_{\mathrm{X}}$ og $P_{\mathrm{Y}}$ da automatisk er absolut kontinuerte mht. $\lambda$, med tætheder givet ved (jvf. 13.3.4 i [M\&l]):

$$
f_{\mathrm{X}}(x)=\int_{\mathbb{R}} f(x, t) \lambda(\mathrm{d} t), \quad \text { og } \quad f_{\mathrm{Y}}(y)=\int_{\mathbb{R}} f(s, y) \lambda(\mathrm{d} s), \quad(x, y \in \mathbb{R})
$$


Det følger da, at der ved formlen:

$$
\varphi(A, y)=\left\{\begin{array}{ll}
\frac{1}{f_Y(y)} \int_A f(x, y) \lambda(\mathrm{d} x), & \text { hvis } f_Y(y)>0, \\
\delta_0(A), & \text { hvis } f_Y(y)=0,
\end{array}(A \in \mathcal{B}(\mathbb{R}), y \in \mathbb{R}),\right.
$$

defineres en betinget fordeling af $X$ givet (værdien af) $Y$.
(a) For fast $y$ i $\mathbb{R}$ ses det umiddelbart, at $A \mapsto \varphi(A, y)$ er et sandsynlighedsmål på $(\mathbb{R}, \mathcal{B}(\mathbb{R}))$. Hvis $f_{\mathrm{Y}}(y)>0$ er det målet med tæthed $f_{\mathrm{Y}}(y)^{-1} f(\cdot, y)$ med hensyn til $\lambda$.
(b) For fast $A$ i $\mathcal{B}(\mathbb{R})$ følger det fra Tonelli's Sætning, at funktionen

$$
y \mapsto \int_A f(x, y) \lambda(\mathrm{d} x)=\int_{\mathbf{R}} f(x, y) 1_A(x) \lambda(\mathrm{d} x)
$$

er $\mathcal{B}(\mathbb{R})$-målelig.
Dermed sikrer Sætning 4.4.3 i [M\&I], at

$$
y \mapsto \varphi(A, y)= \begin{cases}\frac{1}{f_{\mathrm{Y}}(y)} \int_A f(x, y) \lambda(\mathrm{d} x), & \text { hvis } f_{\mathrm{Y}}(y)>0 \\ \delta_0(A), & \text { hvis } f_{\mathrm{Y}}(y)=0\end{cases}
$$

er $\mathcal{B}(\mathbb{R})$-målelig.
$$
\begin{aligned}
&\text { (c) For endnu en Borel-mængde } B \text { i } \mathcal{B}(\mathbb{R}) \text { finder vi endelig, at }\\
&\begin{aligned}
\int_B \varphi(A, y) P_{\mathrm{Y}}(\mathrm{~d} y) & =\int_{B \cap\left\{f_{\mathrm{Y}}>0\right\}} \varphi(A, y) f_{\mathrm{Y}}(y) \lambda(\mathrm{d} y) \\
& =\int_{B \cap\left\{f_{\mathrm{Y}}>0\right\}}\left(\frac{1}{f_{\mathrm{Y}}(y)} \int_A f(x, y) \lambda(\mathrm{d} x)\right) f_{\mathrm{Y}}(y) \lambda(\mathrm{d} y) \\
& =\int_{B \cap\left\{f_{\mathrm{r}}>0\right\}}\left(\int_A f(x, y) \lambda(\mathrm{d} x)\right) \lambda(\mathrm{d} y) \\
& =\int_B\left(\int_A f(x, y) \lambda(\mathrm{d} x)\right) \lambda(\mathrm{d} y) \stackrel{\text { Tonelli }}{=} \int_{A \times B} f \mathrm{~d} \lambda_2 \\
& =P((\mathrm{X}, \mathrm{Y}) \in A \times B) \\
& =P(\mathrm{X} \in A, \mathrm{Y} \in B)
\end{aligned}
\end{aligned}
$$
\end{example}
\begin{example}
    Antag, at (X,Y) er 2-dimensionalt normalfordelt med middelværdi-vektor $\left[\begin{array}{l}0 \\ 0\end{array}\right]$ og kovariansmatrix $\Sigma=\left(\begin{array}{ll}5 & 2 \\ 2 & 1\end{array}\right)$.
Mao. er ( $\mathrm{X}, \mathrm{Y}$ ) absolut kontinuert med $\lambda_2$-tæthed givet ved

$$
\begin{aligned}
f_{(\mathrm{X}, \mathrm{Y})}(x, y) & =\frac{1}{\sqrt{(2 \pi)^2 \operatorname{det}(\Sigma)}} \exp \left(-\frac{1}{2}\left\langle\Sigma^{-1}\binom{x}{y},\binom{x}{y}\right\rangle\right) \\
& =\frac{1}{2 \pi} \exp \left(-\frac{1}{2}\left\langle\left(\begin{array}{cc}
1 & -2 \\
-2 & 5
\end{array}\right)\binom{x}{y},\binom{x}{y}\right\rangle\right) \\
& =\frac{1}{2 \pi} \exp \left(-\frac{1}{2}\left(x^2-4 x y+5 y^2\right)\right)
\end{aligned}
$$


Bemærk også, at Y $\sim N(0,1)$, dvs. Y har $\lambda$-tæthed:

$$
f_{\mathrm{Y}}(y)=\frac{1}{\sqrt{2 \pi}} \exp \left(-\frac{1}{2} y^2\right)
$$
Det følger derfor fra Eksempel 8.1.4, at $P_{\mathrm{X}}(\cdot \mid \mathrm{Y}=y)$ er målet med $\lambda$-tæthed:

$$
\begin{aligned}
x \mapsto \frac{f_{(\mathrm{X}, \mathrm{Y})}(x, y)}{f_{\mathrm{Y}}(y)} & =\frac{\frac{1}{2 \pi} \exp \left(-\frac{1}{2}\left(x^2-4 x y+5 y^2\right)\right)}{\frac{1}{\sqrt{2 \pi}} \exp \left(-\frac{1}{2} y^2\right)} \\
& =\frac{1}{\sqrt{2 \pi}} \exp \left(-\frac{1}{2}\left(x^2-4 x y+4 y^2\right)\right) \\
& =\frac{1}{\sqrt{2 \pi}} \exp \left(-\frac{1}{2}(x-2 y)^2\right)
\end{aligned}
$$


Det følger således at, $P_{\mathrm{X}}(\cdot \mid \mathrm{Y}=y)=N(2 y, 1)$ for alle $y$ i $\mathbb{R}$.
\end{example}
\begin{example}
Lad $(\Omega, \mathcal{F}, P)$ være et sandsynlighedsfelt, og lad X og Y være stok. fkt. med værdier i målelige rum hhv. $\left(E_1, \mathcal{E}_1\right)$ og $\left(E_2, \mathcal{E}_2\right)$.

Da er X og Y uafhængige, hvis og kun hvis der findes et sandsynlighedsmål $\mu$ på $\left(E_1, \mathcal{E}_1\right)$, således at der ved

$$
\varphi(A, y)=\mu(A), \quad\left(A \in \mathcal{E}_1, y \in E_2\right)
$$

defineres en betinget fordeling af $X$ givet (værdien af) $Y$.
I bekræftende fald gælder der, at $\mu=P_{\mathrm{X}}$.
Antag nemlig, at X og Y er uafhængige, og definér $\varphi: \mathcal{E}_1 \times E_2 \rightarrow[0,1]$ ved

$$
\varphi(A, y)=P_{\mathrm{X}}(A), \quad\left(A \in \mathcal{E}_1, y \in E_2\right)
$$
(i) Det er klart, at $A \mapsto \varphi(A, y)$ er et sandsynlighedsmål for ethvert $y$ i $E_2$.
\\(ii) For enher fast mængde $A$ i $\mathcal{E}_1$ er det også klart, at $y \mapsto \varphi(A, y)$ er $\mathcal{E}_2$-målelig.
\\(iii) For en vilkårlig mængde $B$ fra $\mathcal{E}_2$ finder vi endelig, at

$$
\begin{aligned}
\int_B \varphi(A, y) P_{\mathrm{Y}}(\mathrm{~d} y) & =\int_B P_{\mathrm{X}}(A) P_{\mathrm{Y}}(\mathrm{~d} y)=P_{\mathrm{X}}(A) P_{\mathrm{Y}}(B) \\
& =P(\mathrm{X} \in A) P(\mathrm{Y} \in B)=P(\mathrm{X} \in A, \mathrm{Y} \in B)
\end{aligned}
$$
Antag omvendt, at der findes et ssh-mål $\mu$ på $\left(E_1, \mathcal{E}_1\right)$, således at

$$
\varphi(A, y)=\mu(A), \quad\left(A \in \mathcal{E}_2, y \in E_2\right)
$$

definerer en betinget fordeling af $X$ givet (værdien af) $Y$.
For vilkårlige $A$ i $\mathcal{E}_1$ og $B$ i $\mathcal{E}_2$ følger det da, at

$$
\begin{aligned}
P(\mathrm{X} \in A, \mathrm{Y} \in B) & =\int_B \varphi(A, y) P_{\mathrm{Y}}(\mathrm{~d} y)=\int_B \mu(A) P_{\mathrm{Y}}(\mathrm{~d} y) \\
& =\mu(A) P_{\mathrm{Y}}(B)=\mu(A) P(\mathrm{Y} \in B)
\end{aligned}
$$


Sættes specielt $B=E_2$, fremgår det, at

$$
P(\mathrm{X} \in A)=P\left(\mathrm{X} \in A, \mathrm{Y} \in E_2\right)=\mu(A) P\left(\mathrm{Y} \in E_2\right)=\mu(A)
$$

således at $\mu=P_{\mathbf{x}}$.
Dermed viser udregningen ovenfor videre, at

$$
P(\mathrm{x} \in A, \mathrm{Y} \in B)=P_{\mathrm{X}}(A) P(\mathrm{Y} \in B)=P(\mathrm{X} \in A) P(\mathrm{Y} \in B) .
$$

\end{example}
\begin{theorem-box}
    \begin{proposition}
        Lad $(\Omega, \mathcal{F}, P)$ være et sandsynlighedsfelt, og lad X og Y være stokastiske funktioner med værdier i målelige rum hhv. $\left(E_1, \mathcal{E}_1\right)$ og $\left(E_2, \mathcal{E}_2\right)$.

Antag, at $\varphi, \tilde{\varphi}: \mathcal{E}_1 \times E_2 \rightarrow[0,1]$ er to betingede fordelinger af X givet (værdien af) Y.

Antag endvidere, at $\mathcal{E}_1$ er tælleligt frembragt.
Da findes en mængde $N$ fra $\mathcal{E}_2$, således at
\begin{enumerate}
    \item[\textnormal{(i)}] $P(Y \in N)=0$.
    \item[\textnormal{(ii)}] $\varphi(A, y)=\tilde{\varphi}(A, y)$ for alle $A$ i $\mathcal{E}_1$ og $y$ i $N^c$.
\end{enumerate}
    \end{proposition}
\end{theorem-box}
\subsection{Transformation og integration med funktioner af én variabel}
\begin{theorem-box}
    \begin{proposition}
        Lad $(\Omega, \mathcal{F}, P)$ være et sandsynlighedsfelt, og lad X og Y være stokastiske funktioner herpå med værdier i målelige rum hhv. $\left(E_1, \mathcal{E}_1\right)$ og $\left(E_2, \mathcal{E}_2\right)$.

Lad $\left(E_3, \mathcal{E}_3\right)$ være endnu et måleligt rum, og betragt en $\mathcal{E}_1-\mathcal{E}_3$-målelig afbildning $\psi: E_1 \rightarrow E_3$.

Antag, at der findes en betinget fordeling

$$
P_{\mathrm{X}}(\cdot \mid \mathrm{Y}=\cdot): \mathcal{E}_1 \times E_2 \rightarrow[0,1]
$$

af X givet værdien af Y . Da er afbildningen

$$
\varphi(C, y)=P_{\mathrm{X}}\left(\psi^{-1}(C) \mid \mathrm{Y}=y\right), \quad\left(C \in \mathcal{E}_3, y \in E_2\right),
$$

en betinget fordeling af $\psi(\mathrm{X})$ givet værdien af Y .
    \end{proposition}
\end{theorem-box}
\begin{theorem-box}
    \begin{proposition}
        Lad $(\Omega, \mathcal{F}, P)$ være et ssh.-felt, og lad X og Y være stok. fkt. med værdier i målelige rum $h \boldsymbol{h} \boldsymbol{v} .\left(E_1, \mathcal{E}_1\right)$ og $\left(E_2, \mathcal{E}_2\right)$.
Antag, at der findes en betinget fordeling $P_{\mathrm{X}}(\cdot \mid \mathrm{Y}=\cdot$ ) af X givet (værdien af) Y.
Antag videre, at $\psi: E_1 \rightarrow \mathbb{R}$ er en $\mathcal{E}_1-\mathcal{B}(\mathbb{R})$-målelig funktion, således at $\mathbb{E}[\|\psi(\mathrm{X})\|<\infty$.
\begin{enumerate}
    \item[\textnormal{(i)}] Mængden $N_\psi:=\left\{y \in E_2 \mid \psi \notin \mathcal{L}^1\left(P_{\mathrm{X}}(\cdot \mid \mathrm{Y}=y)\right)\right\}$ er element $i \mathcal{E}_2$, og $P\left(Y \in N_\psi\right)=0$.
    \item[\textnormal{(ii)}]Funktionen

$$
w(y)= \begin{cases}\int_{E_1} \psi(x) P_{\mathrm{X}}(\mathrm{~d} x \mid \mathrm{Y}=y), & \text { hvis } y \in N_\psi^c \\ 0, & \text { hvis } y \in N_\psi\end{cases}
$$
er en version af $y \mapsto \mathbb{E}[\psi(\mathrm{X}) \mid \mathrm{Y}=y]$.
\end{enumerate}
    \end{proposition}
\end{theorem-box}
\begin{example}
    Antag, at ( $\mathrm{X}, \mathrm{Y}$ ) er 2-dimensionalt normalfordelt med middelværdi-vektor $\underline{0}$ og kovariansmatrix $\Sigma=\left(\begin{array}{ll}5 & 2 \\ 2 & 1\end{array}\right)$.

Vi har tidligere set, at $P_{\mathrm{X}}(\cdot \mid \mathrm{Y}=y)=N(2 y, 1)$ for alle $y$ i $\mathbb{R}$.
Vha. Sætning 8.2.2 følger det så specielt for $P_Y$-n.a. $y$, at

$$
\begin{gathered}
\mathbb{E}[\mathrm{X} \mid \mathrm{Y}=y]=\int_{\mathbb{R}} x P_{\mathrm{X}}(\mathrm{~d} x \mid \mathrm{Y}=y)=\int_{\mathbb{R}} x N(2 y, 1)(\mathrm{d} x)=2 y \\
\mathbb{E}\left[\mathrm{X}^2 \mid \mathrm{Y}=y\right]=\int_{\mathbb{R}} x^2 P_{\mathrm{X}}(\mathrm{~d} x \mid \mathrm{Y}=y)=\int_{\mathbb{R}} x^2 N(2 y, 1)(\mathrm{d} x)=1+4 y^2
\end{gathered}
$$

således at $\mathbb{E}[\mathrm{X} \mid \mathrm{Y}]=2 \mathrm{Y}, \mathbb{E}\left[\mathrm{X}^2 \mid \mathrm{Y}\right]=1+4 \mathrm{Y}^2$, og $\mathbb{V}[\mathrm{X} \mid \mathrm{Y}]=1$.
\end{example}
\subsection{Transformation og integration med funktioner af to variable}
\begin{theorem-box}
    \begin{lemma}
           Lad $(\Omega, \mathcal{F}, P)$ være et sandsynlighedsfelt, og lad X og Y være stok. fkt. med værdier i målelige rum hhv. $\left(E_1, \mathcal{E}_1\right)$ og $\left(E_2, \mathcal{E}_2\right)$.

Antag, at der findes en betinget fordeling $P_{\mathrm{X}}(\cdot \mid \mathrm{Y}=\cdot$ ) af X givet (værdien af) Y.

For enhver mængde H i $\mathcal{E}_1 \otimes \mathcal{E}_2$ gælder der da, at funktionen

$$
w_H(y):=\int_{E_1} 1_H(x, y) P_{\mathrm{X}}(\mathrm{~d} x \mid \mathrm{Y}=y), \quad\left(y \in E_2\right)
$$

er en version af $y \mapsto \mathbb{E}\left[1_H(\mathrm{X}, \mathrm{Y}) \mid \mathrm{Y}=y\right]$.
Der gælder altså, at

$$
\mathbb{E}\left[1_H(\mathrm{X}, \mathrm{Y}) \mid \mathrm{Y}=y\right]=\int_{E_1} 1_H(x, y) P_{\mathrm{X}}(\mathrm{~d} x \mid \mathrm{Y}=y), \quad\left(y \in E_2\right)
$$
     
    \end{lemma}
\end{theorem-box}
\begin{theorem-box}
    \begin{corollary}
        Lad $(\Omega, \mathcal{F}, P)$ være et sandsynlighedsfelt, og lad X og Y være stok. fkt. med værdier i målelige rum hhv. $\left(E_1, \mathcal{E}_1\right)$ og $\left(E_2, \mathcal{E}_2\right)$.

Antag, at der findes en betinget fordeling $P_{\mathrm{X}}(\cdot \mid \mathrm{Y}=\cdot)$ af X givet (værdien af) Y.

Antag videre, at $\left(E_3, \mathcal{E}_3\right)$ er et måleligt rum, og at $\psi: E_1 \times E_2 \rightarrow E_3$ er en $\mathcal{E}_1 \otimes \mathcal{E}_2-\mathcal{E}_3$-målelig funktion.

Da er afbildningen $\varphi: \mathcal{E}_3 \times E_2 \rightarrow[0,1]$ givet ved

$$
\varphi(C, y)=P_{\mathrm{X}}\left(\psi(\cdot, y)^{-1}(C) \mid \mathrm{Y}=y\right), \quad\left(C \in \mathcal{E}_3, y \in E_2\right)
$$

en betinget fordeling af $\psi(\mathrm{X}, \mathrm{Y})$ givet (værdien af) Y.
Med andre ord gælder der altså, at

$$
P_{\psi(\mathrm{X}, \mathrm{Y})}(\cdot \mid \mathrm{Y}=y)=P_{\mathrm{X}}(\cdot \mid \mathrm{Y}=y) \circ \psi(\cdot, y)^{-1} \quad \text { for } P_{\mathrm{Y}}-\text { n.a. } \text { y } \text { i } E_2 \text {. }
$$

    \end{corollary}
\end{theorem-box}
\begin{theorem-box}
    \begin{proposition}
        Lad $(\Omega, \mathcal{F}, P)$ være et sandsynlighedsfelt, og lad X og Y være stok. fkt. med værdier i målelige rum hhv. $\left(E_1, \mathcal{E}_1\right)$ og $\left(E_2, \mathcal{E}_2\right)$.

Antag, at der findes en betinget fordeling $P_{\mathrm{X}}(\cdot \mid \mathrm{Y}=\cdot$ ) af X givet (værdien af) Y.

Antag videre, at $\psi: E_1 \times E_2 \rightarrow \mathbb{R}$ er en $\mathcal{E}_1 \otimes \mathcal{E}_2-\mathcal{B}(\mathbb{R})$-målelig funktion, således at $\mathbb{E}[\|\psi(\mathrm{X}, \mathrm{Y})\|<\infty$.
\begin{enumerate}
    \item[\textnormal{(i)}] Mængden $N:=\left\{y \in E_2 \mid \psi(\cdot, y) \notin \mathcal{L}^1\left(P_{\mathrm{X}}(\cdot \mid \mathrm{Y}=y)\right)\right\}$ er element $i$ $\mathcal{E}_2, \operatorname{og} P(\mathrm{Y} \in N)=0$.
    \item[\textnormal{(ii)}]  Funktionen

$$
W(y)= \begin{cases}\int_{E_1} \psi(x, y) P_{\mathrm{X}}(\mathrm{~d} x \mid \mathrm{Y}=y), & \text { hvis } y \in N^c \\ 0, & \text { hvis } y \in N\end{cases}
$$

er en version af $y \mapsto \mathbb{E}[\psi(\mathrm{X}, \mathrm{Y}) \mid \mathrm{Y}=y]$.
\end{enumerate}
    \end{proposition}
\end{theorem-box}
\begin{example}
    Antag, at ( $\mathrm{X}, \mathrm{Y}$ ) er 2-dimensionalt normalfordelt med middelværdi-vektor $\underline{0}$ og kovariansmatrix $\Sigma=\left(\begin{array}{ll}5 & 2 \\ 2 & 1\end{array}\right)$.
Vi ønsker at bestemme $\mathbb{E}[\cos (\mathrm{XY}) \mid \mathrm{Y}]$.
Vi har tidligere set, at $P_{\mathrm{X}}(\cdot \mid \mathrm{Y}=y)=N(2 y, 1)$ for alle $y$ i $\mathbb{R}$.
Vi finder så ved brug af Sætning 8.3.4, at

$$
\begin{aligned}
\mathbb{E} & {[\cos (\mathrm{XY}) \mid \mathrm{Y}=y]=\int_{\mathbb{R}} \cos (x y) P_{\mathrm{X}}(\mathrm{~d} x \mid \mathrm{Y}=y) } \\
& =\int_{\mathbb{R}} \cos (x y) N(2 y, 1)(\mathrm{d} x)=\operatorname{Re}\left(\int_{\mathbb{R}} \mathrm{e}^{\mathrm{i} x y} N(2 y, 1)(\mathrm{d} x)\right) \\
& =\operatorname{Re}(\widehat{N(2 y, 1)}(y)) \stackrel{1.1 .3}{\stackrel{\downarrow}{=}} \operatorname{Re}\left(\mathrm{e}^{\mathrm{i} 2 y(y)} \mathrm{e}^{-y^2 / 2}\right)=\cos \left(2 y^2\right) \mathrm{e}^{-y^2 / 2} .
\end{aligned}
$$


Vi kan dermed slutte, at

$$
\mathbb{E}[\cos (\mathrm{XY}) \mid \mathrm{Y}]=\cos \left(2 \mathrm{Y}^2\right) \mathrm{e}^{-\mathrm{Y}^2 / 2}
$$

\end{example}
\subsection{Eksistens af betingede fordelinger}
\begin{theorem-box}
    \begin{proposition}
        Lad X være en reel stokastisk variabel på $(\Omega, \mathcal{F}, P)$, og lad Y være en stokastisk funktion på $(\Omega, \mathcal{F}, P)$ med værdier $i$ et måleligt rum $(E, \mathcal{E})$.

Da findes en betinget fordeling $P_{\mathrm{X}}(\cdot \mid \mathrm{Y}=\cdot)$ af X givet værdien af Y .
    \end{proposition}
\end{theorem-box}
\begin{theorem-box}
    \begin{lemma}
        Lad $G: \mathbb{Q} \rightarrow[0,1]$ være en voksende funktion, således at

$$
\lim _{n \rightarrow \infty} G(-n)=0, \quad \text { og } \quad \lim _{n \rightarrow \infty} G(n)=1
$$


Definér funktionen $F: \mathbb{R} \rightarrow[0,1]$ ved

$$
F(x)=\inf \{G(q) \mid q \in(x, \infty) \cap \mathbb{Q}\}, \quad(x \in \mathbb{R})
$$


Da er F voksende, højrekontinuert, og der gælder at

$$
\lim _{x \rightarrow-\infty} F(x)=0, \quad \text { og } \quad \lim _{x \rightarrow \infty} F(x)=1
$$


Med andre ord er $F$ fordelingsfunktionen for et sandsynlighedsmål på $\mathbb{R}$ (Lebesgue-Stieltjes målet hørende til F - jvf. 3.5.7 i [M\&I]).
    \end{lemma}
\end{theorem-box}
\begin{theorem-box}
    \begin{definition}
        Et måleligt rum $(E, \mathcal{E})$ kaldes for et Borel-rum, hvis der findes en mængde $M$ i $\mathcal{B}(\mathbb{R})$, og en afbildning $\psi: E \rightarrow M$, således at
        \begin{enumerate}
            \item $\psi$ er bijektiv.
            \item $\psi$ er $\mathcal{E}-\mathcal{B}(\mathbb{R})_M$-målelig.
            \item $\psi^{\langle-1\rangle}: M \rightarrow E$ er $\mathcal{B}(\mathbb{R})_{M^{-}} \mathcal{E}$-målelig.
        \end{enumerate}
Husk, at

$$
\mathcal{B}(\mathbb{R})_M=\{M \cap A \mid A \in \mathcal{B}(\mathbb{R})\}=\{B \in \mathcal{B}(\mathbb{R}) \mid B \subseteq M\}
$$

    \end{definition}
\end{theorem-box}
\begin{theorem-box}
    \begin{proposition}
        Lad $(\Omega, \mathcal{F}, P)$ være et sandsynlighedsfelt, og lad X og Y være stokastiske funktioner herpå med værdier i målelige rum hhv. $\left(E_1, \mathcal{E}_1\right)$ og $\left(E_2, \mathcal{E}_2\right)$.

Antag, at $\left(E_1, \mathcal{E}_1\right)$ er et Borel-rum.
Da findes en betinget fordeling

$$
P_{\mathrm{X}}(\cdot \mid \mathrm{Y}=\cdot): \mathcal{E}_1 \times E_2 \rightarrow[0,1]
$$

af X givet værdien af Y.
    \end{proposition}
\end{theorem-box}

\section{Martingaler}
\subsection{Definition, eksempler og grundlæggende egenskaber}
\begin{theorem-box}
    \begin{definition}
        Lad $(\Omega, \mathcal{F}, P)$ være et sandsynlighedsfelt.
        \begin{enumerate}
            \item Et filter på $(\Omega, \mathcal{F})$ er en voksende følge $\left(\mathcal{F}_n\right)_{n \in \mathbb{N}_0}$ af del- $\sigma$-algebraer af $\mathcal{F}$.
            \item Hvis $\left(\mathcal{F}_n\right)_{n \in \mathbb{N}_0}$ er et filter på $(\Omega, \mathcal{F})$, siges $\left(\Omega, \mathcal{F}, \mathcal{F}_n, P\right)$ at være et filtreret sandsynlighedsfelt.
            \item En følge $\left(\mathrm{X}_n\right)_{n \in \mathbb{N}_0}$ af stokastiske variable på $(\Omega, \mathcal{F}, P)$ kaldes tilpasset med hensyn til et filter $\left(\mathcal{F}_n\right)_{n \in \mathbb{N}_0}$, hvis $X_n$ er $\mathcal{F}_n$-målelig for alle $n$ i $\mathbb{N}_0$.
            \item En følge $\left(\mathrm{X}_n\right)_{n \in \mathbb{N}_0}$ af stokastiske variable på $(\Omega, \mathcal{F}, P)$ kaldes forudsigelig (eller predictabel) med hensyn til et filter $\left(\mathcal{F}_n\right)_{n \in \mathbb{N}_0}$, hvis $\mathrm{x}_n$ er $\mathcal{F}_{(n-1) \vee 0^{-m a l e l i g}}$ for alle $n$ i $\mathbb{N}_0$.
        \end{enumerate}
    \end{definition}
\end{theorem-box}
\begin{theorem-box}
    \begin{definition}[Martingaler, sub-martingaler og super-martingaler]
        Lad $\left(\Omega, \mathcal{F}, \mathcal{F}_n, P\right)$ være et filtreret sandsynlighedsfelt, og lad $\left(\mathrm{X}_n\right)_{n \in \mathbb{N}_0}$ være en følge af stokastiske variable herpå. Antag, at
\begin{enumerate}
    \item[(a)] $\mathrm{X}_n \in \mathcal{L}^1(P)$ for alle $n$.
    \item[(b)] $\left(\mathrm{X}_n\right)_{n \in \mathbb{N}_0}$ er tilpasset med hensyn til $\left(\mathcal{F}_n\right)$.
    \\Vi siger da, at
        \item[(c1)] $\left(\mathrm{X}_n, \mathcal{F}_n\right)_{n \in \mathbb{N}_0}$ er en martingal, hvis $\mathbb{E}\left[\mathrm{X}_{n+1} \mid \mathcal{F}_n\right]=\mathrm{X}_n P$-n.o. for alle $n$.
        \item[c2] $\left(\mathrm{X}_n, \mathcal{F}_n\right)_{n \in \mathbb{N}_0}$ er en sub-martingal, hvis $\mathbb{E}\left[\mathrm{X}_{n+1} \mid \mathcal{F}_n\right] \geq \mathrm{X}_n P$-n.o. for alle $n$.
        \item[c3]  $\left(\mathrm{X}_n, \mathcal{F}_n\right)_{n \in \mathbb{N}_0}$ er en super-martingal, hvis $\mathbb{E}\left[\mathrm{X}_{n+1} \mid \mathcal{F}_n\right] \leq \mathrm{X}_n$ P-n.o. for alle $n$.
\end{enumerate}
    \end{definition}
\end{theorem-box}
\begin{remark}
    Lad $\left(\Omega, \mathcal{F}, \mathcal{F}_n, P\right)$ være et filtreret sandsynlighedsfelt, og lad $\left(\mathrm{X}_n\right)$ være en følge af stokastiske variable.
    \begin{enumerate}
        \item Hvis $\left(\mathrm{X}_n, \mathcal{F}_n\right)$ er en martingal, gælder der, at
        
        $$
        \mathbb{E}\left[\mathrm{X}_m \mid \mathcal{F}_n\right] \stackrel{\substack{.2 .4}}{\stackrel{1}{=}} \mathbb{E}\left[\mathbb{E}\left[\mathrm{X}_m \mid \mathcal{F}_{m-1}\right] \mid \mathcal{F}_n\right]=\mathbb{E}\left[\mathrm{X}_{m-1} \mid \mathcal{F}_n\right]=\cdots=\mathbb{E}\left[\mathrm{X}_n \mid \mathcal{F}_n\right]=\mathrm{X}_n
        $$
        
        for alle $m, n$ i $\mathbb{N}$, så $n<m$. Specielt ses, at $\mathbb{E}\left[\mathrm{X}_m\right]=\mathbb{E}\left[\mathrm{X}_n\right]$.
        \item Hvis $\left(\mathrm{X}_n, \mathcal{F}_n\right)$ er en sub-martingal, gælder der, at
        
        $$
        \mathbb{E}\left[\mathrm{X}_m \mid \mathcal{F}_n\right]=\mathbb{E}\left[\mathbb{E}\left[\mathrm{X}_m \mid \mathcal{F}_{m-1}\right] \mid \mathcal{F}_n\right] \geq \mathbb{E}\left[\mathrm{X}_{m-1} \mid \mathcal{F}_n\right] \geq \cdots \geq \mathbb{E}\left[\mathrm{X}_n \mid \mathcal{F}_n\right]=\mathrm{X}_n
        $$
        
        for alle $m, n$ i $\mathbb{N}$, så $n<m$. Specielt ses, at $\mathbb{E}\left[\mathrm{X}_m\right] \geq \mathbb{E}\left[\mathrm{X}_n\right]$.
        \item Hvis $\left(\mathrm{X}_n, \mathcal{F}_n\right)$ er en super-martingal, gælder der, at $\mathbb{E}\left[\mathrm{X}_m \mid \mathcal{F}_n\right] \leq \mathrm{X}_n \quad$ og $\quad \mathbb{E}\left[\mathrm{X}_m\right] \leq \mathbb{E}\left[\mathrm{X}_n\right] \quad$ for alle $m, n$ i $\mathbb{N}$, så $n<m$.
    \end{enumerate}
    Lad $\left(\Omega, \mathcal{F}, \mathcal{F}_n, P\right)$ være et filtreret sandsynlighedsfelt, og lad $\left(\mathrm{X}_n\right)_{n \in \mathbb{N}_0}$ være en følge af stokastiske variable herpå.

Der gælder da, at

$$
\left(\mathrm{X}_n, \mathcal{F}_n\right) \text { er en sub-MG } \Longleftrightarrow\left(-\mathrm{X}_n, \mathcal{F}_n\right) \text { er en super-MG, }
$$

og
$\left(\mathrm{X}_n, \mathcal{F}_n\right)$ er en $\mathrm{MG} \Longleftrightarrow\left(\mathrm{X}_n, \mathcal{F}_n\right)$ er både en sub-MG og en super-MG.
\end{remark}
\begin{example}
    En spiller deltager i en (uendelig) følge af uafhængige spil på et casino.
For hvert $n$ i $\mathbb{N}_0$ sætter vi

$$
\mathrm{X}_n=\text { gevinsten (eller tabet) ved det } n \text { 'te spil. }
$$


Vi antager, at hvert spil er fair, dvs. at $\mathbb{E}\left[\mathrm{X}_n\right]=0$ for alle $n$ i $\mathbb{N}_0$ (specielt antages det, at $\left.\mathrm{X}_n \in \mathcal{L}^1(P)\right)$.

Vi sætter endelig

$$
\mathrm{S}_n=\text { spillerens samlede gevinst (eller tab) efter det } n \text { 'te spil }
$$


$$
=\sum_{j=0}^n \mathrm{X}_j
$$


Da er $\left(\mathrm{S}_n\right)_{n \in \mathbb{N}_0}$ en martingal med hensyn til følgen:

$$
\mathcal{F}_n=\sigma\left(\mathrm{X}_0, \ldots, \mathrm{X}_n\right), \quad\left(n \in \mathbb{N}_0\right)
$$

\end{example}
\begin{example}[Lévy martingaler]
    Lad $\left(\Omega, \mathcal{F}, \mathcal{F}_n, P\right)$ være et filtreret sandsynlighedsfelt, og lad X være en stokastisk variabel i $\mathcal{L}^1(P)$.

Sæt $\mathrm{X}_n=\mathbb{E}\left[\mathrm{X} \mid \mathcal{F}_n\right]$ for alle $n$ i $\mathbb{N}_0$.
Da er $\left(\mathrm{X}_n, \mathcal{F}_n\right)_{n \in \mathbb{N}_0}$ en martingal.
Det er nemlig klart, at $\mathrm{X}_n=\mathbb{E}\left[\mathrm{X} \mid \mathcal{F}_n\right]$ er integrabel og $\mathcal{F}_n$-målelig for ethvert $n$ i $\mathbb{N}_0$.

For hvert $n$ i $\mathbb{N}_0$ finder vi videre, at

$$
\mathbb{E}\left[\mathrm{X}_{n+1} \mid \mathcal{F}_n\right]=\mathbb{E}\left[\mathbb{E}\left[\mathrm{X} \mid \mathcal{F}_{n+1}\right] \mid \mathcal{F}_n\right] \stackrel{7.2 .4}{\stackrel{\downarrow}{\rightleftharpoons}} \mathbb{E}\left[\mathrm{X} \mid \mathcal{F}_n\right] \stackrel{?}{=} \mathrm{X}_n \quad \text { P-n.o. }
$$


Ifølge 7.7.1 er $\left\{\mathrm{X}_n \mid n \in \mathbb{N}_0\right\}$ uniformt integrabel.
\end{example}
\begin{example}
Se øvelse 9.1
\end{example}
\begin{theorem-box}
    \begin{lemma}
        Lad $\left(\Omega, \mathcal{F}, \mathcal{F}_n, P\right)$ være et filtreret sandsynlighedsfelt, og lad $\left(\mathrm{X}_n\right)_{n \in \mathbb{N}_0}$ og $\left(\mathrm{Y}_n\right)_{n \in \mathbb{N}_0}$ være følger af stokastiske variable herpå.
\begin{enumerate}
    \item[\textnormal{(i)}]  Hvis $\left(\mathrm{X}_n, \mathcal{F}_n\right)$ og $\left(\mathrm{Y}_n, \mathcal{F}_n\right)$ er martingaler, og $a, b \in \mathbb{R}$, da er $\left(a \mathrm{X}_n+b \mathrm{Y}_n, \mathcal{F}_n\right)$ igen en martingal.
    \item[\textnormal{(ii)}] Hvis $\left(\mathrm{X}_n, \mathcal{F}_n\right)$ og $\left(\mathrm{Y}_n, \mathcal{F}_n\right)$ er sub-martingaler, og $a, b \in[0, \infty)$, da er $\left(a \mathrm{X}_n+b \mathrm{Y}_n, \mathcal{F}_n\right)$ igen en sub-martingal.
    \item[\textnormal{(iii)}] Hvis $\left(\mathrm{X}_n, \mathcal{F}_n\right)$ og $\left(\mathrm{Y}_n, \mathcal{F}_n\right)$ er super-martingaler, og $a, b \in[0, \infty)$, da er $\left(a \mathrm{X}_n+b \mathrm{Y}_n, \mathcal{F}_n\right)$ igen en super-martingal.
\end{enumerate}
    \end{lemma}
\end{theorem-box}
\begin{theorem-box}
    \begin{proposition}[Konvekse transformationer af martingaler og sub-martingaler]
        Lad $\left(\Omega, \mathcal{F}, \mathcal{F}_n, P\right)$ være et filtreret sandsynlighedsfelt, og lad $\left(\mathrm{X}_n\right)_{n \in \mathbb{N}_0}$ være en følge af stokastiske variable herpå.

Lad videre $\varphi: \mathbb{R} \rightarrow \mathbb{R}$ være en konveks-funktion, således at $\varphi\left(\mathrm{X}_n\right) \in \mathcal{L}^1(P)$ for alle $n$.

Da gælder følgende udsagn:
\begin{enumerate}
    \item[\textnormal{(i)}] Hvis $\left(\mathrm{X}_n, \mathcal{F}_n\right)$ er en martingal, da er $\left(\varphi\left(\mathrm{X}_n\right), \mathcal{F}_n\right)$ en submartingal.
    \item[\textnormal{(ii)}] Hvis $\left(\mathrm{X}_n, \mathcal{F}_n\right)$ er en sub-martingal, og $\varphi$ yderligere er voksende, da er $\left(\varphi\left(\mathrm{X}_n\right), \mathcal{F}_n\right)$ igen en submartingal.
\end{enumerate}
    \end{proposition}
\end{theorem-box}
\begin{theorem-box}
    \begin{lemma}
       Lad $\left(\Omega, \mathcal{F}, \mathcal{F}_n, P\right)$ være et filtreret ssh-felt, og lad $\left(\mathrm{X}_n\right)_{n \in \mathbb{N}_0}$ være en følge af stokastiske variable herpå.
\begin{enumerate}
    \item[\textnormal{(i)}] Hvis $\left(\mathrm{X}_n, \mathcal{F}_n\right)$ er en sub-martingal, så gæ/der der, at

    $$
    \sup _{n \in \mathbb{N}_0} \mathbb{E}\left[\left|\mathrm{X}_n\right|\right]<\infty \Longleftrightarrow \sup _{n \in \mathbb{N}_0} \mathbb{E}\left[\mathrm{X}_n^{+}\right]<\infty .
    $$
    \item[\textnormal{(ii)}] Hvis $\left(\mathrm{X}_n, \mathcal{F}_n\right)$ er en sub-MG, og der findes en stok. var. $Z$ i $\mathcal{L}^1(P)$, så $\mathrm{X}_n \leq Z$ P-n.o. for alle $n$, da er $\sup _{n \in \mathbb{N}_0} \mathbb{E}\left[\left|\mathrm{X}_n\right|\right]<\infty$.

    \item[\textnormal{(iii)}] Hvis $\left(\mathrm{X}_n, \mathcal{F}_n\right)$ er en super-martingal, så gæ/der der, at

    $$
    \sup _{n \in \mathbb{N}_0} \mathbb{E}\left[\left|\mathrm{X}_n\right|\right]<\infty \Longleftrightarrow \sup _{n \in \mathbb{N}_0} \mathbb{E}\left[\mathrm{X}_n^{-}\right]<\infty
    $$
    \item[\textnormal{(iv)}] Hvis $\left(\mathrm{X}_n, \mathcal{F}_n\right)$ er en super-MG, og der findes en stok. var. $Z$ i $\mathcal{L}^1(P)$, så $\mathrm{X}_n \geq Z$ P-n.o. for alle $n$, da er $\sup _{n \in \mathbb{N}_0} \mathbb{E}\left[\left|\mathrm{X}_n\right|\right]<\infty$.
\end{enumerate}
    \end{lemma}
\end{theorem-box}
\subsection{Konstruktioner med martingaler}
\begin{theorem-box}
    \begin{proposition}[Doob-dekompositionen]
        $\operatorname{Lad}\left(\Omega, \mathcal{F}, \mathcal{F}_n, P\right)$ være et filtreret sandsynlighedsfelt, og lad $\left(\mathrm{X}_n\right)_{n \in \mathbb{N}_0}$ være en tilpasset følge af stokastiske variable fra $\mathcal{L}^1(P)$.
Da kan $\left(\mathrm{X}_n\right)_{n \in \mathbb{N}_0}$ dekomponeres på formen:

$$
\mathrm{X}_n=\mathrm{M}_n+\mathrm{A}_n, \quad\left(n \in \mathbb{N}_0\right)
$$

hvor
\begin{enumerate}
    \item  $\left(\mathrm{M}_n, \mathcal{F}_n\right)_{n \in \mathbb{N}_0}$ er en martingal.
    \item $\mathrm{A}_0 \equiv 0$, og $\mathrm{A}_n \in \mathcal{L}^1(P)$, og $\mathrm{A}_n$ er $\mathcal{F}_{n-1}$-målelig for alle $n i \mathbb{N}$.
\end{enumerate}

Hvis $\mathrm{X}_n=\tilde{\mathrm{M}}_n+\tilde{\mathrm{A}}_n$ er endnu en dekomposition af $\mathrm{X}_n$, således at (a) og (b) er opfyldte, da gælder der, at

$$
\mathrm{M}_n=\tilde{\mathrm{M}}_n \quad P \text {-n.o. } \quad \text { og } \quad \mathrm{A}_n=\tilde{\mathrm{A}}_n \quad P \text {-n.o. } \quad \text { for alle } n i \mathbb{N}_0 .
$$


Hvis $\left(\mathrm{X}_n, \mathcal{F}_n\right)_{n \in \mathbb{N}_0}$ er en sub-martingal (hhv. en super-martingal), da er følgen $\left(\mathrm{A}_n\right)_{n \in \mathbb{N}_0}$ voksende P-n.o. (hhv. aftagende P-n.o.)
    \end{proposition}
\end{theorem-box}
\begin{theorem-box}
    \begin{proposition}[Martingal transforms]
        Lad $\left(\Omega, \mathcal{F}, \mathcal{F}_n, P\right)$ være et filtreret sandsynlighedsfelt, og lad $\left(\mathrm{X}_n\right)_{n \in \mathbb{N}_0}$ være en tilpasset følge af stokastiske variable fra $\mathcal{L}^1(P)$.
Lad endvidere $\left(\mathrm{V}_n\right)_{n \in \mathbb{N}_0}$ være en forudsigelig proces (dvs. $\mathrm{V}_n$ er $\mathcal{F}_{0 \vee(n-1)}$-målelig for alle $n$ ), og antag, at hvert $\mathrm{V}_n$ er begrænset.
Definér nu

$$
\Delta \mathrm{X}_n=\mathrm{X}_n-\mathrm{X}_{n-1} \quad \text { for alle } n i \mathbb{N},
$$

og

$$
\mathrm{V} \bullet \mathrm{X}_n=\mathrm{V}_0 \mathrm{X}_0+\sum_{k=1}^n \mathrm{~V}_k \Delta \mathrm{X}_k \quad \text { for alle } n i \mathbb{N}_0
$$
\begin{enumerate}
    \item[\textnormal{(i)}] Hvis $\left(\mathrm{X}_n, \mathcal{F}_n\right)_{n \in \mathbb{N}_0}$ er en martingal, da er $\left(\mathrm{V} \bullet \mathrm{X}_n, \mathcal{F}_n\right)_{n \in \mathbb{N}_0}$ igen en $M G$.
    \item[\textnormal{(ii)}]  Hvis $\mathrm{V}_n \geq 0$ for alle $n$, og $\left(\mathrm{X}_n, \mathcal{F}_n\right)_{n \in \mathbb{N}_0}$ er en sub-MG (hhv. super- $M G$ ), da er $\left(\mathrm{V} \bullet \mathrm{X}_n, \mathcal{F}_n\right)_{n \in \mathbb{N}_0}$ igen en sub-MG (hhv. super-MG).

\end{enumerate}
\end{proposition}
\end{theorem-box}
\subsection{Stoppetider}
\begin{theorem-box}
    \begin{definition}[Stoppetider]
        Lad $\left(\Omega, \mathcal{F}, \mathcal{F}_n, P\right)$ være et filtreret sandsynlighedsfelt.
En stoppetid (med hensyn til filteret $\left.\left(\mathcal{F}_n\right)_{n \in \mathbb{N}_0}\right)$ er en afbildning
$\tau: \Omega \rightarrow \mathbb{N}_0 \cup\{+\infty\}$, der opfylder betingelsen:

$$
\{\tau>n\} \in \mathcal{F}_n \quad \text { for alle } n \text { i } \mathbb{N}_0
$$


Betingelsen kan ækvivalent formuleres som:

$$
\{\tau \leq n\} \in \mathcal{F}_n \quad \text { for alle } n \text { i } \mathbb{N}_0
$$

eller som

$$
\{\tau=n\} \in \mathcal{F}_n \quad \text { for alle } n \text { i } \mathbb{N}_0
$$


Specielt følger det, at $\tau$ er $\mathcal{F}-\mathcal{P}\left(\mathbb{N}_0 \cup\{\infty\}\right)$-målelig.
\\En stoppetid $\tau: \Omega \rightarrow \mathbb{N}_0 \cup\{\infty\}$ kaldes
\\- endelig, hvis $P(\tau<\infty)=1$.
\\- begrænset, hvis der findes en konstant $K$ i $\mathbb{N}$, således at $\tau(\omega) \leq K$ for alle $\omega$ i $\Omega$.
\end{definition}
\end{theorem-box}
\begin{remark}
    
\end{remark}
\begin{example}
    \begin{enumerate}
        \item $\text { For ethvert } n \text { i } \mathbb{N}_0 \cup\{\infty\} \text { er } \tau \equiv n \text { en stoppetid. }$
        \item Antag, at $\left(\mathrm{X}_n\right)_{n \in \mathbb{N}_0}$ er en tilpasset følge af stokastiske variable, og at $\left(A_n\right)_{n \in \mathbb{N}_0}$ er en følge af Borel-mængder i $\mathbb{R}$.

        Da definerer udtrykket
        
        $$
        \tau_A(\omega)=\inf \left\{k \in \mathbb{N}_0 \mid x_k(\omega) \in A_k\right\}, \quad(\omega \in \Omega)
        $$
        
        en stoppetid.
        Her benyttes konventionen: $\inf (\emptyset)=\infty$.
        For ethvert $n$ i $\mathbb{N}_0$ har vi nemlig, at
        
        $$
        \{\tau>n\}=\left\{\mathrm{x}_0 \notin A_0\right\} \cap\left\{\mathrm{X}_1 \notin A_1\right\} \cap \cdots \cap\left\{\mathrm{X}_n \notin A_n\right\} \stackrel{?}{\in} \mathcal{F}_n .
        $$
        \item (C) En afbildning $\tau: \Omega \rightarrow \mathbb{N}_0 \cup\{\infty\}$ er en stoppetid, hvis og kun hvis der findes en følge $\left(F_n\right)_{n \in \mathbb{N}_0}$ af hændelser, således at

        $$
        \tau(\omega)=\inf \left\{k \in \mathbb{N}_0 \mid \omega \in F_k\right\}
        $$
        (*)
        og
        
        $$
        F_n \in \mathcal{F}_n \quad \text { for alle } n \text { i } \mathbb{N}_0
        $$
        (**)
        
        Hvis (*) og (**) er opfyldte, har vi nemlig for heert $n$ i $\mathbb{N}_0$, at
        
        $$
        \{\tau>n\}=F_0^c \cap \cdots \cap F_n^c \stackrel{?}{\in} \mathcal{F}_n .
        $$
        
        
        Omvendt kan vi for en givet stoppetid $\tau$ definere
        
        $$
        F_n=\{\tau \leq n\}, \quad\left(n \in \mathbb{N}_0\right)
        $$
        
        hvorved (*) og (**) er opfyldte.
    \end{enumerate}
\end{example}
\begin{theorem-box}
    \begin{proposition}
        Lad $(\Omega, \mathcal{F}, \mathcal{F}_n, P)$ være et filtreret sandsynligheds-felt. 

        Da gælder følgende udsagn:
        \begin{enumerate}
            \item[\textnormal{(i)}] Lad $\tau_1,\tau_1$ være stoppetider på $(\Omega, \mathcal{F}, \mathcal{F}_n, P)$.\\Da er $\tau_1+\tau_2, \tau_1\vee\tau_2$ og $\tau_1\wedge\tau_2$ igen stoppetider. 
            \item[\textnormal{(ii)}] Lad $(\tau_k)_{k\in\N}$ være en følge af stoppetider på $(\Omega, \mathcal{F}, \mathcal{F}_n, P)$.\\Da er $\sup_{k\in\N}\tau_k, \inf_{k\in\N}\tau_k$ og $\sum_{k=1}^{\infty}\tau_k$ igen stoppetider. 
        \end{enumerate}
    \end{proposition}
\end{theorem-box}
\begin{theorem-box}
    \begin{definition}[$\sigma$-algebraen $\mathcal{F}_\tau$]
        Lad $(\Omega, \mathcal{F}, P)$ være et sandsynlighedsfelt, og lad $\left(\mathcal{F}_n\right)_{n \in \mathbb{N}_0}$ være et filter herpå.

        Vi definerer så
        
        $$
        \mathcal{F}_{\infty}=\sigma\left(\bigcup_{n \in \mathbb{N}_0} \mathcal{F}_n\right) .
        $$
        
        
        Lad videre $\tau: \Omega \rightarrow \mathbb{N}_0 \cup\{\infty\}$ være en stoppetid mht. $\left(\mathcal{F}_n\right)_{n \in \mathbb{N}_0}$.
        Vi definerer så
        
        $$
        \mathcal{F}_\tau=\left\{F \in \mathcal{F}_{\infty} \mid F \cap\{\tau=n\} \in \mathcal{F}_n \text { for alle } n \text { i } \mathbb{N}_0\right\}
        $$
        
        
        Bemærk, at for $F$ fra $\mathcal{F}_\tau$ gælder der automatisk, at
        
        $$
        F \cap\{\tau=\infty\} \in \mathcal{F}_{\infty}
        $$
        
        idet $F \in \mathcal{F}_{\infty}$, og $\{\tau=\infty\}=\bigcap_{k \in \mathbb{N}_0}\{\tau>k\} \in \mathcal{F}_{\infty}$.
    \end{definition}
\end{theorem-box}
\begin{theorem-box}
    \begin{lemma}
        Lad $\left(\Omega, \mathcal{F}, \mathcal{F}_n, P\right)$ være et filtreret sandsynlighedsfelt.
\begin{enumerate}
    \item[\textnormal{(i)}]  For enhver stoppetid $\tau: \Omega \rightarrow \mathbb{N}_0 \cup\{\infty\}$ er $\mathcal{F}_\tau$ en $\sigma$-algebra.
    \item[\textnormal{(ii)}]  For enhver $\mathcal{F}_\tau$-målelig stokastisk variabel $\mathrm{Y}: \Omega \rightarrow \mathbb{R}$ gælder der, at $\mathrm{Y} 1_{\{\tau=n\}}$ er $\mathcal{F}_n$-målelig for alle $n i \mathbb{N}_0 \cup\{\infty\}$.
\end{enumerate}
    \end{lemma}
\end{theorem-box}
\begin{theorem-box}
    \begin{proposition}
        Lad $\tau$ være en stoppetid med hensyn til filteret $\left(\mathcal{F}_n\right)_{n \in \mathbb{N}_0}$.
Da gælder følgende udsagn:
\begin{enumerate}
    \item[\textnormal{(i)}] $A \cap\{\tau=n\} \in \mathcal{F}_\tau \cap \mathcal{F}_n$ for alle $n i \mathbb{N}_0 \cup\{\infty\}$ og $A$ i $\mathcal{F}_n$.
    \item[\textnormal{(ii)}] $\tau \operatorname{er} \mathcal{F}_\tau-\mathcal{P}\left(\mathbb{N}_0 \cup\{\infty\}\right)$-målelig.
    \item[\textnormal{(iii)}] For ethvert $n$ i $\mathbb{N}_0 \cup\{\infty\}$ og X i $\mathcal{L}^1(P)$ gælder formlen:

    $$
    \mathbb{E}\left[\mathrm{X} \mid \mathcal{F}_\tau\right] 1_{\{\tau=n\}}=\mathbb{E}\left[\mathrm{X} \mid \mathcal{F}_n\right] 1_{\{\tau=n\}} \stackrel{7.3 .2}{\stackrel{\downarrow}{=}} \mathbb{E}\left[\mathrm{X} 1_{\{\tau=n\}} \mid \mathcal{F}_n\right] \quad \text { P-n.o. }
    $$
\end{enumerate}
    \end{proposition}
\end{theorem-box}
\begin{theorem-box}
    \begin{proposition}
        Lad $\left(\Omega, \mathcal{F}, \mathcal{F}_n, P\right)$ være et filtreret sandsynligheds-felt, og lad $\tau, \tau_1, \tau_2$ være stoppetider herpå.
\begin{enumerate}
    \item[\textnormal{(i)}] Hvis $\tau \equiv n \in \mathbb{N}_0 \cup\{\infty\}$, gælder der, at $\mathcal{F}_\tau=\mathcal{F}_n$.
    \item[\textnormal{(ii)}]Hvis $\tau_1 \leq \tau_2$, gælder der, at $\mathcal{F}_{\tau_1} \subseteq \mathcal{F}_{\tau_2}$.
    \item[\textnormal{(iii)}] $\mathcal{F}_{\tau_1 \wedge \tau_2}=\mathcal{F}_{\tau_1} \cap \mathcal{F}_{\tau_2}$.
    \item[\textnormal{(iv)}]  $\{\tau \leq n\} \in \mathcal{F}_\tau \cap \mathcal{F}_n=\mathcal{F}_{n \wedge \tau}$ for alle $n i \mathbb{N}_0$.
    \item[\textnormal{(v)}] For alle $A$ i $\mathcal{F}_{\tau_1}$ gælder der, at

    $$
    A \cap\left\{\tau_1<\tau_2\right\} \in \mathcal{F}_{\tau_1 \wedge \tau_2}, \quad \text { og } \quad A \cap\left\{\tau_1 \leq \tau_2\right\} \in \mathcal{F}_{\tau_1 \wedge \tau_2}
    $$
    \item[\textnormal{(vi)}] For enhver mængde $A$ i $\mathcal{F}_{\tau_1 \wedge \tau_2}$ er $\tau_A:=\tau_1 1_A+\tau_2 1_{A^c}$ igen en stoppetid.
\end{enumerate}
    \end{proposition}
\end{theorem-box}
\begin{theorem-box}
    \begin{proposition}
        Lad $\left(\Omega, \mathcal{F}, \mathcal{F}_n, P\right)$ være et filtreret sandsynligheds-felt, og lad $\tau$ være en stoppetid herpå.

Da gælder formlen:

$$
\mathcal{F}_\tau=\sigma\left(\bigcup_{n \in \mathbb{N}_0} \mathcal{F}_{\tau \wedge n}\right)
$$

    \end{proposition}
\end{theorem-box}
\begin{theorem-box}
    \begin{proposition}
        Lad $\left(\Omega, \mathcal{F}, \mathcal{F}_n, P\right)$ være et filtreret sandsynlighedsfelt, lad $\sigma, \tau$ være stoppetider herpå, og betragt de tilhørende $\sigma$-algebraer $\mathcal{F}_\tau$ og $\mathcal{F}_\sigma$.

For enhver stokastisk variabel X fra $\mathcal{L}^1(P)$ gælder da følgende udsagn:
\begin{enumerate}
    \item[\textnormal{(i)}]Hvis $\mathrm{X} @ \mathcal{F}_\tau$, gælder formlen:

    $$
    \mathbb{E}\left[\mathrm{X} \mid \mathcal{F}_\sigma\right]=\mathbb{E}\left[\mathrm{X} \mid \mathcal{F}_{\sigma \wedge \tau}\right] .
    $$
    \item[\textnormal{(ii)}] Generelt gælder formlen:

    $$
    \mathbb{E}\left[\mathbb{E}\left[\mathrm{X} \mid \mathcal{F}_\tau\right] \mid \mathcal{F}_\sigma\right]=\mathbb{E}\left[\mathrm{X} \mid \mathcal{F}_{\sigma \wedge \tau}\right]=\mathbb{E}\left[\mathbb{E}\left[\mathrm{X} \mid \mathcal{F}_\sigma\right] \mid \mathcal{F}_\tau\right]
    $$
\end{enumerate}
    \end{proposition}
\end{theorem-box}
\subsection*{Den stokastiske variabel $\X_\tau$}
Lad $\left(\mathrm{X}_{n \in \mathbb{N}}\right)$ være en tilpasset følge af stok. var. på $\left(\Omega, \mathcal{F}, \mathcal{F}_n, P\right)$.
Vi definerer da den stokastiske variable $\mathrm{X}_{\infty}: \Omega \rightarrow \mathbb{R}$ ved

$$
\mathrm{X}_{\infty}(\omega)= \begin{cases}\lim _{n \rightarrow \infty} \mathrm{X}_n(\omega), & \text { hvis } \lim _{n \rightarrow \infty} \mathrm{X}_n(\omega) \text { eksisterer i } \mathbb{R} \\ 0, & \text { ellers }\end{cases}
$$


Det følger fra Korollar 4.3.11 i [M\&I], at $\mathrm{X}_{\infty}$ er $\mathcal{F}_{\infty}-\mathcal{B}(\mathbb{R})$-målelig.
For enhver stoppetid $\tau$ definerer vi den stok. var. $\mathrm{X}_\tau$ ved formlen:

$$
\begin{aligned}
\mathrm{X}_\tau(\omega) & =\mathrm{X}_{\tau(\omega)}(\omega)= \begin{cases}\mathrm{X}_n(\omega), & \text { hvis } \tau(\omega)=n \text { for et } n \text { i } \mathbb{N}_0 \\
\mathrm{X}_{\infty}(\omega), & \text { hvis } \tau(\omega)=\infty\end{cases} \\
& =\mathrm{X}_{\infty}(\omega) 1_{\{\tau=\infty\}}(\omega)+\sum_{k=0}^{\infty} \mathrm{X}_k(\omega) 1_{\{\tau=k\}}(\omega)
\end{aligned}
$$


Specielt viser (2), at $\mathrm{X}_\tau @ \mathcal{F}_{\infty}$.
Husk: $\quad \mathrm{X}_\tau(\omega)= \begin{cases}\mathrm{X}_n(\omega), & \text { hvis } \tau(\omega)=n \text { for et } n \text { i } \mathbb{N}_0, \\ \mathrm{X}_{\infty}(\omega), & \text { hvis } \tau(\omega)=\infty,\end{cases}$
Alternativt kan vi skrive $\mathrm{X}_\tau$ på formen:

$$
\begin{aligned}
\mathrm{X}_\tau(\omega) & = \begin{cases}\mathrm{X}_n(\omega), & \text { hvis } \tau(\omega)=n \text { for et } n \mathrm{i} \mathbb{N}_0, \\
\lim _{k \rightarrow \infty} \mathrm{X}_k(\omega), & \text { hvis } \tau(\omega)=\infty, \text { og } \lim _{k \rightarrow \infty} \mathrm{X}_k(\omega) \text { eksisterer } \mathrm{i} \mathbb{R}, \\
0, & \text { ellers }\end{cases} \\
& = \begin{cases}\lim _{k \rightarrow \infty} \mathrm{X}_{\tau(\omega) \wedge k}(\omega), & \text { hvis grænseværdien eksisterer } \mathrm{i} \mathbb{R}, \\
0, & \text { ellers. }\end{cases}
\end{aligned}
$$


Specielt ser vi, at

$$
\left|\mathrm{X}_\tau\right| \leq \liminf _{k \rightarrow \infty}\left|\mathrm{X}_{\tau \wedge k}\right| .
$$
\begin{theorem-box}
    \begin{definition}
        For enhver stoppetid $\tau$ definerer vi den stok. var. $\mathrm{X}_\tau$ ved formlen:

$$
\begin{aligned}
\mathrm{X}_\tau(\omega) & =\mathrm{X}_{\tau(\omega)}(\omega)= \begin{cases}\mathrm{X}_n(\omega), & \text { hvis } \tau(\omega)=n \text { for et } n \text { i } \mathbb{N}_0 \\
\mathrm{X}_{\infty}(\omega), & \text { hvis } \tau(\omega)=\infty\end{cases} \\
& =\mathrm{X}_{\infty}(\omega) 1_{\{\tau=\infty\}}(\omega)+\sum_{k=0}^{\infty} \mathrm{X}_k(\omega) 1_{\{\tau=k\}}(\omega)
\end{aligned}
$$
    \end{definition}
\end{theorem-box}
\begin{remark}
    Alternativt kan vi skrive $\mathrm{X}_\tau$ på formen:

$$
\begin{aligned}
\mathrm{X}_\tau(\omega) & = \begin{cases}\mathrm{X}_n(\omega), & \text { hvis } \tau(\omega)=n \text { for et } n \mathrm{i} \mathbb{N}_0, \\
\lim _{k \rightarrow \infty} \mathrm{X}_k(\omega), & \text { hvis } \tau(\omega)=\infty, \text { og } \lim _{k \rightarrow \infty} \mathrm{X}_k(\omega) \text { eksisterer } \mathrm{i} \mathbb{R}, \\
0, & \text { ellers }\end{cases} \\
& = \begin{cases}\lim _{k \rightarrow \infty} \mathrm{X}_{\tau(\omega) \wedge k}(\omega), & \text { hvis grænseværdien eksisterer } \mathrm{i} \mathbb{R}, \\
0, & \text { ellers. }\end{cases}
\end{aligned}
$$

\end{remark}
\begin{theorem-box}
    \begin{proposition}
        $\operatorname{Lad}\left(\Omega, \mathcal{F}, \mathcal{F}_n, P\right)$ være et filtreret sandsynligheds-felt, og lad $\left(\mathrm{X}_n\right)_{n \in \mathbb{N}}$ være en tilpasset følge af stokastiske variable herpå.

Lad videre $\tau$ være en stoppetid på $\left(\Omega, \mathcal{F}, \mathcal{F}_n, P\right)$.
Da gælder følgende udsagn:
\begin{enumerate}
    \item[\textnormal{(i)}] $\mathrm{X}_\tau \Subset \mathcal{F}_\tau$.
    \item[\textnormal{(ii)}]Hvis $\sup _{n \in \mathbb{N}} \mathbb{E}\left[\left|\mathrm{X}_{\tau \wedge n}\right|\right]<\infty$, gælder der, at $\mathrm{X}_\tau \in \mathcal{L}^1(P)$.
    \item[\textnormal{(iii)}]Hvis $\tau$ er begrænset, og $\mathrm{X}_n \in \mathcal{L}^1(P)$ for alle $n$ i $\mathbb{N}$, gælder der, at $\mathrm{X}_\tau \in \mathcal{L}^1(P)$.
\end{enumerate}
    \end{proposition}
\end{theorem-box}
\subsection{Optional sampling(første version)}
\begin{theorem-box}
    \begin{proposition}
        $\operatorname{Lad}\left(\Omega, \mathcal{F}, \mathcal{F}_n, P\right)$ være et filtreret sandsynlighedsfelt, lad $\tau$ være en stoppetid, og lad $\left(\mathrm{X}_n\right)_{n \in \mathbb{N}_0}$ være en følge af stokastiske variable.

Da gælder følgende udsagn:
\begin{enumerate}
    \item[\textnormal{(i)}] Hvis $\left(\mathrm{X}_n, \mathcal{F}_n\right)_{n \in \mathbb{N}_0}$ er en sub-martingal, da er $\left(\mathrm{X}_{n \wedge \tau}, \mathcal{F}_n\right)_{n \in \mathbb{N}_0}$ igen en sub-martingal.
    \item[\textnormal{(ii)}]Hvis $\left(\mathrm{X}_n, \mathcal{F}_n\right)_{n \in \mathbb{N}_0}$ er en super-martingal, da er $\left(\mathrm{X}_{n \wedge \tau}, \mathcal{F}_n\right)_{n \in \mathbb{N}_0}$ igen en super-martingal.
    \item[\textnormal{(iii)}] Hvis $\left(\mathrm{X}_n, \mathcal{F}_n\right)_{n \in \mathbb{N}_0}$ er en martingal, da er $\left(\mathrm{X}_{n \wedge \tau}, \mathcal{F}_n\right)_{n \in \mathbb{N}_0}$ igen en martingal.
\end{enumerate}
    \end{proposition}
\end{theorem-box}
\begin{theorem-box}
    \begin{proposition}
        Lad $\left(\Omega, \mathcal{F}, \mathcal{F}_n, P\right)$ være et filtreret sandsynlighedsfelt, og lad $\sigma, \tau$ være begrænsede stoppetider, således at $\sigma \leq \tau$.
\begin{enumerate}
    \item[\textnormal{(i)}] Hvis $\left(\mathrm{X}_n, \mathcal{F}_n\right)_{n \in \mathbb{N}_0}$ er en sub-martingal, gælder der, at $\mathrm{X}_\sigma, \mathrm{X}_\tau \in \mathcal{L}^1(P)$, og at

    $$
    \mathbb{E}\left[\mathrm{x}_0\right] \leq \mathbb{E}\left[\mathrm{x}_\sigma\right] \leq \mathbb{E}\left[\mathrm{x}_\tau\right], \quad \text { og } \quad \mathbb{E}\left[\mathrm{x}_\tau \mid \mathcal{F}_\sigma\right] \geq \mathrm{X}_\sigma \quad \text { P-n.o. }
    $$
    
    \item[\textnormal{(ii)}] Hvis $\left(\mathrm{X}_n, \mathcal{F}_n\right)_{n \in \mathbb{N}_0}$ er en super-martingal, gælder der, at $\mathrm{X}_\sigma, \mathrm{X}_\tau \in \mathcal{L}^1(P)$, og at

    $$
    \mathbb{E}\left[\mathrm{x}_0\right] \geq \mathbb{E}\left[\mathrm{x}_\sigma\right] \geq \mathbb{E}\left[\mathrm{X}_\tau\right], \quad \text { og } \quad \mathbb{E}\left[\mathrm{X}_\tau \mid \mathcal{F}_\sigma\right] \leq \mathrm{X}_\sigma \quad \text { P-n.o. }
    $$
    \item[\textnormal{(iii)}] Hvis $\left(\mathrm{X}_n, \mathcal{F}_n\right)_{n \in \mathbb{N}_0}$ er en martingal, gælder der, at $\mathrm{X}_\sigma, \mathrm{X}_\tau \in \mathcal{L}^1(P)$, og at $\quad \mathbb{E}\left[\mathrm{X}_0\right]=\mathbb{E}\left[\mathrm{X}_\sigma\right]=\mathbb{E}\left[\mathrm{X}_\tau\right], \quad$ og $\quad \mathbb{E}\left[\mathrm{X}_\tau \mid \mathcal{F}_\sigma\right]=\mathrm{X}_\sigma \quad$ P-n.o.
  
\end{enumerate}
  \end{proposition}
\end{theorem-box}
\begin{theorem-box}
    \begin{corollary}
        $\operatorname{Lad}\left(\Omega, \mathcal{F}, \mathcal{F}_n, P\right)$ være et filtreret sandsynlighedsfelt, og lad $\tau$ være en stoppetid. Lad videre $\left(\mathrm{X}_n\right)_{n \in \mathbb{N}}$ være en tilpasset følge fra $\mathcal{L}^1(P)$.
\begin{enumerate}
    \item[\textnormal{(i)}] Hvis $\left(\mathrm{X}_n, \mathcal{F}_n\right)$ er en sub-MG (hhv. super-MG), da er $\left(\mathrm{X}_{n \wedge \tau}, \mathcal{F}_n\right)$ igen en sub-MG (hhv. super-MG), og der gælder implikationen:

    $$
    \sup _{n \in \mathbb{N}_0} \mathbb{E}\left[\left|\mathrm{X}_n\right|\right]<\infty \Longrightarrow \sup _{n \in \mathbb{N}_0} \mathbb{E}\left[\left|\mathrm{X}_{n \wedge \tau}\right|\right]<\infty
    $$
    \item[\textnormal{(ii)}]\begin{enumerate}
        \item[\textnormal{(a)}]  Hvis $\left(\mathrm{X}_n, \mathcal{F}_n\right)$ er en sub-MG og $\left\{\mathrm{X}_n^{+} \mid n \in \mathbb{N}_0\right\}$ er uniformt integrabel, da er $\left\{\mathrm{X}_{n \wedge \tau}^{+} \mid n \in \mathbb{N}_0\right\}$ ligeledes uniformt integrabel.
        \item[\textnormal{(b)}] Hvis $\left(\mathrm{X}_n, \mathcal{F}_n\right)$ er en super- $M G$ og $\left\{\mathrm{X}_n^{-} \mid n \in \mathbb{N}_0\right\}$ er uniformt integrabel, da er $\left\{\mathrm{X}_{n \wedge \tau}^{-} \mid n \in \mathbb{N}_0\right\}$ ligeledes uniformt integrabel.
        \item[\textnormal{(c)}] Hvis $\left(\mathrm{X}_n, \mathcal{F}_n\right)$ er en uniformt integrabel $M G$, da er $\left\{\mathrm{X}_{n \wedge \tau} \mid n \in \mathbb{N}_0\right\}$ ligeledes uniformt integrabel.
    \end{enumerate} 
\end{enumerate}
    \end{corollary}
\end{theorem-box}
\subsection{Martingale Maximal-uligheder}
\begin{theorem-box}
    \begin{proposition}
        Lad $\left(\Omega, \mathcal{F}, \mathcal{F}_n, P\right)$ være et filtreret sandsynlighedsfelt, lad $\left(\mathrm{X}_n\right)_{n \in \mathbb{N}}$ være en tilpasset følge af integrable stokastiske variable, og lad $\ell$ være et strengt positivt tal.



        \begin{enumerate}
            \item[\textnormal{(i)}] Hvis $\left(\mathrm{X}_n, \mathcal{F}_n\right)$ er en sub-MG, da gælder for hvert $n$ i $\mathbb{N}$ uligheden:

            $$
            \ell P\left(\max _{0 \leq k \leq n} \mathrm{X}_k>\ell\right) \leq \int_{\left\{\max _{0 \leq k \leq n} \mathrm{x}_k>\ell\right\}} \mathrm{X}_n \mathrm{~d} P \leq \mathbb{E}\left[\mathrm{X}_n^{+}\right] .
            $$
            \item[\textnormal{(ii)}] Hvis $\left(\mathrm{X}_n, \mathcal{F}_n\right)$ er en sub-MG, da gælder for hvert $n$ i $\mathbb{N}$ uligheden:

            $$
            \ell P\left(\min _{0 \leq k \leq n} \mathrm{X}_k<-\ell\right) \leq \mathbb{E}\left[\mathrm{X}_n^{+}\right]-\mathbb{E}\left[\mathrm{X}_0\right] .
            $$
            \item[\textnormal{(iii)}] Hvis $\left(\mathrm{X}_n, \mathcal{F}_n\right)$ er en sub-MG eller en super-MG, da gælder for hvert $n i$ $\mathbb{N}$ ulighederne:

            $$
            \ell P\left(\max _{0 \leq k \leq n}\left|\mathrm{X}_k\right|>\ell\right) \leq 2 \mathbb{E}\left[\left|\mathrm{X}_n\right|\right]+\mathbb{E}\left[\left|\mathrm{X}_0\right|\right] \leq 3 \max _{0 \leq k \leq n} \mathbb{E}\left[\left|\mathrm{X}_k\right|\right]
            $$
                    
            \item[\textnormal{(iv)}]
            Hvis $\left(\mathrm{X}_n, \mathcal{F}_n\right)$ er en sub-MG eller en super-MG, da gælder uligheden:
            
            $$
            \ell P\left(\sup _{k \in \mathbb{N}_0}\left|\mathrm{x}_k\right|>\ell\right) \leq 3 \sup _{k \in \mathbb{N}_0} \mathbb{E}\left[\left|\mathrm{x}_k\right|\right] .
            $$
        \end{enumerate}
    \end{proposition}
\end{theorem-box}
\begin{theorem-box}
    \begin{corollary}
        $\operatorname{Lad}\left(\Omega, \mathcal{F}, \mathcal{F}_n, P\right)$ være et filtreret sandsynlighedsfelt, og lad $\left(\mathrm{X}_n, \mathcal{F}_n\right)$ være en sub-MG eller en super-MG. Antag, at

$$
\sup _{n \in \mathbb{N}_0} \mathbb{E}\left[\left|\mathrm{X}_n\right|\right]<\infty
$$


Da gælder der, at

$$
P\left(\sup _{n \in \mathbb{N}_0}\left|\mathrm{X}_n\right|<\infty\right)=1
$$

    \end{corollary}
\end{theorem-box}
\begin{theorem-box}
    \begin{proposition}
        $\operatorname{Lad}\left(\Omega, \mathcal{F}, \mathcal{F}_n, P\right)$ være et filtreret sandsynlighedsfelt, og lad $\left(\mathrm{X}_n, \mathcal{F}_n\right)$ være en tilpasset følge, således at $\left(\left|\mathrm{X}_n\right|, \mathcal{F}_n\right)$ er en sub-MG.

Lad videre $p$ være et tal $i(1, \infty)$, og lad q $i(1, \infty)$ være bestemt ved, at $\frac{1}{p}+\frac{1}{q}=1$.
\begin{enumerate}
    \item[\textnormal{(i)}] For ethvert $n i \mathbb{N}_0$ gælder uligheden:

    $$
    \left\|\max _{0 \leq k \leq n}\left|\mathrm{X}_k\right|\right\|_p \leq q\left\|\mathrm{X}_n\right\|_p
    $$
    \item[\textnormal{(ii)}] Der gælder endvidere uligheden:

    $$
    \mathbb{E}\left[\sup _{n \in \mathbb{N}_0}\left|\mathrm{X}_n\right|^p\right] \leq q^p \sup _{n \in \mathbb{N}_0} \mathbb{E}\left[\left|\mathrm{X}_n\right|^p\right]
    $$    
\end{enumerate}
    \end{proposition}
\end{theorem-box}
\begin{theorem-box}
    \begin{lemma}
        Lad X og Y være ikke-negative stokastiske variable på $(\Omega, \mathcal{F}, P)$, og antag, at

$$
\forall \ell>0: \ell P(\mathrm{x}>\ell) \leq \int_{\{\mathrm{x}>\ell\}} \mathrm{Y} \mathrm{~d} P .
$$


For ethvert p i $(1, \infty)$ gælder da uligheden:

$$
\mathbb{E}\left[\mathrm{X}^p\right]^{1 / p} \leq \frac{p}{p-1} \mathbb{E}\left[\mathrm{Y}^p\right]^{1 / p} .
$$

    \end{lemma}
\end{theorem-box}
\subsection{Opkrydsninger og Martingal Konvergens Sætningen}
Definition af opkrydsninger
Lad $r, s$ være reelle tal, således at $r<s$.
For en følge $\left(x_n\right)_{n \in \mathbb{N}_0}$ af reelle tal, definerer vi tallene

$$
0 \leq \tau_1 \leq \sigma_1 \leq \tau_2 \leq \sigma_2 \leq \tau_3 \leq \sigma_3 \leq \cdots
$$

ved:

$$
\begin{array}{lll}
\tau_1=\inf \left\{n \in \mathbb{N}_0 \mid x_n<r\right\}, & \text { og } & \sigma_1=\inf \left\{n>\tau_1 \mid x_n>s\right\} \\
\tau_2=\inf \left\{n>\sigma_1 \mid x_n<r\right\}, & \text { og } & \sigma_2=\inf \left\{n>\tau_2 \mid x_n>s\right\}
\end{array}
$$

og generelt for $k i\{2,3,4, \ldots\}$ sætter vi

$$
\tau_k=\inf \left\{n>\sigma_{k-1} \mid x_n<r\right\}, \quad \text { og } \quad \sigma_k=\inf \left\{n>\tau_k \mid x_n>s\right\} .
$$


Som sædvanlig benytter vi her konventionen: $\inf (\emptyset)=+\infty$.
Bemærk, at der for ethvert $k \mathbf{i} \mathbb{N}$ gælder implikationerne:

$$
\tau_k=\sigma_k \Longrightarrow \tau_k=\infty, \quad \text { og } \quad \sigma_k=\tau_{k+1} \Longrightarrow \sigma_k=\infty
$$
Vi definerer derefter antallet af opkrydsninger fra $r$ til $s$ for $\left(x_n\right)_{n \in \mathbb{N}_0}$ ved formlen:

$$
\begin{aligned}
U_{r, s}=\sum_{k=1}^{\infty} 1_{[0, \infty)}\left(\sigma_k\right)= & \#\left\{k \in \mathbb{N} \mid \sigma_k<\infty\right\} \\
= & \begin{cases}0, & \text { hvis } \sigma_1=\infty \\
m, & \text { hvis } \sigma_m<\infty, \text { og } \sigma_{m+1}=\infty \\
\infty, & \text { hvis } \sigma_m<\infty \text { for alle } m \text { i } \mathbb{N}\end{cases}
\end{aligned}
$$


Vi definerer endvidere antallet af opkrydsninger fra $r$ til $s i$ "tidsintervallet" $[0, n]$ for $\left(x_n\right)_{n \in \mathbb{N}_0}$ ved formlen:

$$
\begin{aligned}
U_{r, s}^{(n)}=\sum_{k=1}^{\infty} 1_{[0, n]}\left(\sigma_k\right) & =\#\left\{k \in \mathbb{N} \mid \sigma_k \leq n\right\} \\
& = \begin{cases}0, & \text { hvis } \sigma_1>n \\
m, & \text { hvis } \sigma_m \leq n, \text { og } \sigma_{m+1}>n .\end{cases}
\end{aligned}
$$
\begin{remark}
    \begin{enumerate}
        \item For ethvert $n$ i $\mathbb{N}$ gælder der, at

        $$
        U_{r, s}^{(n)} \leq\left[\frac{n+1}{2}\right] .
        $$
        
\item Vi finder ved anvendelse af Monoton Konvergens, at

$$
U_{r, s}^{(n)}=\sum_{k=1}^{\infty} 1_{[0, n]}\left(\sigma_k\right) \uparrow \sum_{k=1}^{\infty} 1_{[0, \infty)}\left(\sigma_k\right)=U_{r, s}
$$

for $n \rightarrow \infty$.
 \item Der gælder bi-implikationen:

 $$
 U_{r, s}=\infty \Longleftrightarrow \#\left\{k \in \mathbb{N}_0 \mid x_k<r\right\}=+\infty=\#\left\{k \in \mathbb{N}_0 \mid x_k>s\right\} .
 $$
 
    \end{enumerate}
\end{remark}
\begin{theorem-box}
    \begin{lemma}
        For enhver følge $\left(x_n\right)_{n \in \mathbb{N}_0}$ af reelle tal er følgende udsagn ækvivalente:
\begin{enumerate}
    \item[\textnormal{(i)}] $\lim _{n \rightarrow \infty} x_n$ eksisterer $i \overline{\mathbb{R}}$
    \item[\textnormal{(ii)}] $U_{r, s}<\infty$ for alle $r$, s i $\mathbb{R}$, således at $r<s$.
    \item[\textnormal{(iii)}]  $U_{r, s}<\infty$ for alle $r, s i \mathbb{Q}$, således at $r<s$.
\end{enumerate}
    \end{lemma}
\end{theorem-box}
\begin{theorem-box}
    \begin{proposition}[Doobs opkrydsningsmulighed]
        Lad $\left(\Omega, \mathcal{F}, \mathcal{F}_n, P\right)$ være et filtreret sandsynlighedsfelt, og lad $\left(\mathrm{X}_n, \mathcal{F}_n\right)$ være en super-martingal herpå.

Lad endvidere $r, s$ være reelle tal, således at $r<s$.
Definér

$$
\tau_1=\inf \left\{n \in \mathbb{N}_0 \mid \mathrm{X}_n<r\right\}, \quad \text { og } \quad \sigma_1=\inf \left\{n>\tau_1 \mid \mathrm{X}_n>s\right\},
$$

og for $k i\{2,3,4, \ldots\}$

$$
\tau_k=\inf \left\{n>\sigma_{k-1} \mid \mathrm{X}_n<r\right\}, \quad \text { og } \quad \sigma_k=\inf \left\{n>\tau_k \mid \mathrm{X}_n>s\right\} .
$$


Definér endvidere,

$$
\mathrm{U}_{r, s}^{(n)}=\sum_{k=1}^{\infty} 1_{[0, n]}\left(\sigma_k\right), \quad(n \in \mathbb{N}),
$$

og

$$
\mathrm{U}_{r, s}=\sum_{k=1}^{\infty} 1_{[0, \infty)}\left(\sigma_k\right) .
$$
Da gælder følgende udsagn:
\begin{enumerate}
    \item[\textnormal{(i)}]  $\tau_k, \sigma_k$ er stoppetider for alle $k i \mathbb{N}$.
    \item[\textnormal{(ii)}] For ethvert $n$ i $\mathbb{N}$ er $\mathrm{U}_{r, s}^{(n)} \mathcal{F}$-målelig, og der gælder ulighederne:

    $$
    (s-r) \mathbb{E}\left[\mathrm{U}_{r, s}^{(n)}\right] \leq \mathbb{E}\left[\mathrm{X}_n^{-}\right]+r^{+}
    $$
    \item[\textnormal{(iii)}] Også $\mathrm{U}_{r, s}$ er $\mathcal{F}$-målelig, og der gælder uligheden:

    $$
    (s-r) \mathbb{E}\left[\mathrm{U}_{r, s}\right] \leq r^{+}+\sup _{n \in \mathbb{N}_0} \mathbb{E}\left[\mathrm{X}_n^{-}\right]
    $$
\end{enumerate}
    \end{proposition}
\end{theorem-box}
\begin{remark}
    
\end{remark}
\begin{theorem-box}
    \begin{proposition}[Martingal konvergenssætningen]
        Lad $\left(\Omega, \mathcal{F}, \mathcal{F}_n, P\right)$ være et filtreret sandsynlighedsfelt, og lad $\left(\mathrm{X}_n, \mathcal{F}_n\right)$ være en sub-martingal eller en super-martingal.

Antag, at $\sup _{n \in \mathbb{N}_0} \mathbb{E}\left[\left|\mathrm{X}_n\right|\right]<\infty$.
Da eksisterer $\lim _{n \rightarrow \infty} \mathrm{X}_n(\omega) i \mathbb{R}$ for P-n.a. $\omega$.
Med andre ord gælder der, at $\mathrm{X}_n \rightarrow \mathrm{X}_{\infty}$ P-n.o., hvor

$$
\mathrm{X}_{\infty}(\omega)= \begin{cases}\lim _{n \rightarrow \infty} \mathrm{X}_n(\omega), & \text { hvis } \lim _{n \rightarrow \infty} \mathrm{X}_n(\omega) \text { eksisterer } i \mathbb{R}, \\ 0, & \text { ellers. }\end{cases}
$$


Der gæ/der endvidere, at

$$
\mathbb{E}\left[\left|\mathrm{X}_{\infty}\right|\right]<\infty .
$$

    \end{proposition}
\end{theorem-box}
\subsection{Uniformt integrable (sub- og super-)martingaler}
\begin{theorem-box}
    \begin{corollary}
        $\operatorname{Lad}\left(\Omega, \mathcal{F}, \mathcal{F}_n, P\right)$ være et filtreret sandsynlighedsfelt, og lad $\left(\mathrm{X}_n, \mathcal{F}_n\right)_{n \in \mathbb{N}_0}$ være en sub-martingal eller en super-martingal.

Hvis $\left\{\mathrm{X}_n \mid n \in \mathbb{N}_0\right\}$ er uniformt integrabel, gælder der, at $\mathrm{X}_n \rightarrow \mathrm{X}_{\infty} \quad$ n.o. og i 1-middel.
    \end{corollary}
\end{theorem-box}
\begin{theorem-box}
    \begin{corollary}
        $\operatorname{Lad}\left(\Omega, \mathcal{F}, \mathcal{F}_n, P\right)$ være et filtreret sandsynlighedsfelt, og lad $\left(\mathrm{X}_n\right)_{n \in \mathbb{N}_0}$ være en følge af stokastiske variable herpå.
\begin{enumerate}
    \item[\textnormal{(i)}] Hvis $\left(\mathrm{X}_n, \mathcal{F}_n\right)_{n \in \mathbb{N}_0}$ er en sub-martingal, og $\left\{\mathrm{X}_n^{+} \mid n \in \mathbb{N}_0\right\}$ er uniformt integrabel, så gælder der, at $\mathrm{X}_{\infty} \in \mathcal{L}^1(P)$, og

    $$
    \mathrm{X}_n \leq \mathbb{E}\left[\mathrm{X}_{\infty} \mid \mathcal{F}_n\right] \quad \text { for alle } n i \mathbb{N}_0
    $$
    \item[\textnormal{(ii)}] Hvis $\left(\mathrm{X}_n, \mathcal{F}_n\right)_{n \in \mathbb{N}_0}$ er en martingal, og $\left\{\mathrm{X}_n \mid n \in \mathbb{N}_0\right\}$ er uniformt integrabel, så gælder der, at $\mathrm{X}_{\infty} \in \mathcal{L}^1(P)$, og

    $$
    \mathrm{X}_n=\mathbb{E}\left[\mathrm{X}_{\infty} \mid \mathcal{F}_n\right] \quad \text { for alle } n i \mathbb{N}_0
    $$
\end{enumerate}
    \end{corollary}
\end{theorem-box}
\begin{theorem-box}
    \begin{proposition}
        Lad $\left(\Omega, \mathcal{F}, \mathcal{F}_n, P\right)$ være et filtreret sandsynlighedsfelt, og sæt

$$
\mathcal{F}_{\infty}=\sigma\left(\bigcup_{n \in \mathbb{N}_0} \mathcal{F}_n\right)
$$


Lad videre X være en stokastisk variabel i $\mathcal{L}^1(P)$, og definér

$$
\mathrm{X}_n=\mathbb{E}\left[\mathrm{X} \mid \mathcal{F}_n\right], \quad\left(n \in \mathbb{N}_0\right)
$$


Da gælder følgende udsagn:
\begin{enumerate}
    \item[\textnormal{(i)}]$\mathrm{X}_n \rightarrow \mathbb{E}\left[\mathrm{X} \mid \mathcal{F}_{\infty}\right]$ P-n.o. og i 1-middel.
    \item[\textnormal{(ii)}]Hvis X yderligere er $\mathcal{F}_{\infty}$-målelig, så gælder der, at $\mathrm{X}_n \rightarrow \mathrm{X}$ P-n.o. og i 1-middel.
    \item[\textnormal{(iii)}] For enhver stoppetid $\tau$ med hensyn til $\left(\mathcal{F}_n\right)$ gælder der, at

    $$
    \mathbb{E}\left[\mathrm{X} \mid \mathcal{F}_\tau\right]=\mathrm{X}_\tau
    $$
\end{enumerate}
    \end{proposition}
\end{theorem-box}

\subsection{Optional Sampling (anden version)}
\begin{theorem-box}
    \begin{definition}
        Lad $\Omega, \mathcal{F}, \mathcal{F}_n, P$ være et filtreret sandsynlighedsfelt, og betragt enmm tilpasset følge $(Y_n)_{n\geq 0}$ af stokastiske variable herpå. Lad viodere $\tau$ være en stoppetid mht. $(\mathcal{F_n})$. Vi siger da, at $\tau$ er optional for $(Y_n)$, hvis familien $\{Y_{\tau \wedge n|n\in\N_0}\}$ er uniformt integrabel.
    \end{definition}
\end{theorem-box}
\begin{remark}
Bemærkninger om optionalitet
Lad $\left(\Omega, \mathcal{F}, \mathcal{F}_n, P\right)$ være et filtreret sandsynlighedsfelt, og lad $\left(\mathrm{Y}_n\right)_{n \in \mathrm{~N}_0}$ være en tilpasset følge af stokastiske variable herpå.
(1) Antag, at $\sigma$ og $\tau$ er to stoppetider, som begge er optionale for $\left(\mathrm{Y}_n\right)_{n \in \mathbb{N}_0}$.

Da er $\sigma \wedge \tau$ og $\sigma \vee \tau$ igen optionale for $\left(\mathrm{Y}_n\right)_{n \in \mathrm{~N}_0}$.
Vi har nemlig for ethvert $n$ i $\mathrm{N}_0$, at

$$
\left|\mathrm{Y}_{(\sigma \wedge \tau) \wedge n}\right|=\left|\mathrm{Y}_{\sigma \wedge n}\right| 1_{\{\sigma \leq \tau\}}+\left|\mathrm{Y}_{\tau \wedge n}\right| 1_{\{\sigma>\tau\}} \leq\left|\mathrm{Y}_{\sigma \wedge n}\right|+\left|\mathrm{Y}_{\tau \wedge n}\right|,
$$

og

$$
\left|\mathrm{Y}_{(\sigma \vee \tau) \wedge n}\right|=\left|\mathrm{Y}_{\tau \wedge n}\right| 1_{\{\sigma \leq \tau\}}+\left|\mathrm{Y}_{\sigma \wedge n}\right| 1_{\{\sigma>\tau\}} \leq\left|\mathrm{Y}_{\tau \wedge n}\right|+\left|\mathrm{Y}_{\sigma \wedge n}\right|,
$$

hvor $\left\{\left|\mathrm{Y}_{\sigma \wedge n}\right| \mid n \in \mathbb{N}_0\right\}+\left\{\left|\mathrm{Y}_{\tau \wedge n}\right| \mid n \in \mathbb{N}_0\right\}$ er UI (jvf. 3.1.5 og 3.1.3).
\\(2) Hvis $\tau$ optional for $\left(\mathrm{Y}_n\right)_{n \in \mathrm{~N}_0}$, gælder der specielt, at $\mathrm{Y}_\tau \in \mathcal{L}^1(P)$. Der gælder nemlig

$$
\left(\mathrm{Y}_{\tau \wedge n}\right)_{n \in \mathbb{N}_0} \text { er UI } \stackrel{\substack{3.3.2 \\ \downarrow}}{\Longrightarrow} \sup _{n \in \mathbb{N}_0} \mathrm{E}\left[\left|\mathrm{Y}_{\tau \wedge n}\right|\right]<\infty \stackrel{9.3.13}{\Longrightarrow} \mathrm{Y}_\tau \in \mathcal{L}^1(P) .
$$

\end{remark}
\begin{theorem-box}
    \begin{lemma}[Kriterier for optionalitet]
        Lad $\left(\Omega, \mathcal{F}, \mathcal{F}_n, P\right)$ være et filtreret sandsynlighedsfelt, lad $\left(\mathrm{Y}_n\right)_{n \in \mathbb{N}_0}$ være en tilpasset følge af stokastiske variable herpå, og lad $\tau$ være en stoppetid mht. $\left(\mathcal{F}_n\right)$.

Da er følgende betingelser hver især tilstrækkelige for, at $\tau$ er optional for $\left(Y_n\right)_{n \in \mathbb{N}_0}$ :
\begin{enumerate}
    \item Der findes en stokastisk variabel Y i $\mathcal{L}^1(P)$, således at

$$
\sup _{n \in \mathbb{N}_0}\left|\mathrm{Y}_{\tau \wedge n}\right| \leq|\mathrm{Y}| P \text {-n.o. }
$$
[Jvf. Lemma 3.1.3(ii).]
\item  Der findes $\alpha$ i $(1, \infty)$, således at $\sup _{n \in \mathbb{N}_0} \mathbb{E}\left[\left|\mathrm{Y}_{\tau \wedge n}\right|^\alpha\right]<\infty$.
[Jvf. Eksempel 3.1.6.]
\item $\mathrm{Y}_{\tau \wedge n} \in \mathcal{L}^1(P)$ for alle $n$, og der findes $Z$ i $\mathcal{L}^1(P)$, så $\mathrm{Y}_{\tau \wedge n} \rightarrow Z$ i 1 -middel for $n \rightarrow \infty$.
[Jvf. Sætning 3.2.1.]
\end{enumerate}
    \end{lemma}
\end{theorem-box}
\begin{example}
    (A) Betingelse (a) er specielt opfyldt, hvis $\mathrm{Y}_n \in \mathcal{L}^1(P)$ for alle $n$, og der findes en konstant $M$ i $\mathbb{N}$, saledes at $\tau \leq M P$-n.o.

For ethvert $n$ i $\mathbb{N}_0$ har vi nemlig da, at

$$
\begin{aligned}
\left|\mathrm{Y}_{\tau \wedge n}\right| & \stackrel{\mathrm{m.o.o} .}{=} \sum_{k=0}^M\left|\mathrm{Y}_{\tau \wedge n}\right| 1_{\{\tau \wedge n=k\}} \\
& =\sum_{k=0}^M\left|\mathrm{Y}_k\right| 1_{\{\tau \wedge n=k\}} \leq \sum_{k=0}^M\left|\mathrm{Y}_k\right| \in \mathcal{L}^1(P)
\end{aligned}
$$
\\(B) Betragt reelle tal $a<b$ og stoppetiden (jvf. 9.3.3)

$$
\tau_{a, b}=\inf \left\{n \in \mathbb{N}_0 \mid \mathrm{Y}_n \notin(a, b)\right\} .
$$


Antag, at $\mathrm{Y}_0 \in \mathcal{L}^1(P)$, og at der findes en konstant $M \mathrm{i}(0, \infty)$, således at $\left|\mathrm{Y}_n-\mathrm{Y}_{n-1}\right| \leq M P$-n.o. for alle $n$ i $\mathbb{N}$.

Da er $\tau_{a, b}$ optional for $\left(\mathrm{Y}_n\right)_{n \in \mathbb{N}_0}$. For alle $n$ i $\mathbb{N}_0$ har vi nemlig, at

$$
\begin{aligned}
& \left|\mathrm{Y}_{\tau_{a, b} \wedge n}\right|=\left|\mathrm{Y}_{\tau_{a, b}}\right| 1_{\left\{\tau_{a, b} \leq n\right\}}+\left|\mathrm{Y}_n\right| 1_{\left\{\tau_{a, b}>n\right\}} \\
& =\left|\mathrm{Y}_0\right| 1_{\left\{\tau_{a, b}=0\right\}}+\left|\mathrm{Y}_{\tau_{a, b}-1}+\left(\mathrm{Y}_{\tau_{a, b}}-\mathrm{Y}_{\tau_{a, b}-1}\right)\right| 1_{\left\{1 \leq \tau_{a, b} \leq n\right\}} \\
& \quad+\left|\mathrm{Y}_n\right| 1_{\left\{\tau_{a, b}>n\right\}} \\
& \stackrel{\text { n.o. }}{\leq}\left|\mathrm{Y}_0\right|+((|a| \vee|b|)+M) 1_{\left\{1 \leq \tau_{a, b} \leq n\right\}}+(|a| \vee|b|) 1_{\left\{\tau_{a, b}>n\right\}} \\
& =\left|\mathrm{Y}_0\right|+2(|a| \vee|b|)+M \in \mathcal{L}^1(P)
\end{aligned}
$$

\end{example}
\begin{theorem-box}
    \begin{proposition}
        Lad $\left(\Omega, \mathcal{F}, \mathcal{F}_n, P\right)$ være et filtreret sandsynlighedsfelt, lad $\left(\mathrm{Y}_n\right)_{n \in \mathbb{N}_0}$ være en tilpasset følge af stokastiske variable herpå, og lad $\tau$ være en endelig stoppetid mht. $\left(\mathcal{F}_n\right)$.

Da er $\tau$ optional for $\left(\mathrm{Y}_n\right)_{n \in \mathbb{N}_0}$, hvis og kun hvis følgende 3 betingelser alle er opfyldte:
\begin{enumerate}
    \item $\mathbb{E}\left[\left|Y_\tau\right|\right]<\infty$,
    \item $\int_{\{\tau>n\}}\left|\mathrm{Y}_n\right| \mathrm{d} P<\infty \quad$ for alle $n$ i $\mathbb{N}_0$,
    \item $\lim _{n \rightarrow \infty} \int_{\{\tau>n\}}\left|\mathrm{Y}_n\right| \mathrm{d} P=0$.
\end{enumerate}
    \end{proposition}
\end{theorem-box}
\begin{theorem-box}
    \begin{proposition}
        Lad $\left(\Omega, \mathcal{F}, \mathcal{F}_n, P\right)$ være et filtreret sandsynlighedsfelt, lad $\left(\mathrm{Y}_n\right)_{n \in \mathbb{N}_0}$ være en tilpasset følge af stokastiske variable herpå, og lad $\sigma, \tau$ være stoppetider $m h t .\left(\mathcal{F}_n\right)_{n \in \mathbb{N}_0}$.

Antag, at $\sigma \leq \tau$, og at $\tau$ er optional for $\left(\mathrm{Y}_n\right)_{n \in \mathbb{N}_0}$.
Da er $\sigma$ optional for $\left(\mathrm{Y}_n\right)_{n \in \mathbb{N}_0}$, hvis og kun hvis $\mathbb{E}\left[\left|\mathrm{Y}_\sigma\right|\right]<\infty$.
    \end{proposition}
\end{theorem-box}
\begin{theorem-box}
    \begin{proposition}
        Lad $\left(\Omega, \mathcal{F}, \mathcal{F}_n, P\right)$ være et filtreret sandsynlighedsfelt, og lad $\left(\mathrm{X}_n\right)_{n \in \mathbb{N}_0}$ være en tilpasset følge af stokastiske variable herpå.
Lad videre $\sigma$ og $\tau$ være stoppetider mht. $\left(\mathcal{F}_n\right)_{n \in \mathbb{N}_0}$, og antag, at $\sigma \leq \tau$.
\begin{enumerate}
    \item[\textnormal{(i)}] Antag, at $\left(\mathrm{X}_n, \mathcal{F}_n\right)$ er en sub-MG, og at $\tau$ er optional for $\left(\mathrm{X}_n^{+}\right)_{n \in \mathbb{N}_0}$. Da gælder der, at $\mathrm{X}_\sigma, \mathrm{X}_\tau \in \mathcal{L}^1(P)$, og at

$$
\mathbb{E}\left[\mathrm{X}_0\right] \leq \mathbb{E}\left[\mathrm{X}_\sigma\right] \leq \mathbb{E}\left[\mathrm{X}_\tau\right], \quad \text { samt } \quad \mathrm{X}_\sigma \leq \mathbb{E}\left[\mathrm{X}_\tau \mid \mathcal{F}_\sigma\right] \quad \text { P-n.o. }
$$
    \item[\textnormal{(ii)}] Antag, at $\left(\mathrm{X}_n, \mathcal{F}_n\right)$ er en super-MG, og at $\tau$ er optional for $\left(\mathrm{X}_n^{-}\right)_{n \in \mathbb{N}_0}$. Da gælder der, at $\mathrm{X}_\sigma, \mathrm{X}_\tau \in \mathcal{L}^1(P)$, og at

$$
\mathbb{E}\left[\mathrm{X}_0\right] \geq \mathbb{E}\left[\mathrm{X}_\sigma\right] \geq \mathbb{E}\left[\mathrm{X}_\tau\right], \quad \text { samt } \quad \mathrm{X}_\sigma \geq \mathbb{E}\left[\mathrm{X}_\tau \mid \mathcal{F}_\sigma\right] \quad \text { P-n.o. }
$$
    \item[\textnormal{(iii)}] Antag, at $\left(\mathrm{X}_n, \mathcal{F}_n\right)$ er en MG, og at $\tau$ er optional for $\left(\mathrm{X}_n\right)_{n \in \mathbb{N}_0}$. Da gælder der, at $\mathrm{X}_\sigma, \mathrm{X}_\tau \in \mathcal{L}^1(P)$, og at

$$
\mathbb{E}\left[\mathrm{X}_0\right]=\mathbb{E}\left[\mathrm{X}_\sigma\right]=\mathbb{E}\left[\mathrm{X}_\tau\right], \quad \text { samt } \quad \mathrm{X}_\sigma=\mathbb{E}\left[\mathrm{X}_\tau \mid \mathcal{F}_\sigma\right] \quad \text { P-n.o. }
$$
\end{enumerate}
    \end{proposition}
\end{theorem-box}
\begin{theorem-box}
    \begin{corollary}
        Lad $\left(\Omega, \mathcal{F}, \mathcal{F}_n, P\right)$ være et filtreret sandsynlighedsfelt, og lad $\left(\mathrm{X}_n, \mathcal{F}_n\right)_{n \in \mathbb{N}_0}$ være en ikke-negativ super-martingal.
\begin{enumerate}
    \item[\textnormal{(i)}] For enhver stoppetid $\tau$ mht. $\left(\mathcal{F}_n\right)_{n \in \mathbb{N}_0}$ gælder der, at $\mathrm{X}_\tau \in \mathcal{L}^1(P)$.
    \item[\textnormal{(ii)}] Hvis $\sigma, \tau$ er stoppetider mht. $\left(\mathcal{F}_n\right)_{n \in \mathbb{N}_0}$, og $\sigma \leq \tau$, gælder der, at

$$
\mathbb{E}\left[\mathrm{x}_0\right] \geq \mathbb{E}\left[\mathrm{x}_\sigma\right] \geq \mathbb{E}\left[\mathrm{x}_\tau\right] \geq \mathbb{E}\left[\mathrm{x}_{\infty}\right], \quad \text { og } \quad \mathrm{x}_\sigma \geq \mathbb{E}\left[\mathrm{x}_\tau \mid \mathcal{F}_\sigma\right] \text { P-n.o. }
$$
    \item[\textnormal{(iii)}] For et vilkårligt tal $\ell i(0, \infty)$ gælder der, at

$$
\ell P\left(\sup _{n \in \mathbb{N}_0} \mathrm{x}_n>\ell\right) \leq \mathbb{E}\left[\mathrm{x}_0\right]
$$
\end{enumerate}
    \end{corollary}
\end{theorem-box}
\begin{theorem-box}
    \begin{corollary}
        Lad $\left(\Omega, \mathcal{F}, \mathcal{F}_n, P\right)$ være et filtreret sandsynlighedsfelt, lad $\left(\mathrm{X}_n\right)_{n \in \mathbb{N}_0}$ være en tilpasset følge af stokastiske variable herpå, og lad $\sigma, \tau$ være stoppetider mht. $\left(\mathcal{F}_n\right)_{n \in \mathbb{N}_0}$.
        \begin{enumerate}
            \item[\textnormal{(i)}] Hvis $\left(\mathrm{X}_n, \mathcal{F}_n\right)_{n \in \mathbb{N}_0}$ er en sub-MG, og $\tau$ er optional for $\left(\mathrm{X}_n^{+}\right)_{n \in \mathbb{N}_0}$, så gælder der, at

$$
\mathrm{X}_{\sigma \wedge \tau} \leq \mathbb{E}\left[\mathrm{X}_\tau \mid \mathcal{F}_\sigma\right] \quad \text { P-n.o. }
$$
            \item[\textnormal{(ii)}] Hvis $\left(\mathrm{X}_n, \mathcal{F}_n\right)_{n \in \mathbb{N}_0}$ er en super-MG, og $\tau$ er optional for $\left(\mathrm{X}_n^{-}\right)_{n \in \mathbb{N}_0}$, så gælder der, at

$$
\mathrm{X}_{\sigma \wedge \tau} \geq \mathbb{E}\left[\mathrm{X}_\tau \mid \mathcal{F}_\sigma\right] \quad \text { P-n.o. }
$$

            \item[\textnormal{(iii)}] Hvis $\left(\mathrm{X}_n, \mathcal{F}_n\right)_{n \in \mathbb{N}_0}$ er en MG, og $\tau$ er optional for $\left(\mathrm{X}_n\right)_{n \in \mathbb{N}_0}$, så gælder der, at

$$
\mathrm{X}_{\sigma \wedge \tau}=\mathbb{E}\left[\mathrm{X}_\tau \mid \mathcal{F}_\sigma\right] \quad \text { P-n.o. }
$$

        \end{enumerate}
    \end{corollary}
\end{theorem-box}
\begin{theorem-box}
    \begin{corollary}
        Lad $\left(\Omega, \mathcal{F}, \mathcal{F}_n, P\right)$ være et filtreret sandsynlighedsfelt, lad $\left(\mathrm{X}_n\right)_{n \in \mathbb{N}_0}$ være en tilpasset følge af stokastiske variable herpå, og lad $\sigma, \tau$ være stoppetider mht. $\left(\mathcal{F}_n\right)_{n \in \mathbb{N}_0}$, således at $\sigma \leq \tau$.
        \begin{enumerate}
            \item[\textnormal{(i)}] Hvis $\left(\mathrm{X}_n, \mathcal{F}_n\right)_{n \in \mathbb{N}_0}$ er en sub-MG, og $\left\{\mathrm{X}_n^{+} \mid n \in \mathbb{N}_0\right\}$ er UI, da gælder der, at $\mathrm{X}_\sigma, \mathrm{X}_\tau \in \mathcal{L}^1(P)$, og at
$\mathbb{E}\left[\mathrm{X}_0\right] \leq \mathbb{E}\left[\mathrm{X}_\sigma\right] \leq \mathbb{E}\left[\mathrm{X}_\tau\right] \leq \mathbb{E}\left[\mathrm{X}_{\infty}\right], \quad$ samt at $\quad \mathrm{X}_\sigma \leq \mathbb{E}\left[\mathrm{X}_\tau \mid \mathcal{F}_\sigma\right] \quad$ P-n.o.
            \item[\textnormal{(ii)}] Hvis $\left(\mathrm{X}_n, \mathcal{F}_n\right)_{n \in \mathbb{N}_0}$ er en super-MG, og $\left\{\mathrm{X}_n^{-} \mid n \in \mathbb{N}_0\right\}$ er Ul, da gælder der, at $\mathrm{X}_\sigma, \mathrm{X}_\tau \in \mathcal{L}^1(P)$, og at
$\mathbb{E}\left[\mathrm{X}_0\right] \geq \mathbb{E}\left[\mathrm{X}_\sigma\right] \geq \mathbb{E}\left[\mathrm{X}_\tau\right] \geq \mathbb{E}\left[\mathrm{X}_{\infty}\right], \quad$ samt at $\quad \mathrm{X}_\sigma \geq \mathbb{E}\left[\mathrm{X}_\tau \mid \mathcal{F}_\sigma\right] \quad$ P-n.o.
            \item[\textnormal{(iii)}] Hvis $\left(\mathrm{X}_n, \mathcal{F}_n\right)_{n \in \mathbb{N}_0}$ er en $M G$, og $\left\{\mathrm{X}_n \mid n \in \mathbb{N}_0\right\}$ er UI, da gælder der, at $\mathrm{X}_\sigma, \mathrm{X}_\tau \in \mathcal{L}^1(P)$, og at
$\mathbb{E}\left[\mathrm{X}_0\right]=\mathbb{E}\left[\mathrm{X}_\sigma\right]=\mathbb{E}\left[\mathrm{X}_\tau\right]=\mathbb{E}\left[\mathrm{X}_{\infty}\right], \quad$ samt at $\quad \mathrm{X}_\sigma=\mathbb{E}\left[\mathrm{X}_\tau \mid \mathcal{F}_\sigma\right] \quad$ P-n.o.
        \end{enumerate}
    \end{corollary}
\end{theorem-box}
\end{document}
