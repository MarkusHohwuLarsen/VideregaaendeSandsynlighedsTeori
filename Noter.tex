\documentclass{article}
\usepackage[utf8]{inputenc}
\usepackage{graphicx}
\usepackage{listings}
\usepackage{color}
\usepackage{float}
\usepackage{amsmath} 
\usepackage{hyperref}
% \usepackage{calrsfs}
\usepackage{mathrsfs}
\usepackage{aligned-overset}
\usepackage{amssymb}
\usepackage{mathtools}
\usepackage[mathscr]{euscript}
\usepackage{tikz}
\usepackage{bbm}
\usepackage[most]{tcolorbox}
\usepackage{booktabs}
\usepackage{amsthm}
\newcommand{\N}{\mathbb{N}}
\newcommand{\Q}{\mathbb{Q}}
\newcommand{\R}{\mathbb{R}}
\newcommand{\C}{\mathbb{C}}
\newcommand{\F}{\mathbb{F}}
\newcommand{\E}{\mathbb{E}}
\newcommand{\1}{\mathbbm{1}}
\newcommand{\X}{\mathsf{X}}
\newcommand{\Y}{\mathsf{Y}}
\newcommand{\B}{\mathcal{B}}
\newcommand{\lclass}{\mathcal{L}}
\newcommand{\Prob}{\mathbb{P}}
\newcommand{\deriv}{\operatorname{d}}
\newcommand{\icomp}{\operatorname{i}}
\newcommand{\varx}{\varphi_\X}
\newcommand{\pfield}{(\Omega, \mathcal{F}, P)}
\newtheorem{theorem}{Theorem}[subsection]
\newtheorem{definition}[theorem]{Definition} 
\newtheorem{lemma}[theorem]{Lemma} 
\newtheorem{corollary}[theorem]{Korollar} 
\newtheorem{remark}[theorem]{Bemærkning} 
\newtheorem{proposition}[theorem]{Sætning} 
\newtheorem{example}[theorem]{Eksempel} 

\usepackage{geometry}
    \geometry{
        a4paper,
        left=3.5cm,
        right=3.5cm   ,
    }
\definecolor{dkgreen}{rgb}{0,0.6,0}
\definecolor{gray}{rgb}{0.5,0.5,0.5}
\definecolor{mauve}{rgb}{0.58,0,0.82}
\newtcolorbox{theorem-box}{
    colback=gray!10, % Light grey background
    colframe=black,  % Black frame
    sharp corners,   % Square corners
    boxrule=0.8pt,   % Border thickness
    before skip=10pt, % Space before the box
    after skip=10pt,  % Space after the box
}

\newtheoremstyle{boxed}  % Define a new theorem style
  {10pt}   % Space above
  {10pt}   % Space below
  {}       % Body font
  {}       % Indent amount
  {\bfseries} % Theorem head font (bold)
  {.}      % Punctuation after theorem head
  { }      % Space after theorem head
  {\thmname{#1}~\thmnumber{#2}\thmnote{ (#3)}}  % Theorem head spec

\theoremstyle{boxed}
\lstset{frame=tb,
  language=Python,
  aboveskip=3mm,
  belowskip=3mm,
  showstringspaces=false,
  columns=flexible,
  basicstyle={\small\ttfamily},
  numbers=none,
  numberstyle=\tiny\color{gray},
  keywordstyle=\color{blue},
  commentstyle=\color{dkgreen},
  stringstyle=\color{mauve},
  breaklines=true,
  breakatwhitespace=true,
  tabsize=3
}
\usepackage{graphicx} % Required for inserting images
\renewcommand{\thesubsection}{\thesection.\arabic{subsection}}

\title{Noter - Videregående sandsynlighedsteori}
\begin{document}

\section{Fourier-transformation og karakteristiske funktioner}
\subsection{Definition og indledende bemærkninger}
\begin{theorem-box}
\begin{definition}
    Lad $\mu$ være et sandsynlighedsmål på $(\R^d, \mathcal{B}(\R^d))$. Den Fourier-transofrmerede af $\mu$ er funktionen $\hat{\mu}:\R^d\rightarrow \mathbb{C}$ givet ved 
    $$\hat{\mu}(t)=\int_{\R^d}e^{\operatorname{i}\langle t,x\rangle}\mu(\operatorname{d}x)=\int_{\R^d}\cos(\langle t,x\rangle)\mu \operatorname{d}x+\operatorname{i}\int_{\R^d}\sin(\langle t,x\rangle)\mu \operatorname{d}x$$
    for ethvert $t$ i $\R^d$. I tilfældet $d=1$ ser vi specielt, at 
    $$\hat{\mu}(t)=\int_{\R}e^{\operatorname{i}tx}\mu(\operatorname{d}x)=\int_{\R}\cos(tx)\mu \operatorname{d}x+\operatorname{i}\int_{\R}\sin(tx)\mu \operatorname{d}x$$
    for ethvert $t$ i $\R$
\end{definition}
\end{theorem-box}

    \begin{remark}
        Antag, at $\mu$ er et sandsynlighedsmål på $(\R, \B(\R))$ med tæthed $f$ fra $\lclass^1(\lambda)^+$ med hensyn til $\lambda$. Det følger da for ethvert $t$ i $\R$, at 
        $$\hat{\mu}(t)=\int_{\R}e^{\operatorname{i}tx}\mu(\operatorname{d}x)=\int_{\R}e^{\operatorname{i}tx}f(x)\lambda(\operatorname{d}x)=\sqrt{2\pi}\hat{f}(-t),$$
        hvor $\hat{f}$ betegner den Fourier-transformerede af $f$ (jvf. Definition 12.1.1 i [M\&I])
    \end{remark}
\begin{example}[Den Fourier-transformerede af normalfordelingen]
    Vi har
    $$\widehat{N(\xi, \sigma^2)}(t)=e^{\operatorname{i}t\xi}e^{-\sigma^2t^2/2}$$
    for ethvert $t$ i $\R$
\end{example}
\begin{theorem-box}
    \begin{proposition}
        Lad $\mu$ og $\nu$ være ssh.-mål på $(\R^d, \B(\R^d))$ hhv. $(\R^m, \B(\R^m))$. Da gælder følgende udsagn:
        \begin{enumerate}
            \item[(i)] $|\hat{\mu}(t)|\leq \hat{\mu}(0) = 1 $ for alle $t$ i $\R^d$
            \item[(ii)] $\hat{\mu}:\R^d \rightarrow \C$ er en kontinuert funktion.
            \item[(iii)] $\hat{\mu}(-t)=\overline{\hat{\mu}(t)}$ for alle $t$ i $\R^d$
            \item[(iv)] Hvis $d=m$ gælder der, at 
            $$\int_{\R^d}\hat{\mu}(t)\nu(\deriv t)=\int_{\R^d}\hat{\nu}(t)\mu(\deriv t)$$ 
            \item[(v)]$\widehat{\mu\otimes\nu}(t,s)=\hat{\mu}(t)\hat{\nu}(s)$ for alle $(t,s)$ i $\R^d\times\R^m=\R^{d+m}$
            \item[(vi)] $\widehat{\mu * \nu}(t,s)=\hat{\mu}(t)\cdot\hat{\nu}(t)$ for alle $t$ i $\R^d$
        \end{enumerate}
    \end{proposition}
\end{theorem-box}

\begin{theorem-box}
    \begin{definition} 
        Lad $\mathsf{X}$ være en d-dimensionel stokastisk vektor defineret op sandsynlighedsfeltet $\pfield$. Den karakteristiske funktion for $\mathsf{X}$ er funktionen $\varphi_{\mathsf{X}}:\R^d\rightarrow \C$ givet ved 
        $$\varphi_\mathsf{X}=\hat{P_\X}, \quad \text{hvor }P_\X=P \circ \X^{-1}$$
        For ethvert $t$ i $\R^d$ hat vi altså, at
        $$\varphi_\X(t)=\int_{\R^d}e^{\icomp \langle t,x\rangle}P_\X(\deriv x) = \int_{\Omega}e^{\icomp \langle t, \X(\omega)\rangle}P(\deriv\omega)=\E[e^{\icomp \langle t,x\rangle}]$$
    \end{definition}
\end{theorem-box}
\begin{example}
    Hvis $\X$ er normalfordelt, så har vi
    $$\varphi_\X(t)=\widehat{N(\xi, \sigma^2)}(t)=e^{\icomp t\xi}e^{-\sigma^2t^2/2}$$
    for alle t i $\R$
\end{example}
\begin{theorem-box}
    \begin{corollary}[Egenskaber ved den karakteristiske funktion]
        Lad $\X$ og $\Y$ være hhv. d- og m-dimensionale stokastiske vektorere definerede på $\pfield$. Da gælder følgende udsagn:
        \begin{enumerate}
            \item[(i)] $|\varphi_\X(t)|\leq \varphi_\X(0)=1$ for alle t i $\R^d$
            \item[(ii)] Funktionen $\varphi_\X:\R^d\rightarrow \C$ er kontinuert.
            \item[(iii)] $\varphi_\X(-t)=\varphi_{-\X}(t)=\overline{\varx(t)}$ for alle t i $\R^d$
            \item[(iv)] For enhver $m\times n$ matrix A og enhver vektor b i $\R^m$ gælder formlen:
            $$\varphi_{A\X+b}(s)=e^{\icomp\langle s,b\rangle}\varx (A^Ts), \quad (s\in\R^m)$$
            \item[(v)] Hvis $d=m$, gælder formlen: $\E[\varphi_\Y(\X)]=\E[\varx(\Y)]$
            \item[(vi)] Hvis $\X$ og $\Y$ er uafhængige, gælder formlen: 
            $$\varphi_{(\X,\Y)}(t,s)=\varx(t)\varphi_\Y(s), \quad (t\in\R^d,s\in\R^m)$$ 
            \item[(vii)] Hvis $d=m$, og $\X$ og $\Y$ er uafhængige, gælder formlen:
            $$\varphi_{\X+\Y}(t)=\varx(t)\varphi_\Y(t)=\varphi_{(\X,\Y)}(t,t), \quad (t\in\R^d)$$
        \end{enumerate}
    \end{corollary}
\end{theorem-box}
\subsection{Entydighed og Inversionsætningen for karakteristiske funktioner}
\begin{theorem-box}
    \begin{lemma}
        Lad $\mu$ og $\nu$ være to mål på $(\R^d, \B(\R^d))$ og antag at $\mu((-n,n)^d)<\infty$ for alle n i $\N$, samt at
        $$\int_{\R^d}\psi \deriv \mu =\int_{\R^d}\psi \deriv \nu\quad \text{for alle }\psi \text{ i } C_c(\R^d,\R)^+$$
        Da gælder der, at $\mu=\nu$
    \end{lemma}
\end{theorem-box}
\begin{theorem-box}
    \begin{lemma}
        Lad $\X$ og $\Y$ være uafhængiged d-dimensionale stokastiske vektorer definerede på sandsynlighedsfeltet $\pfield$, og antag, at $\X$ er absolut kontinuert med tæthed $f_\X$ (med hensyn til $\lambda_d$).

        Da er $\X+\Y$ ligeledes absolut kontinuert med $\lambda_d$-tæthed $f_{\X+\Y}$ givet ved
        $$f_{\X+\Y}(z)=\int_{\R^d}f_\X(z-y)P_\Y(\deriv y), \quad (z\in\R^d)$$
    \end{lemma}
\end{theorem-box}

\begin{theorem-box}
    \begin{lemma}
        Lad $\X$ og $\mathsf{U}$ være uafhængige d-dimensionale stokastiske vektorer på sandsynlighedsfeltet $\pfield$, og antag, at $\mathsf{U}=(\mathsf{U}_1, \dots, \mathsf{U}_d)$, hvor $\mathsf{U}_1, \dots, \mathsf{U}_d$ er uafhængige identisk $N(0,1)$-fordelte stokastiske variable. 
        
        For ethvert $\sigma$ i $(0,\infty)$ gælder der da, at $\X+\sigma\mathsf{U}$ er absolut kontinuert med tæthed $f_{\X+\sigma\mathsf{U}}$ givet ved:
        $$f_{\X+\sigma\mathsf{U}}(t)=(2\pi)^{-d}\int_{\R^d}e^{-\frac{1}{2}\sigma^2\|s\|^2}e^{-\icomp \langle t,s\rangle}\varx(s)\lambda_d(\deriv s), \quad (t\in\R^d),$$
        hvor $\varx$ er den karakteristiske funktion for $\X$
    \end{lemma}
\end{theorem-box}
\begin{theorem-box}
    \begin{lemma}
        Lad $\X$ og $\mathsf{U}$ være d-dimensionale stokastiske vektorer defineret på $\pfield$, og betragt for ethvert n i $\N$ den stokastiske vektor $\X+\frac{1}{n}\mathsf{U}$.

        For enhver funktion $\psi$ fra $C_b(\R^d,\C)$ gælder der da, at 
        $$\int_{\R^d}\psi(t)P_{\X+\frac{1}{n}\mathsf{U}}(\deriv t)\xrightarrow[n\rightarrow \infty]{} \int_{\R^d}\psi(t)P_\X(\deriv t)$$
    \end{lemma}
\end{theorem-box}
\begin{theorem-box}
    \begin{proposition}
        \begin{enumerate}
            \item[(i)] Lad $\X$ og $\Y$ være d-dimensionale stokastiske vektorer. Da gælder implikationen $$\varx=\varphi_\Y\Longrightarrow \X\sim\Y$$
            \item[(ii)] Lad $\mu$ og $\nu$ være sandsynlighedsmål på $(\R^d, \B(\R^d))$. Da gælder implikationen:
            $$\hat{\mu}=\hat{\nu}\Longrightarrow \mu = \nu$$ 
        \end{enumerate}
    \end{proposition}
\end{theorem-box}
\begin{remark}
    $\X$ og $\Y$ behøver ikke at være defineret på samme sandsynlighedsfelt. Der kan derfor findes stokastiske variable $\tilde{\X}, \tilde{\Y}$ således at $\tilde{\X}\sim \X$ og $\tilde{\Y}\sim \Y$ og ræssonere:$$\varx = \varphi_\Y \Longleftrightarrow \varphi_{\tilde{\X}}=\varphi_{\tilde{\Y}}\Longrightarrow \tilde{\X}\sim \tilde{\Y} \Longrightarrow \X \sim \Y$$
\end{remark}
\begin{theorem-box}
    \begin{corollary}
        Lad $\X$ og $\Y$ være hhv. d- og m-dimensionale stokastiske vektorer definerede på sandsynlighedsfeltet $\pfield$. Da er $\X$ og $\Y$ uafhængige, hvis og kun hvis der gælder, at $$\varphi_{(\X,\Y)}(t,s)=\varx(t)\varphi_\Y(s)\quad \text{for alle }t\text{ i }\R^d, \text{og }s\text{ i }\R^m$$
    \end{corollary}
\end{theorem-box}
\begin{theorem-box}
    \begin{proposition}
        Lad $\X$ være en d-dimensional stokastisk vektor på sandsynlighedsfeltet $\pfield$, og antag, at dens karakteristiske funktion $\varx$ er element i $\lclass^1_{\C}(\lambda_d)$. Da er $P_\X$ absolut kontinuert med tæthed $f_\X$ givet ved:
        $$f_\X(t)=(2\pi)^{-d}\int_{\R^d}e^{\icomp \langle t,s\rangle}\varx(s)\lambda_d(\deriv s), \quad (t\in\R^d)$$
    \end{proposition}
\end{theorem-box}
\end{document}