\documentclass{article}
\usepackage[utf8]{inputenc}
\usepackage{graphicx}
\usepackage{listings}
\usepackage{color}
\usepackage{float}
\usepackage{amsmath} 
\usepackage{hyperref}
% \usepackage{calrsfs}
\usepackage{mathrsfs}
\usepackage{aligned-overset}
\usepackage{amssymb}
\usepackage{mathtools}
\usepackage[mathscr]{euscript}
\usepackage{tikz}
\usepackage{bbm}
\usepackage[most]{tcolorbox}
\usepackage{booktabs}
\usepackage{amsthm}
\newcommand{\N}{\mathbb{N}}
\newcommand{\Q}{\mathbb{Q}}
\newcommand{\R}{\mathbb{R}}
\newcommand{\C}{\mathbb{C}}
\newcommand{\F}{\mathbb{F}}
\newcommand{\E}{\mathbb{E}}
\newcommand{\1}{\mathbbm{1}}
\newcommand{\X}{\mathsf{X}}
\newcommand{\Y}{\mathsf{Y}}
\newcommand{\B}{\mathcal{B}}
\newcommand{\lclass}{\mathcal{L}}
\newcommand{\Prob}{\mathbb{P}}
\newcommand{\deriv}{\operatorname{d}}
\newcommand{\icomp}{\operatorname{i}}
\newcommand{\varx}{\varphi_\X}
\newcommand{\pfield}{(\Omega, \mathcal{F}, P)}
\newtheorem{theorem}{Theorem}[subsection]
\newtheorem{definition}[theorem]{Definition} 
\newtheorem{lemma}[theorem]{Lemma} 
\newtheorem{corollary}[theorem]{Korollar} 
\newtheorem{remark}[theorem]{Bemærkning} 
\newtheorem{proposition}[theorem]{Sætning} 
\newtheorem{example}[theorem]{Eksempel} 
\newtheorem{problemstilling}[theorem]{Problemstilling} 

\usepackage{geometry}
    \geometry{
        a4paper,
        left=3.5cm,
        right=3.5cm   ,
    }
\definecolor{dkgreen}{rgb}{0,0.6,0}
\definecolor{gray}{rgb}{0.5,0.5,0.5}
\definecolor{mauve}{rgb}{0.58,0,0.82}
\newtcolorbox{theorem-box}{
    colback=gray!10, % Light grey background
    colframe=black,  % Black frame
    sharp corners,   % Square corners
    boxrule=0.8pt,   % Border thickness
    before skip=10pt, % Space before the box
    after skip=10pt,  % Space after the box
}

\newtheoremstyle{boxed}  % Define a new theorem style
  {10pt}   % Space above
  {10pt}   % Space below
  {}       % Body font
  {}       % Indent amount
  {\bfseries} % Theorem head font (bold)
  {.}      % Punctuation after theorem head
  { }      % Space after theorem head
  {\thmname{#1}~\thmnumber{#2}\thmnote{ (#3)}}  % Theorem head spec

\theoremstyle{boxed}
\lstset{frame=tb,
  language=Python,
  aboveskip=3mm,
  belowskip=3mm,
  showstringspaces=false,
  columns=flexible,
  basicstyle={\small\ttfamily},
  numbers=none,
  numberstyle=\tiny\color{gray},
  keywordstyle=\color{blue},
  commentstyle=\color{dkgreen},
  stringstyle=\color{mauve},
  breaklines=true,
  breakatwhitespace=true,
  tabsize=3
}
\usepackage{graphicx} % Required for inserting images
\renewcommand{\thesubsection}{\thesection.\arabic{subsection}}

\title{Noter - Videregående sandsynlighedsteori}
\begin{document}

\section{Fourier-transformation og karakteristiske funktioner}
\subsection{Definition og indledende bemærkninger}
\begin{theorem-box}
\begin{definition}
    Lad $\mu$ være et sandsynlighedsmål på $(\R^d, \mathcal{B}(\R^d))$. Den Fourier-transofrmerede af $\mu$ er funktionen $\hat{\mu}:\R^d\rightarrow \mathbb{C}$ givet ved 
    $$\hat{\mu}(t)=\int_{\R^d}e^{\operatorname{i}\langle t,x\rangle}\mu(\operatorname{d}x)=\int_{\R^d}\cos(\langle t,x\rangle)\mu \operatorname{d}x+\operatorname{i}\int_{\R^d}\sin(\langle t,x\rangle)\mu \operatorname{d}x$$
    for ethvert $t$ i $\R^d$. I tilfældet $d=1$ ser vi specielt, at 
    $$\hat{\mu}(t)=\int_{\R}e^{\operatorname{i}tx}\mu(\operatorname{d}x)=\int_{\R}\cos(tx)\mu \operatorname{d}x+\operatorname{i}\int_{\R}\sin(tx)\mu \operatorname{d}x$$
    for ethvert $t$ i $\R$
\end{definition}
\end{theorem-box}

    \begin{remark}
        Antag, at $\mu$ er et sandsynlighedsmål på $(\R, \B(\R))$ med tæthed $f$ fra $\lclass^1(\lambda)^+$ med hensyn til $\lambda$. Det følger da for ethvert $t$ i $\R$, at 
        $$\hat{\mu}(t)=\int_{\R}e^{\operatorname{i}tx}\mu(\operatorname{d}x)=\int_{\R}e^{\operatorname{i}tx}f(x)\lambda(\operatorname{d}x)=\sqrt{2\pi}\hat{f}(-t),$$
        hvor $\hat{f}$ betegner den Fourier-transformerede af $f$ (jvf. Definition 12.1.1 i [M\&I])
    \end{remark}
\begin{example}[Den Fourier-transformerede af normalfordelingen]
    Vi har
    $$\widehat{N(\xi, \sigma^2)}(t)=e^{\operatorname{i}t\xi}e^{-\sigma^2t^2/2}$$
    for ethvert $t$ i $\R$
\end{example}
\begin{theorem-box}
    \begin{proposition}
        Lad $\mu$ og $\nu$ være ssh.-mål på $(\R^d, \B(\R^d))$ hhv. $(\R^m, \B(\R^m))$. Da gælder følgende udsagn:
        \begin{enumerate}
            \item[\textnormal{(i)}] $|\hat{\mu}(t)|\leq \hat{\mu}(0) = 1 $ for alle $t$ i $\R^d$
            \item[\textnormal{(ii)}] $\hat{\mu}:\R^d \rightarrow \C$ er en kontinuert funktion.
            \item[\textnormal{(iii)}] $\hat{\mu}(-t)=\overline{\hat{\mu}(t)}$ for alle $t$ i $\R^d$
            \item[\textnormal{(iv)}] Hvis $d=m$ gælder der, at 
            $$\int_{\R^d}\hat{\mu}(t)\nu(\deriv t)=\int_{\R^d}\hat{\nu}(t)\mu(\deriv t)$$ 
            \item[\textnormal{(v)}]$\widehat{\mu\otimes\nu}(t,s)=\hat{\mu}(t)\hat{\nu}(s)$ for alle $(t,s)$ i $\R^d\times\R^m=\R^{d+m}$
            \item[\textnormal{(vi)}] $\widehat{\mu * \nu}(t,s)=\hat{\mu}(t)\cdot\hat{\nu}(t)$ for alle $t$ i $\R^d$
        \end{enumerate}
    \end{proposition}
\end{theorem-box}

\begin{theorem-box}
    \begin{definition} 
        Lad $\mathsf{X}$ være en d-dimensionel stokastisk vektor defineret op sandsynlighedsfeltet $\pfield$. Den karakteristiske funktion for $\mathsf{X}$ er funktionen $\varphi_{\mathsf{X}}:\R^d\rightarrow \C$ givet ved 
        $$\varphi_\mathsf{X}=\hat{P_\X}, \quad \text{hvor }P_\X=P \circ \X^{-1}$$
        For ethvert $t$ i $\R^d$ hat vi altså, at
        $$\varphi_\X(t)=\int_{\R^d}e^{\icomp \langle t,x\rangle}P_\X(\deriv x) = \int_{\Omega}e^{\icomp \langle t, \X(\omega)\rangle}P(\deriv\omega)=\E[e^{\icomp \langle t,x\rangle}]$$
    \end{definition}
\end{theorem-box}
\begin{example}
    Hvis $\X$ er normalfordelt, så har vi
    $$\varphi_\X(t)=\widehat{N(\xi, \sigma^2)}(t)=e^{\icomp t\xi}e^{-\sigma^2t^2/2}$$
    for alle t i $\R$
\end{example}
\begin{theorem-box}
    \begin{corollary}[Egenskaber ved den karakteristiske funktion]
        Lad $\X$ og $\Y$ være hhv. d- og m-dimensionale stokastiske vektorere definerede på $\pfield$. Da gælder følgende udsagn:
        \begin{enumerate}
            \item[\textnormal{(i)}] $|\varphi_\X(t)|\leq \varphi_\X(0)=1$ for alle t i $\R^d$
            \item[\textnormal{(ii)}] Funktionen $\varphi_\X:\R^d\rightarrow \C$ er kontinuert.
            \item[\textnormal{(iii)}] $\varphi_\X(-t)=\varphi_{-\X}(t)=\overline{\varx(t)}$ for alle t i $\R^d$
            \item[\textnormal{(iv)}] For enhver $m\times n$ matrix A og enhver vektor b i $\R^m$ gælder formlen:
            $$\varphi_{A\X+b}(s)=e^{\icomp\langle s,b\rangle}\varx (A^Ts), \quad (s\in\R^m)$$
            \item[\textnormal{(v)}] Hvis $d=m$, gælder formlen: $\E[\varphi_\Y(\X)]=\E[\varx(\Y)]$
            \item[\textnormal{(vi)}] Hvis $\X$ og $\Y$ er uafhængige, gælder formlen: 
            $$\varphi_{(\X,\Y)}(t,s)=\varx(t)\varphi_\Y(s), \quad (t\in\R^d,s\in\R^m)$$ 
            \item[\textnormal{(vii)}] Hvis $d=m$, og $\X$ og $\Y$ er uafhængige, gælder formlen:
            $$\varphi_{\X+\Y}(t)=\varx(t)\varphi_\Y(t)=\varphi_{(\X,\Y)}(t,t), \quad (t\in\R^d)$$
        \end{enumerate}
    \end{corollary}
\end{theorem-box}
\subsection{Entydighed og Inversionsætningen for karakteristiske funktioner}
\begin{theorem-box}
    \begin{lemma}
        Lad $\mu$ og $\nu$ være to mål på $(\R^d, \B(\R^d))$ og antag at $\mu((-n,n)^d)<\infty$ for alle n i $\N$, samt at
        $$\int_{\R^d}\psi \deriv \mu =\int_{\R^d}\psi \deriv \nu\quad \text{for alle }\psi \text{ i } C_c(\R^d,\R)^+$$
        Da gælder der, at $\mu=\nu$
    \end{lemma}
\end{theorem-box}
Altså at $\psi$ tilhører alle kontinuerte funktioner med kompakt støtte.
\begin{theorem-box}
    \begin{lemma}
        Lad $\X$ og $\Y$ være uafhængiged d-dimensionale stokastiske vektorer definerede på sandsynlighedsfeltet $\pfield$, og antag, at $\X$ er absolut kontinuert med tæthed $f_\X$ (med hensyn til $\lambda_d$).

        Da er $\X+\Y$ ligeledes absolut kontinuert med $\lambda_d$-tæthed $f_{\X+\Y}$ givet ved
        $$f_{\X+\Y}(z)=\int_{\R^d}f_\X(z-y)P_\Y(\deriv y), \quad (z\in\R^d)$$
    \end{lemma}
\end{theorem-box}

\begin{theorem-box}
    \begin{lemma}
        Lad $\X$ og $\mathsf{U}$ være uafhængige d-dimensionale stokastiske vektorer på sandsynlighedsfeltet $\pfield$, og antag, at $\mathsf{U}=(\mathsf{U}_1, \dots, \mathsf{U}_d)$, hvor $\mathsf{U}_1, \dots, \mathsf{U}_d$ er uafhængige identisk $N(0,1)$-fordelte stokastiske variable. 
        
        For ethvert $\sigma$ i $(0,\infty)$ gælder der da, at $\X+\sigma\mathsf{U}$ er absolut kontinuert med tæthed $f_{\X+\sigma\mathsf{U}}$ givet ved:
        $$f_{\X+\sigma\mathsf{U}}(t)=(2\pi)^{-d}\int_{\R^d}e^{-\frac{1}{2}\sigma^2\|s\|^2}e^{-\icomp \langle t,s\rangle}\varx(s)\lambda_d(\deriv s), \quad (t\in\R^d),$$
        hvor $\varx$ er den karakteristiske funktion for $\X$
    \end{lemma}
\end{theorem-box}
\begin{theorem-box}
    \begin{lemma}
        Lad $\X$ og $\mathsf{U}$ være d-dimensionale stokastiske vektorer defineret på $\pfield$, og betragt for ethvert n i $\N$ den stokastiske vektor $\X+\frac{1}{n}\mathsf{U}$.

        For enhver funktion $\psi$ fra $C_b(\R^d,\C)$ gælder der da, at 
        $$\int_{\R^d}\psi(t)P_{\X+\frac{1}{n}\mathsf{U}}(\deriv t)\xrightarrow[n\rightarrow \infty]{} \int_{\R^d}\psi(t)P_\X(\deriv t)$$
    \end{lemma}
\end{theorem-box}
\begin{theorem-box}
    \begin{proposition}
        \begin{enumerate}
            \item[\textnormal{(i)}] Lad $\X$ og $\Y$ være d-dimensionale stokastiske vektorer. Da gælder implikationen $$\varx=\varphi_\Y\Longrightarrow \X\sim\Y$$
            \item[\textnormal{(ii)}] Lad $\mu$ og $\nu$ være sandsynlighedsmål på $(\R^d, \B(\R^d))$. Da gælder implikationen:
            $$\hat{\mu}=\hat{\nu}\Longrightarrow \mu = \nu$$ 
        \end{enumerate}
    \end{proposition}
\end{theorem-box}
\begin{remark}
    $\X$ og $\Y$ behøver ikke at være defineret på samme sandsynlighedsfelt. Der kan derfor findes stokastiske variable $\tilde{\X}, \tilde{\Y}$ således at $\tilde{\X}\sim \X$ og $\tilde{\Y}\sim \Y$ og ræssonere:$$\varx = \varphi_\Y \Longleftrightarrow \varphi_{\tilde{\X}}=\varphi_{\tilde{\Y}}\Longrightarrow \tilde{\X}\sim \tilde{\Y} \Longrightarrow \X \sim \Y$$
\end{remark}
\begin{theorem-box}
    \begin{corollary}
        Lad $\X$ og $\Y$ være hhv. d- og m-dimensionale stokastiske vektorer definerede på sandsynlighedsfeltet $\pfield$. Da er $\X$ og $\Y$ uafhængige, hvis og kun hvis der gælder, at $$\varphi_{(\X,\Y)}(t,s)=\varx(t)\varphi_\Y(s)\quad \text{for alle }t\text{ i }\R^d, \text{og }s\text{ i }\R^m$$
    \end{corollary}
\end{theorem-box}
\begin{theorem-box}
    \begin{proposition}[Inversionssætningen for karakteristiske funktioner]
        Lad $\X$ være en d-dimensional stokastisk vektor på sandsynlighedsfeltet $\pfield$, og antag, at dens karakteristiske funktion $\varx$ er element i $\lclass^1_{\C}(\lambda_d)$. Da er $P_\X$ absolut kontinuert med tæthed $f_\X$ givet ved:
        $$f_\X(t)=(2\pi)^{-d}\int_{\R^d}e^{\icomp \langle t,s\rangle}\varx(s)\lambda_d(\deriv s), \quad (t\in\R^d)$$
    \end{proposition}
\end{theorem-box}
\begin{proof}
    Proof
\end{proof}
\subsection{Differentiabilitet og Taylor-udvikling for karakteristiske funktioner}
\begin{theorem-box}
    \begin{proposition}[Differentiabilitet for den karakteristiske funktion]
        \begin{enumerate}
            \item[\textnormal{(i)}] Lad $\mu$ være et sandsynligehdsmål på $(\R, \B(\R))$, og antag at $p\in\N_0$, således at $\int_\R|x|^p\mu(\deriv x)<\infty$.
            \\Da er $\hat{\mu}$ p-gange differentiabel med afledede
            $$\hat{\mu}^{(k)}(t)=\icomp^k\int_{\R}\mathsf{x}^ke^{\icomp\mathsf{x}t}\mu(\deriv x), \quad (t\in\R, k=0,1,\dots, p).$$
            \item[\textnormal{(ii)}] Lad $\X$ være en stokastisk variabel defineret på $\pfield$, og antag, at $p\in\N_0$, således at $\E[|\X|^p]<\infty$.
            \\Da er $\varx$ p-gange differentiabel med afledede:
            $$\varx^{(k)}(t)=\icomp\E[\X^ke^{\icomp t\X}], \quad (t\in\R, k=0,1,\dots, p).$$ 
            Specielt er 
            $$\E[\X^k]=\icomp^{-k}\varx^{(k)}(0), \quad (k=0,1,\dots, p).$$
        \end{enumerate}
    \end{proposition}
\end{theorem-box}
Bemærk at ii) følger af i)
\begin{theorem-box}
    \begin{lemma}
        For hvert $n\in\N_0$ defineres funktionen $r_n:\R\rightarrow \C$ ved $$r_n(t)=e^{\icomp t}-\sum_{k=0}^n\frac{\icomp^k t^k}{k!}, \quad (t\in\R).$$
        Da gælder vurderingen:
        $$|r_n(t)|\leq \frac{2|t|^n}{n!}\wedge \frac{|t|^{n+1}}{(n+1)!},\quad  (t\in\R).$$
    \end{lemma}
\end{theorem-box}
Ovenstående kan også skrives som 
$$|r_n(t)|\leq \min\{\frac{2|t|^n}{n!},\frac{|t|^{n+1}}{(n+1)!}\},\quad  (t\in\R).$$
\begin{theorem-box}
    \begin{corollary}
        Lax $\X$ være en stokastisk variabel på $\pfield$, således at $\E[X^2]<\infty$. For ethvert $\alpha$ i $[2,3]$ gælder da vurderingen:
        $$\left|\varx(t)-1-\icomp t\E[\X]+\frac{1}{2}t^2\E[\X^2]\right|\leq |t|^\alpha\E[|\X|^\alpha], \quad (t\in\R).$$
    \end{corollary}
\end{theorem-box}
\begin{theorem-box}
    \begin{corollary}
        Lad $\X$ være en stokastisk variabel på $\pfield$, og antag, at $\sigma^2:=\E[\X^2]<\infty$, samt at $\E[\X]=0.$
        
        Da gælder der, at 
        $$\frac{\varx(t)-1}{t^2}\longrightarrow -\frac{\sigma^2}{2}\quad \text{ for }t\rightarrow 0.$$
    \end{corollary}
\end{theorem-box}
\begin{theorem-box}
    \begin{proposition}[Taylor-udvikling af den karakteristiske funktion]
        Lad $\X$ være en stokastisk variabel på $\pfield$ med momenter af enhver orden. Antag yderligere at følgende betingelse er opfyldt: 
        \begin{align}\exists\rho\in(0,\infty):\lim_{n\rightarrow\infty}\frac{\rho^n\E[|\X|^n]}{n!}=0,\end{align}
        og vælg et $\rho$ i henhold hertil. For ethvert a i $\R$ gælder der da, at Taylor-rækken for $\varx$ i a er konvergent i $[a-\rho, a+\rho]$ med sum $\varx$, dsv.
        $$\varx(t)=\sum_{k=0}^{\infty}\frac{\varx^{(k)}(a)}{k!}(t-a)^k=\sum_{k=0}^{\infty}\frac{\icomp^k\E[\X^ke^{\icomp a\X}]}{k!}(t-a)^k, \quad (t\in[a-\rho, a+\rho]).$$ 
    \end{proposition}
\end{theorem-box}
\begin{remark}
    Betingelsen (1) er ækvivalent med følgende betingelse:
    $$\exists c\in(0,\infty)\forall n\in\N:\E[|\X|^n]\leq c^n n!$$
    Betingelsen er altså en begrænsning på hvor hurtigt momenterne må vokse med $n$.
\end{remark}
\subsection{Anvendelser af den karakteristiske funktion}
\begin{theorem-box}
    \begin{proposition}
        \begin{enumerate}
            \item[\textnormal{(i)}] Lad $\X$ være en symmetrisk stokastisk variabel, og lad videre $\X_1,\X_2,\X_3,\dots$ være i.i.d stokastiske variable, således at $\X_n\sim \X$ for alle n.
            
            Hvis yderligere $\X\sim\frac{\X_1+\cdots+\X_n}{\sqrt{n}}$ for alle n, da gælder der, at $\X\sim N(0,\sigma^2)$ for passende $\sigma$ i $[0,\infty)$
            \item[\textnormal{(ii)}] Lad $\X$ være en stokastisk variabel, og antag at $\sigma^2:=\E[\X^2]<\infty.$ Antag endvidere, at $\X\sim\frac{\X_1+\X_2}{\sqrt{2}}$, hvor $\X_1,\X_2$ er i.i.d, og $\X_1\sim \X$.

            Da gælder der, at $\X\sim N(0,\sigma^2)$
        \end{enumerate}
    \end{proposition}
\end{theorem-box}
For $\sigma=0$ tænker vi på $\X$ som dirac-målet.
\begin{theorem-box}
    \begin{lemma}
        Lad $(a_n)_n\in\N$ være en følge af komplekse tal, således at $a_n\longrightarrow a \in\C$ for $n\rightarrow \infty$. Da gælder der, at $$\lim_{n\rightarrow\infty}\left(1+\frac{a_n}{n}\right)^n=\exp(a),$$
        hvor $\exp(a)=e^{\operatorname{Re}(a)}(\cos(\operatorname*{Im}(a)))+\icomp\sin(\operatorname{Im}(a)).$
    \end{lemma}
\end{theorem-box}
\subsection{Momentproblemet}
\begin{theorem-box}
    \begin{problemstilling}[Momentproblemet]
        Lad $\X$ og $\Y$ være to stokastiske variable, og antag, at $\E[|\X|^p], \E[|\Y|^p]<\infty$ for alle p i $\N$, samt at
        $$\E[|\X|^p]=\E[|\Y|^p] \quad \text{for alle p i }\N$$
        Under hvilke yderligere betingelser kan man da slutte at $X\sim\Y$?
    \end{problemstilling}
\end{theorem-box}
\begin{theorem-box}
    \begin{proposition}
        Lad $\X$ og $\Y$ være to stokastiske variable, og antag, at $\E[|\X|^p], \E[|\Y|^p]<\infty$ for alle p i $\N$, samt at
        $$\E[|\X|^p]=\E[|\Y|^p] \quad \text{for alle p i }\N$$
        Hvis yderligere 
        $$\exists\rho\in(0,\infty):\E[e^{\rho|\X|}]<\infty,$$
        da gælder der, at $\X\sim\Y$
    \end{proposition}
\end{theorem-box}
Til beviset for Sætning 1.5.2 får vi brog for følgende lemma:
\begin{theorem-box}
    \begin{lemma}
        Lad $\X$ være en stokastisk variabel på $\pfield$. Da er følgende betingelser ækvivalente:
        \begin{enumerate}
            \item[\textnormal{(i)}] $\exists\rho\in(0,\infty):\E[e^{\rho|\X|}]<\infty.$
            \item[\textnormal{(ii)}] $\exists c\in(0,\infty)\forall n\in\N:\E[|\X|^n]\leq c^nn!.$ 
            \item[\textnormal{(iii)}] $\exists c\in(0,\infty)\forall n\in\N:\E[\X^{2n}]\leq c^{2n}(2n)!.$ 
        \end{enumerate}
    \end{lemma}
\end{theorem-box}
\begin{theorem-box}
    \begin{corollary}
        Lad $\X$ og $\Y$ være stokastiske variable på $\pfield$, og antag, at $P_\X$ og $P_\Y$ begge har kompakt støtte, dvs.
        $$\exists b<0:P(\X\in[-b,b])=1=P(\Y\in[-b,b])$$
        Da $\X$ og $\Y$ har momenter af enhver orden. Hvis yderligere $\E[\X^p]=\E[\Y^p]$ for alle p i $\N$, da gælder der, at $\X\sim\Y.$
    \end{corollary}
\end{theorem-box}
\begin{theorem-box}
    \begin{corollary}
        Lad $\X$ være en ikke-negativ stokastisk variabel, og betragt dens \textbf{Laplace transformerede}:
        $$L_\X(s)=\E[e^{-s\X}],\quad (s\in[0,\infty)).$$
        Da er $P_\X$ entydigt bestemet af $L_\X$. Med andre ord: Hvis $\Y$ er en anden ikke-negativ stokastisk variabel, således at $\L_\Y(s)=L_\X(s)$ for alle s i $[0,\infty)$, da gælder der, at $\X\sim\Y.$
    \end{corollary}
\end{theorem-box}
\section{Konvergens i mål og i sandsynlighed}
\subsection{De tre fundamentale konvergenstyper og deres indbyrdes styrkeforhold}
\begin{theorem-box}
    \begin{definition}
        Lad $(X,\mathcal{E}, \mu)$ være et målrum, og lad $(f_n)_{n\in\N}$ være en følge af funktioner fra $\mathcal{M}(\mathcal{E})$, og lad $f$ være endnu en funktion fra $\mathcal{M}(\mathcal{E}).$ Lad endvidere p være et (strengt) positivt tal. Vi siger da, at 
        \begin{enumerate}
            \item[\textnormal{(a)}] $f_n$ konvergerer mod f i $\mu-$mål for $n\rightarrow \infty$, hvis $$\forall\epsilon>0:\mu\left(\{x\in\X\big{|}|f_n(x)-f(x)|>\epsilon\}\right)\longrightarrow 0 \quad \text{for }n\rightarrow \infty.$$
            I så fald benyttes notationen: $f_n\rightarrow f$ i $\mu-$mål.
            \item[\textnormal{(b)}] $f_n$ konvergerer mod f $\mu-$n.o. for $n\rightarrow\infty$, hvis $$\mu\left(\{x\in X|\lim_{n\rightarrow \infty}f_n(x)=f(x)\}^c\right)=0.$$
            I så fald benyttes notationen: $f_n\rightarrow f \mu-$n.o.
            \item[\textnormal{(c)}] $f_n$ konvergerer mod $f$ i $\mu-p$ middel for $n\rightarrow \infty$, hvis 
            $$\int_X|f_n-f|^p\deriv\mu\longrightarrow 0, \quad \text{for }n\rightarrow \infty$$
            I så fald benyttes notationen: $f_n\rightarrow f$ i $\mu-p$-middel.
        \end{enumerate}
    \end{definition}
\end{theorem-box}
\begin{remark}
    Blandt andet linearitet bevarer konvergens.
\end{remark}
\begin{theorem-box}
    \begin{proposition}
        Lad $"\rightarrow"$ betegne én af de tre konvergensformer indført i Definition 2.1.1, og betragt funktioner $f,g,f_1,f_2,f_3,\dots$ fra $\mathcal{M}(\mathcal{E})$. Da gælder implikationen:
        $$f_n\longrightarrow f, \;\; og \;\; f_n\longrightarrow g \;\;\Longrightarrow\;\;f=g\;\; \mu-\text{n.o}$$
    \end{proposition}
\end{theorem-box}
\begin{theorem-box}
    \begin{proposition}
        Lad $(f_n)_{n\in\N}$ være en følge af funktioner fra $\mathcal{M}(\mathcal{E})$, og lad f være endnu en funktion fra $\mathcal{M}(\mathcal{E})$. Lad endvidere p være et positivt tal. Da gælder følgende udsagn:
        \begin{enumerate}
            \item[\textnormal{(i)}] Hvis $f_n\rightarrow f$ i $\mu-p$ middel, så gælder der også at $f_n\rightarrow f$ i $\mu-$mål.
            \item[\textnormal{(ii)}] Hvis $\sum_{n=1}^\infty\int_X|f_n-f|^p\deriv\mu<\infty$, så gælder der, at $f_n\rightarrow f \mu-n.o.$
            \item[\textnormal{(iii)}] Hvis $f_n\rightarrow f$ i $\mu-$mål, så findes en voksende følge $(n_k)_{k\in\N}$ af naturlige tal, således at $f_{n_k}\rightarrow f \mu$-n.o. for $k\rightarrow \infty$ 
        \end{enumerate}
    \end{proposition}
\end{theorem-box}
\begin{theorem-box}
    \begin{proposition}
        Antag, at $\mu$ er et \textbf{endeligt mål}, lad $(f_n)_{n\in\N}$ være en følge af funktioner fra $\mathcal{M}(\mathcal{E})$, og lad f være endnu en funktion fra $\mathcal{M}(\mathcal{E})$. Lad endvidere p,r være positive tal. Da gælder følgende udsagn:
        \begin{enumerate}
            \item[\textnormal{(i)}] Følgende betingelser er endbetydende:
            \begin{enumerate}
                \item[\textnormal{(i1)}] $f_n \rightarrow f \text{i }\mu-\text{mål}.$
                \item[\textnormal{(i2)}] $\forall K\in(0,\infty):\lim_{n\rightarrow\infty}\int_\X|f_n-f| \wedge K \deriv \mu = 0.$
                \item[\textnormal{(i3)}] $\lim_{n\rightarrow \infty}\int_{\X}|f_n-f|\wedge 1 \deriv \mu = 0.$ 
            \end{enumerate}
            \item[\textnormal{(ii)}] Hvis $f_n\rightarrow f$ $\mu-$n.o., så gælder der også, at $f_n\rightarrow f$ i $\mu-$mål.
            \item[\textnormal{(iii)}] Hvis $r<p$, og $f_n\rightarrow f$ i $\mu-$p-middel, da gælder der også at $f_n\rightarrow f$ i $\mu-$r-middel.
            \end{enumerate}
    \end{proposition}
\end{theorem-box}
\subsection{Fuldstændighed}
\begin{theorem-box}
    \begin{definition}
        Lad $(X,\mathcal{E},\mu)$ være et målrum, og lad $(f_n)_{n\in\N}$ være en følge af funktioner fra $\mathcal{M}(\mathcal{E})$. Lad endvidere p være et strengt positivt tal. Vi siger da, at 
        \begin{enumerate}
            \item[\textnormal{(a)}] $(f_n)_{n\in\N}$ er en \textbf{Cauchy-følge} i $\mu-$mål, hvis
            $$\forall\epsilon>0: \lim_{n,m\rightarrow \infty}\mu\left(\{|f_n-f_m|>\epsilon\}\right)=0.$$
            eller udskrevet hvis 
            $$\forall\epsilon,\delta>0\exists N\in\N\forall n,m\geq N: \mu\left(\{|f_n-f_m|>\epsilon\}\right)\leq \delta$$
            \item[\textnormal{(b)}] $(f_n)_{n\in\N}$ er en \textbf{Cauchy-følge} $\mu-$n.o., hvis $\mu(F^C)=0$, hvor
            $$F=\{x\in X|(f_n(x))_{n\in\N}\text{ er en Cauchy-følge i }\R\}$$
            \item[\textnormal{(c)}] $(f_n)_{n\in\N}$ er en \textbf{Cauchy-følge} i $\mu-p$-middel, hvis
            $$\lim_{n,m\rightarrow \infty}\int_\X f_n-f_m|^p\deriv \mu  = 0,$$
            eller udskrevet hvis 
            $$\forall \epsilon>0\exists N\in\N\forall n,m\geq N:\int_\X |f_n-f_m|^p\deriv\mu\leq\epsilon.$$
        \end{enumerate}
    \end{definition}
\end{theorem-box}
\begin{remark}
    Mængden $F$ er målelig - det følger af omskrivningen $$F=\bigcap_{K\in\N}\bigcup_{N\in\N}\bigcap_{n,m\geq N}\left\{x\in X\big{|}|f_n(x)-f_m(x)|\leq \frac{1}{K}\right\}.$$
\end{remark}
\begin{theorem-box}
    \begin{lemma}
        Lad $(X,\mathcal{E},\mu)$ være et målrum, og lad $(f_n)_{n\in\N}$ være en følge af funktioner fra $\mathcal{M}(\mathcal{E})$. Da gælder følgende udsagn:
        \begin{enumerate}
            \item[\textnormal{(i)}] Lad f være endnu en funktion fra $\mathcal{M}(\mathcal{E}),$ og antag, at der findes en følge $(\epsilon_n)_{n\in\N}$ af (strengt) positive tal, således at
            $$\lim_{n\rightarrow \infty}\epsilon_n = 0, \quad \text{og}\quad \sum_{n=1}^{\infty}\mu\left(\left\{|f_n-f|>\epsilon_n\right\}\right)<\infty.$$
            Da gælder der, at
            $$f_n\rightarrow\; \mu-\text{n.o.,}\quad \text{og}\quad f_n\rightarrow f \; \text{i }\mu\text{-mål.}$$
            \item[\textnormal{(ii)}] Antag, at der findes en følge $(\epsilon_n)_{n\in\N}$ af (strengt) positive tal, således at
            $$\sum_{n=1}^{\infty}\epsilon_n < \infty,\quad \text{og} \quad \sum_{n=1}^{\infty} \mu\left(\left\{|f_{n+1}-f_n|>\epsilon_n\right\}\right)<\infty $$
            Da findes der en funtkion f fra $\mathcal{M}(\mathcal{E})$, således at
            $$f_n \rightarrow f \;\mu\text{-n.o.,}\quad \text{og} \quad f_n\rightarrow f \; \text{i }\mu\text{-mål.}$$ 
        \end{enumerate}
    \end{lemma}
\end{theorem-box}
\begin{theorem-box}
    \begin{proposition}
        Lad $(X,\mathcal{E},\mu)$ være et målrum, og lad $(f_n)_{n\in\N}$ være en følge af funktioner fra $\mathcal{M}(\mathcal{E})$. Da er følgende betingelser ækvivalente:
        \begin{enumerate}
            \item[\textnormal{(i)}] Der findes en funktion f fra $\mathcal{M}(\mathcal{E})$, således at $f_n\rightarrow f$ i $\mu-$mål.
            \item[\textnormal{(ii)}] $(f_n)_{n\in\N}$ er en Cauchy-følge i $\mu-$mål.
        \end{enumerate}
        Med andre ord er konvergens i $\mu$-mål et fuldstændigt konvergensbegreb.
        \end{proposition}
\end{theorem-box}
\end{document}